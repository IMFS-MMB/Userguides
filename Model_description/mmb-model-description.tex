\documentclass[11pt,a4paper]{article}

\usepackage{tabulary}
\usepackage{tabularx}
\usepackage{booktabs}
\usepackage{array}
\usepackage[dvips]{graphicx}
\usepackage[longnamesfirst, round]{natbib}
%\usepackage{natbib}
%\biboptions{longnamesfirst,round,semicolon}
\usepackage{a4wide}
\usepackage[a4paper, dvips, mag=1000, truedimen, top=3cm, left=3cm, right=3cm, bottom=3.5cm]{geometry}
%\usepackage{mathptmx}
\usepackage{amssymb}
\usepackage{amsmath}
\usepackage{times}
%\usepackage[sidewaysfigure]{rotating}
\usepackage{float}
%\usepackage[latin1]{inputenc}        %f{\"u}r deutsche Texte
\usepackage[T1]{fontenc}             %f{\"u}r deutsche Texte
\usepackage{srcltx}
\usepackage{hyperref}

\graphicspath{../figures/}
%\setlength{\parindent}{0pt} %keine Einrueckung erster Absatz
%\sloppy

\begin{document}
	\begin{center}
		{\Large \textbf{A detailed overview of available models} }
		\par\end{center}
	
	%\fontsize{11}{18pt}\selectfont
	
	%\section{A detailed overview of available models}
	
	\begin{center}
		\textsc{\textbf{A List of Models Available in the Macroeconomic Model Data Base} } \\
		\textsc{\textbf{(version 3.3)} }
	\end{center}
	\vfill
	%\noindent \footnotesize $^\ast$ In total there are 114 models including all model variations such as adaptive learning versions. The Macroeconomic Model Data Base features 99 distinct models. 
	
	\vspace{-2cm}
	\begin{table}[H]
		%\caption{\textsc{Models Currently Available in the Data Base}}
		%\vspace{.2cm}
		\label{tab:modov}
		\begin{tabularx}{\textwidth}{lll}
			\hline \hline
			&& \\
			\multicolumn{3}{l}{\textsc{1. Calibrated Models}} \\
			%&& \\
			\ref{NKAFL15} & NK\_AFL15 $ ^{\ast}$  & \cite{Angelonietal2015} \\
			%&& \\
			\ref{NKBGEU10} & NK\_BGEU10 & \cite{BlanchardGali10} Calibrated for the European labor market \\
			%&& \\
			& NK\_BGUS10 & \cite{BlanchardGali10} Calibrated for the U.S. labor market\\
			%&& \\
			\ref{NKBGG99} & NK\_BGG99 & \cite{BernankeGertlerGilchrist1999} \\
			%&& \\
			\ref{NKCDK24} & NK\_CDK24 & \cite{chan2024energy}  \\
			%&& \\
			\ref{NKCFP10} & NK\_CFP10 & \cite{carlstrom2010optimal}  \\
			%&& \\
			\ref{NKCGG99} & NK\_CGG99 & \cite{ClaridaGaliGertler1999}  \\
			%&& \\
			\ref{NKCGG02}  & NK\_CGG02 & \cite{ClaridaGaliGertler2002}  \\
			%&& \\
			\ref{NKCK08} & NK\_CK08 & \cite{ChristoffelKuester2008} \\
			%&& \\
			\ref{NKCKL09} & NK\_CKL09 & \cite{ChristoffelKuesterLinzert2009} \\
			%&& \\
			\ref{NKCW09} & NK\_CW09 & \cite{CurdiaWoodford2009} \\
			%&& \\
			\ref{NKDEFK17} & NK\_DEFK17 & \cite{delnegro2017eggertson} \\
			%&& \\
			\ref{NKDT12} & NK\_DT12 & \cite{defiore2012tristani} \\
			%&& \\
			\ref{NKET14} & NK\_ET14 & \cite{EllisonTischbirek2014} \\
			%&& \\
			\ref{NKFLMF18} & NK\_FLMF18 & \cite{filardo2018monetary} \\
			%&& \\
			\ref{NKFNL23} & NK\_FNL23 & \cite{ferraria2023toward} \\
			%&& \\
			\ref{NKGHP16} & NK\_GHP16 & \cite{gnocci2016housework} \\
			%&& \\
			\ref{NKGK11} & NK\_GK11 & \cite{GertlerKaradi2011} \\
			%&& \\
			& NK\_GK09lin & linear model based on the working paper of \cite{GertlerKaradi2011} \\
			%&& \\
			\ref{NKGK13} & NK\_GK13 & \cite{GertlerKaradi2013} \\
			%&& \\
			\ref{NKGLSV07} & NK\_GLSV07 & \cite{gali2007understanding} \\
			%&& \\
			\ref{NKGM05} & NK\_GM05 & \cite{GaliMonacelli2005} \\
			%&& \\
			\ref{NKGM07} & NK\_GM07 & \cite{goodfriend2007banking} \\
			%&& \\
			\ref{NKGM16} & NK\_GM16 & \cite{gali2016understanding} \\
			%&& \\
			\ref{NKGS14} & NK\_GS14 & \cite{gambacorta2014should} \\
			%&& \\
			\ref{NKGSSZ17} & NK\_GSSZ17 & \cite{gilchrist2017inflation} \\
			%&& \\
			\ref{NKIR04} & NK\_IR04 & \cite{Ireland2004} \\
			%&& \\
			\ref{NKJO15} & NK\_JO15ht & \cite{jang2015okano} - high trading\\
			%&& \\
			&  NK\_JO15lt & \cite{jang2015okano} - low trading \\
			%&& \\
			\ref{NKKM16} & NK\_KM16 & \cite{krause2016public} \\
			%&& \\
			\ref{NKKRS12} & NK\_KRS12 & \cite{KannanRabanalScott2012} \\
			%&& \\
			\ref{NKKW16} & NK\_KW16 & \cite{kirchner2016fiscal} \\
			%&& \\
			\ref{NKLWW03} & NK\_LWW03 & \cite{LevinWielandWilliams2003}   \\
			%&& \\
			\ref{NKMCN99cr} & NK\_MCN99cr & \cite{McCallumNelson1999}, (Calvo-Rotemberg model) \\
			%&& \\
			\ref{NKMI14} & NK\_MI14 & \cite{michaillat2014atheory} \\
			%&& \\
			\ref{NKMM10} & NK\_MM10 & \cite{MehMoran2010} \\
			%&& \\
			\ref{NKMPT10} & NK\_MPT10 & \cite{monacelli2010unemployment} \\
			%&& \\
			\ref{NKNS14} & NK\_NS14 &  \cite{NakamuraSteinsson2014} \\
			%&& \\
			\ref{NKPP17} & NK\_PP17 &  \cite{paoli2017coordinating} \\
			%&& \\
			\ref{NKPSV16} & NK\_PSV16 &  \cite{pancrazi2016price} \\
			%&& \\
			\ref{NKRA16} & NK\_RA16 &  \cite{rannenberg2016bank} \\
			%&& \\
			\ref{NKRW06} & NK\_RW06 & \cite{RavennaWalsh2006} \\
			%&& \\
			\ref{NKRW9} & NK\_RW97  & \cite{RotembergWoodford1997} \\
			%&& \\
			\ref{NKST13} & NK\_ST13  & \cite{stracca2013inside} \\
			%&& \\
			\ref{RBCDTT11} & RBC\_DTT11 & \cite{DeFioreetal2011} \\
			&& \\

			%&& \\
			%		&& \\
			\hline \hline
			
			%\multicolumn{3}{l}{\footnotesize \noindent $ ^{\ast \ast}$ Solving these models requires the MATLAB Statistics Toolbox. }
		\end{tabularx}
		\footnotesize \noindent $ ^{\ast}$ Solving this model requires the MATLAB Optimization Toolbox.
		
	\end{table}
	
	
	
	
	
	\vspace{-1cm}
	
	\begin{table}[H]
		%\caption{\textsc{Models Currently Available in the Data Base (May 2009)}}
		%\vspace{.2cm}
		%\label{tab:modov}
		\begin{tabularx}{\textwidth}{lll}
			\hline \hline
			&& \\
			\multicolumn{3}{l}{\textsc{2. Estimated US Models}} \\
			%&& \\
			\ref{USACELm} & US\_ACELm & \cite{AltigChristianoEichenbaumLinde2005}, (monetary policy shock)  \\
			%&& \\
			& US\_ACELswm & no cost channel as in \cite{TaylorWieland2011} (mon. pol. shock) \\
			%				&& \\
			& US\_ACELswt & no cost channel as in \cite{TaylorWieland2011} (tech. shocks) \\
			%&& \\
			& US\_ACELt & \cite{AltigChristianoEichenbaumLinde2005}, (technology shocks) \\
			%&& \\
			
			
			\ref{USAJ16} & US\_AJ16 & \cite{ajello2016financial}\\
			\ref{USBB18} & US\_BB18 & \cite{balke2018oil}\\
			%&& \\
			\ref{USBKM12} & US\_BKM12 & \cite{bils2012reset}\\
			%&& \\
			\ref{USCCF12} & US\_CCF12 & \cite{chen2012macroeconomic}\\
			%&& \\		
			\ref{USCCTW10} & US\_CCTW10 & \cite{SmetsWouters2007} model with rule-of-thumb consumers,\\
			%&& \\
			&& estimated by \cite{CoganCwikTaylorWieland2010}\\
			%&& \\
			\ref{USCD08} & US\_CD08 & \cite{ChristensenDib2008} \\
			%&& \\
			\ref{USCET15} & US\_CET15 & \cite{christiano2015eichenbaum}\\
			%&& \\
			\ref{USCFOP14} & US\_CFOP14 & \cite{Carlstrometal2014}\\ 	
			%&& \\
			\ref{USCFP17} & US\_CFP17exo & \cite{carlstrom2017targeting} - exogenous level of long-term debt\\
			%&& \\	
			& US\_CFP17endo & \cite{carlstrom2017targeting} - endogenous level of long-term debt\\
			%&& \\			
			\ref{USCMR10} & US\_CMR10 & \cite{Christianoetal2010} \\
			& US\_CMR10fa & \cite{Christianoetal2010} - small version with financial accelerator \\
			\ref{USCMR14} & US\_CMR14 $ ^{\ast \ast}$ & \cite{CMR2014}      \\  
			& US\_CMR14noFA $ ^{\ast \ast}$ & \cite{CMR2014}-Version without financial frictions        \\   
			\ref{USCPS10} & US\_CPS10 & \cite{Cogleyetal2010} \\
			
			\ref{USDG08} & US\_DG08 & \cite{DeGraeve2008} \\
			\ref{USDNGS15} & US\_DNGS15 & \cite{del2015inflation} \\
			& US\_DNGS15\_SW & \cite{del2015inflation} w/o financial frictions\\
			& US\_DNGS15\_SWpi & \cite{del2015inflation} w/o financial frictions and time-varying inflation target \\
			& US\_DNGS15\_SWSP & \cite{del2015inflation} reestimation of \cite{SmetsWouters2007} \\
			& & with longer time-series \\
			
			\ref{USFGKR15} & US\_FGKR15 & \cite{FernandezVillaverdeetal2015} \\
			\ref{USFM95} & US\_FM95 & \cite{FuhrerMoore1995}  \\
			\ref{USFMS13} & US\_FMS13 & \cite{Feveetal2013} \\
			\ref{USFRB03} & US\_FRB03 & Federal Reserve Board model linearized as in \cite{LevinWielandWilliams2003} \\
			\ref{USFRB08} & US\_FRB08 & linearized by \cite{BraytonLaubach2008} \\
			& US\_FRB08mx & linearized by \cite{BraytonLaubach2008}, (mixed expectations) \\
			\ref{USFRB22}& US\_FRB22\_mceall & \cite{brayton2022linver}: all expectations are model consistent\\
			& US\_FRB22\_mcapwp & \cite{brayton2022linver}: financial market, wage and price \\&&expectations are model consistent, other expectations are based on a small VAR\\
			& US\_FRB22\_mcap & \cite{brayton2022linver}: financial market expectations are model\\&& consistent, other expectations are based on a small VAR\\
			& US\_FRB22\_var & \cite{brayton2022linver}: all expectations are based on VAR predictions\\
			\ref{USFU19} & US\_FU19 & \cite{fratto2019uhlig} \\
			\ref{USFV10} & US\_FV10 & \cite{fernandez2010econometrics}\\
			\ref{USFV15} & US\_FV15 & \cite{fernandez2015estimating}\\
			\ref{USHL16} & US\_HL16 & \cite{hollander2016liu}\\
			\ref{USIAC05} & US\_IAC05 & \cite{Iacoviello2005} \\
			\ref{USIN10} & US\_IN10 & \cite{IacovielloNeri2010} \\
			\ref{USIR11} & US\_IR11 & \cite{Ireland2011} \\
			
			
			&& \\
			\hline \hline
		\end{tabularx} 
		\footnotesize \noindent $ ^{\ast \ast}$ Solving these models requires the Statistics Toolbox for MATLAB or the statistics and io package for Octave, respectively. %$^{\ast}$ Currently only in the DYNARE 3 version. Those models are excluded from counting the number of models available in the MMB 2.3.
		
	\end{table}
	
	
	
	\vspace{-1cm}
	
	\begin{table}[H]
		%\caption{\textsc{Models Currently Available in the Data Base (May 2009)}}
		%\vspace{.2cm}
		%\label{tab:modov}
		\begin{tabularx}{\textwidth}{lll}
			\hline \hline
			&& \\
			\multicolumn{3}{l}{\textsc{2. Estimated US Models (continued)}} \\
			%&& \\
			\ref{USIR15} & US\_IR15 & \cite{ireland2015monetary}\\
			\ref{USJPT11} & US\_JPT11 & \cite{Justinianoetal2011}\\ 	
			\ref{USKK14} & US\_KK14 & \cite{kliem2014kriwoluzky} \\
			\ref{USKS15} & US\_KS15 & \cite{kriwoluzky2015stoltenberg} \\		
			\ref{USLTW17} & US\_LTW17 & \cite{leeper2017traum} \\
			& US\_LTW17gz &	\cite{leeper2017traum} - different fiscal rule\\
			& US\_LTW17nu & \cite{leeper2017traum} - no government consumption in the utility function\\
			& US\_LTW17rot & \cite{leeper2017traum} - with rule of thumb consumers \\
			\ref{USLWY13} & US\_LWY13 & \cite{leeper2013fiscal}\\
			\ref{USMI07} & US\_MI07 & \cite{Milani2007} \\
			\ref{USMR07} & US\_MR07 & \cite{MankiwReis2007} \\
			\ref{USOR03} & US\_OR03 & \cite{Orphanides2003} \\
			\ref{USOW98} & US\_OW98 & \cite{OrphanidesWieland1998} equivalent to MSR model in \\
			& & \cite{LevinWielandWilliams2003} \\
			
			\ref{USPM08} & US\_PM08 & IMF projection model US, \cite{Carabenciovetal2008} \\
			& US\_PM08fl & IMF projection model US (financial linkages),\cite{Carabenciovetal2008}  \\
			\ref{USPV15} & US\_PV15 & \cite{poutineau2015financial} \\		
			\ref{USRA07} & US\_RA07 & \cite{Rabanal2007} \\
			\ref{USRE09} & US\_RE09 & \cite{reis2009sticky}\\
			\ref{USRS99} & US\_RS99 & \cite{RudebuschSvensson1999} \\
			\ref{USSW07} & US\_SW07 & \cite{SmetsWouters2007} \\
			\ref{USVI16} & US\_VI16bgg & \cite{villa2016}  - with \cite{BernankeGertlerGilchrist1999} financial accelerator \\
			& US\_VI16gk & \cite{villa2016} - with \cite{GertlerKaradi2013} financial friction \\
			\ref{USVMDno} & US\_VMDno & Verona, Martins and Drumond (\cite{Veronaetal2013}) - Normal times \\
			& US\_VMDop & Verona, Martins and Drumond (\cite{Veronaetal2013}) - Optimistic times \\
			%\ref{USNFED08}  & US\_NFED08 $^{\ast}$ & based on \cite{EdgeKileyLaforte2007}, version used for estimation in \\
			%& &  \cite{WielandWolters2011} \\
			\ref{USYR13} & US\_YR13 & \cite{rychalovska2016} \\ 
			
			&& \\
			\multicolumn{3}{l}{\textsc{3. Estimated Euro Area Models }} \\
			%&& \\
			\ref{EAALSV06} & EA\_ALSV06 & \cite{andres2006lopezsalido} \\
			\ref{EAAWM05} & EA\_AWM05 &  ECB's area-wide model linearized as in \cite{DieppeKuesterMcAdam2005}\\
			%and \cite{KuesterWieland2005} \\
			\ref{EABE15} & EA\_BE15 & \cite{benchimol2015money} \\
			\ref{EABF17} & EA\_BF17 & \cite{benchimol2017money} \\
			\ref{EACKL09} & EA\_CKL09 & \cite{ChristoffelKuesterLinzert2009} \\
			%&& \\
			\ref{EACW05ta} & EA\_CW05ta & \cite{CoenenWieland2005}, (Taylor-staggered contracts) \\
			& EA\_CW05fm & \cite{CoenenWieland2005}, (Fuhrer-Moore-staggered contracts) \\
			\ref{EADKR11} & EA\_DKR11 &  \cite{DarracqPariesetal2011} \\
			\ref{EAGE10} & EA\_GE10 & \cite{Gelain2010} \\
			\ref{EAGNSS10} & EA\_GNSS10 & \cite{Geralietal2010} \\
			\ref{EAPV15} & EA\_PV15 & \cite{poutineau2015cross} \\
			\ref{EAPV16} & EA\_PV16 & \cite{priftis2016portfolio} \\
			\ref{EAPV17} & EA\_PV17 & \cite{priftis2017macroecon} \\
			\ref{EAQR14} & EA\_QR14 $ ^{\ast \ast}$ & \cite{QR2014}  \\
			\ref{EAQUEST3} & EA\_QUEST3 & QUEST III Euro Area Model of the DG-ECFIN EU, \cite{RattoRoegerVeld2009} \\
			\ref{EASR07} & EA\_SR07 & Sveriges Riksbank euro area model of \cite{AdolfsonLaseenLindeVillani2007}\\
			\ref{EASW03} & EA\_SW03 & \cite{SmetsWouters2003} \\
			
			
			\ref{EASWW14} & EA\_SWW14 & \cite{smets2014warne} \\

			%5.4 & CA\_ToTEM10$^{\ast}$ & ToTEM model of Canada, based on \cite{MurchisonRennison2006}, \\
			% && 2010 vintage \\
			
			
			%\multicolumn{3}{l}{\textsc{6. Estimated Models with Adaptive Learning}} \\
			% \ref{USMI07AL} & US\_MI07AL & Milani (2007), Adaptive Learning version \\
			%&& \\
			\hline \hline
			\vspace{-0.2cm}
			
		\end{tabularx} 
		
		\footnotesize \noindent  %$^{\ast }$ Currently only in the DYNARE 3 version. Those models are excluded from counting the number of models available in the MMB 2.3. 
		$ ^{\ast \ast}$ Solving these models requires the Statistics Toolbox for MATLAB or the statistics and io package for Octave, respectively. \\
		%Most models assume that expectations of future realizations of model variables such as for example future exchange rates, prices, interest rates, wages and income are formed in a model-consistent, rational manner.  A few models assume backward-looking expectations formation, in particular the models from \cite{RudebuschSvensson1999} and Orphanides (2003).
		%Most, but not all models are linear, or linear approximations of nonlinear models. In this case the variables appear as percentage deviations from their steady state values.  There are many differences in model structure, in terms of size, in terms of countries covered, or the extent of microeconomic foundations considered. \\
		
		
	\end{table}
	%A description of the structure and the most important features of the different models in the macro model data base.
	
	%We illustrate the implications of the model features by plotting impulse response functions to a
	%one unit monetary policy shock. We use the simple rule by \cite{Taylor1993} as it lacks interest
	%rate smoothing and therefore most clearly shows the differences in the persistence of the different
	%models. In addition we show impulse responses using the more realistic interest rule
	%by \cite{LevinWielandWilliams2003} which has a high degree of interest rate smoothing.
	%We also show impulse responses to fiscal policy shocks or more general demand shocks if the
	%model does not include a government sector.
	%Most models assume that expectations of future realizations of model variables such as for example future exchange rates, prices, interest rates, wages and income are formed in a model-consistent, rational manner.  A few models assume backward-looking expectations formation, in particular the models from Rudebusch and Svensson (1999) and Orphanides (2003).
	%The FRB-US model Only one
	%version of the FRB model features expectations that are formed using a VAR as described %below.
	%Most, but not all models are linear, or linear approximations of nonlinear models. In this case the variables appear as percentage deviations from their steady state values.  There are many differences in model structure, in terms of size, in terms of countries covered, or the extent of microeconomic foundations considered. \\
	%The models have different kinds of price rigidities like the one proposed
	%by \cite{Taylor1980}, \cite{FuhrerMoore1995} or \cite{Calvo1983}. The models range from closed
	%economy three equation models to disaggregated multi-country models.\\
	
	
	\vspace{-1cm}
	
	\begin{table}[H]
		%\caption{\textsc{Models Currently Available in the Data Base (May 2009)}}
		%\vspace{.2cm}
		%\label{tab:modov}
		\begin{tabularx}{\textwidth}{lll}
			\hline \hline
			&& \\
			\multicolumn{3}{l}{\textsc{3. Estimated Euro Area Models (continued)}} \\
			\ref{EAVI16} & EA\_VI16bgg & \cite{villa2016} - with \cite{BernankeGertlerGilchrist1999} financial accelerator \\
			& EA\_VI16gk & \cite{villa2016} - with \cite{GertlerKaradi2013} financial friction \\
			&&\\
			\multicolumn{3}{l}{\textsc{4. Estimated/Calibrated Multi-Country Models}} \\
			%&& \\
			\ref{DEREAGEAR16}& DEREA\_GEAR16 & \cite{gadatsch2016fiscal} model of Germany, EMU, and RoW\\
			\ref{ESREAFIMOD12} & ESREA\_FIMOD12 & \cite{stahler2012fimod} model of Spain and EMU\\
			\ref{G2SIGMA08} & G2\_SIGMA08 & The Federal Reserve's SIGMA model from \cite{ErcegGuerrieriGust2008}\\
			&& calibrated to the U.S. economy and a symmetric twin.\\
			\ref{G3CW03} & G3\_CW03 & \cite{CoenenWieland2002} model of USA, Euro Area and Japan \\
			%&& \\
			\ref{G7TAY93} & G7\_TAY93 & \cite{Taylor1993a} model of G7 economies \\
			%&& \\
			\ref{GPM6IMF13} & GPM6\_IMF13 & IMF global projection model with 6 regions \cite{Carabenciovetal2013}\\
			
			\ref{EACZGEM03} & EACZ\_GEM03 & \cite{LaxtonPesenti2003} model calibrated to Euro Area and Czech republic\\
			%&& \\
			\ref{EAESRA09} & EAES\_RA09 & \cite{Rabanal2009}\\
			%&& \\
			\ref{EAUSNAWM08} & EAUS\_NAWM08 & \cite{CoenenMcAdamStraub2008}, New Area Wide model of Euro Area and USA \\
			%&& \\
			\ref{EAUSNAWM08CTWW13} & EAUS\_NAWMctww & \cite{CoganTaylorWielandWolters2013}\\
			
			&& \\
			\multicolumn{3}{l}{\textsc{5. Estimated Models of other Countries }} \\
			%&& \\
			\ref{BRASAMBA08} & BRA\_SAMBA08 & \cite{Gouveaetal2008}, model of the Brazilian economy \\
			\ref{CABMZ12} & CA\_BMZ12 & \cite{Bailliuetal2012}  \\
			%&& \\
			\ref{CALS07} & CA\_LS07 & \cite{LubikSchorfheide2007},\\
			&& small-scale open-economy model of the Canadian economy \\
			%\ref{CAToTEM10} & CA\_ToTEM10 $ ^{\ast \ast}$  &\cite{MurchisonRennison2006} \\
			\ref{CATOTEM10} & CA\_TOTEM10 & \cite{murchison2006rennison}\\
			\ref{CLMS07} & CL\_MS07 & \cite{MedinaSoto2007}, model of the Chilean economy \\
			\ref{FIAINO16} & FI\_AINO16 & \cite{kilponen2016aino},  the AINO II model \\
			%&& \\
			\ref{HKFPP11} & HK\_FPP11 & \cite{FunkePaetzPytlarczyk2011}, open-economy model of the Hong Kong economy \\
			\ref{HKFP13} & HK\_FP13 & \cite{FunkePaetz2013}, open-economy model of the Hong Kong economy \\
			\ref{UKSM11}&UK\_SM11 & \cite{millard2011estimated}, open-economy model of the United Kingdom with energy\\
			%\ref{MP17} & ESP\_MP17 & \cite{martin2017philippon}, model of the Spanish economy \\
			%	& GRC\_MP17 & \cite{martin2017philippon}, model of the Greek economy \\
			%	& IRL\_MP17 & \cite{martin2017philippon}, model of the Irish economy \\
			%	& PRT\_MP17 & \cite{martin2017philippon}, model of the Portuguese economy \\
			&& \\
			\hline \hline
			\vspace{-0.2cm}
		\end{tabularx} 
	\end{table}
	
	\footnotesize \noindent Most models assume that expectations of future realizations of model variables such as for example future exchange rates, prices, interest rates, wages and income are formed in a model-consistent, rational manner.  A few models assume backward-looking expectations formation, in particular the models from \cite{RudebuschSvensson1999} and \cite{Orphanides2003}.
	Most, but not all models are linear, or linear approximations of nonlinear models. In this case the variables appear as percentage deviations from their steady state values.  There are many differences in model structure, in terms of size, in terms of countries covered, or the extent of microeconomic foundations considered. \\
	
	
	
	
	
	\newpage
	
	\section{Calibrated Models}
	
	
	\subsection{NK\_AFL15:\texorpdfstring{\cite{Angelonietal2015}}{Angeloni et al. (2015)}}
	\label{NKAFL15}
	\cite{Angelonietal2015} propose a DSGE model where banks are subject to runs, modelled as a discipline device in the spirit of \cite{diamond2000theory,diamond2001liquidity}. Banks invest in risky projects and fund operations via bank equity and run-prone deposits, such that their funding structure endogenously determines bank fragility. In the model, expansionary monetary policy (either following a monetary policy shock or responding to other exogenous shocks) increases bank leverage and risk. An accelerating balance sheet channel is offset by banks' risk taking such that business cycle fluctuations are dampened relative to a model without a financial sector.
	\begin{itemize}
		\item Aggregate Demand: The representative household derives utility from consumption and disutility from labor. It saves and invests in bank demand deposits and bank capital. Households also own the production sector and receive lump-sum nominal profits. The resulting optimality conditions are a standard intra-temporal labor-consumption trade-off and an Euler equation for savings in the form of bank deposits. In the latter, the return on demand deposits is subject to a time-varying risk premium due to the possibility of runs. 
		\item Aggregate Supply: The firm side is characterized by monopolistic competition and nominal rigidities in the form of \cite{Rotemberg1982} quadratic price adjustment costs. This gives rise to a standard \cite{Rotemberg1982} forward-looking New Keynesian Phillips curve. Capital investment is undertaken by capital producers facing capital adjustment costs. 
		\item  Financial Sector: Firms finance investment fully via bank lending. A fraction of household members are bank capitalists, such that households own financial intermediaries. Bank managers choose banks' funding structure on behalf of depositors and bank capitalists subject to a moral hazard problem resulting from their superior project knowledge. Bank managers are incentivized by the contractual payoff structure and the threat that depositors might run the bank, which entails costly project liquidation. The latter happens if the idiosyncratic bank project returns are too low for all depositors to be reimbursed, a risk that increases with endogenously chosen bank leverage. 
		\item Shocks: A productivity shock, a monetary policy shock and a government shock. 
		\item Calibration/Estimation: The model is calibrated at quarterly frequency. Parameter values for the household and firm sector are standard in line with existing literature. Bank parameter values are calibrated to match US data. All exogenous shocks follow an AR(1) process with persistence parameters and standard deviations as implied by the empirical analysis in Section 3 or other empirical results.
	\end{itemize}
	
	
	\subsection{NK\_BGEU10 and NK\_BGUS10: \texorpdfstring{\cite{BlanchardGali10}}{Blanchard and Galí (2010)}}
	\label{NKBGEU10}
	%\label{NKBGUS10}
	\cite{BlanchardGali10} derive a small-scale New Keynesian Model with
	labor market frictions to analyze optimal monetary policy under different
	labor market conditions. They consider two different calibrations of the
	model. One specification for the European labor market which is assumed to be sclerotic, and another one for the U.S. labor market considered as more fluid.  
	
	\begin{itemize}
		\item Aggregate Demand: Households maximize their lifetime utility, where
		the utility function is separable in consumption and leisure, subject to
		an intertemporal budget constraint. Households supply one homogeneous type of labor, and face an exogenous separation rate.
		\item Aggregate Supply: In this economy, there are two types of firms. Competitive firms produce intermediate goods using labor services. Retail firms have monopolistic power and re-package intermediate output.
		\item Rigidities: Nominal rigidities are modeled by standard \cite{Calvo1983}
		pricing. Real wage is a non-linear function of productivity, introducing
		real wage rigidities.
		\item Shocks: An adverse technology shock, and the common monetary policy shock.
		\item Calibrations: For the fluid US labor market, shorthand named NK\_BGUS10 above, unemployment rate is equal to 5 percent, and the job finding rate is set to a monthly rate of 0.3, consistent with
		US evidence. For the sclerotic European labor market, labeled NK\_BGEU10 above, Unemployment rate
		is equal to 10 percent, and the job finding rate is set to a monthly rate of 0.1. 
	\end{itemize}
	
	
	
	\subsection{NK\_BGG99: \texorpdfstring{\cite{BernankeGertlerGilchrist1999}}{Beranke et al. (1999)}}
	\label{NKBGG99}
	\cite{BernankeGertlerGilchrist1999} introduce credit market imperfections into an otherwise standard New Keynesian model with capital and show that these financial frictions contribute to propagate and amplify the response of key macroeconomic variables to nominal and real shocks. An agency problem arises due to asymmetries of information in borrower-lender relationships. The economy is inhabited by three types of agents, risk-averse households, risk-neutral entrepreneurs and retail firms.
	
	\begin{itemize}
		
		\item Aggregate Demand: Households gain utility from consumption, leisure and real money balances. Household optimization results in a standard dynamic IS equation. Entrepreneurs use capital and labor to produce wholesale goods that are sold to the retail sector. Each period, entrepreneurs have to accumulate capital that becomes available for production in the subsequent period. Entrepreneurs have to borrow from households via a financial intermediary to finance capital purchases. Since the financial intermediary has to pay some auditing costs to observe the idiosyncratic return to capital, an agency problem arises. The optimal contract leads to an aggregate relationship of the spread between the external finance costs and the risk-free rate and entrepreneurs' financial conditions represented by the leverage ratio.
		
		\item Aggregate Supply: Retail firms act under monopolistic competition. They buy wholesale goods produced by entrepreneurs in a competitive market and differentiate them at zero cost. Price stickiness is introduced via the Calvo framework. \cite{BernankeGertlerGilchrist1999} assume that reoptimizing firms have to set prices prior to the realization of shocks in that period, so that previous period's expectations of the output gap and future inflation enter the New Keynesian Phillips curve.
		
		\item Shocks: The model exhibits a technology shock, a demand shock and the common monetary policy shock. Since we have no information about the variances of the shock terms, we set all shock variances equal to zero.
		
		\item Calibration/Estimation: The model is calibrated at quarterly frequency.
		
		%\item Replication: Check the model in the Modelbase!
		
	\end{itemize}
	
	
	\subsection{NK\_CDK24: \texorpdfstring{\cite{chan2024energy}}{Chan et al. (2024)}}
	\label{NKCDK24}
	
	\cite{chan2024energy} develop a medium-scale open economy two-agent New Keynesian (TANK) model and calibrate it for the U.K.
	
 	\begin{itemize}
 	\item Aggregate Demand: Households maximize their lifetime utility, using a CES function of	consumption and energy, subject to an inter-temporal budget constraint. A portion of households are financially unconstrained. They consume and supply labor to unions, save in domestic and foreign nominal risk-free bonds, and receive profits from firm ownership. The remaining fraction of households are hand-to-mouth, implying they only consume their labor income.
	
	\item Aggregate Supply: The single-differentiated final goods are produced by a continuum of monopolistic competitive firms, which use imported energy and labor following a CES function. They decide on labor and capital inputs and set prices according to the Calvo model. Labor is differentiated by a union so that there is some monopoly power over wages, which results in an explicit wage equation. Labor packers buy the labor from the unions and resell it to the final goods producer in a perfectly competitive environment. Sticky wages according to Calvo are additionally assumed. The Calvo model in both wage and price setting is augmented by the assumption that prices that can not be freely set, are partially indexed to past inflation rates. The final goods packers maximize profits subject to an aggregator of final goods, which introduces monopolistic competition in the market for final goods and features a non-constant elasticity of substitution between different final goods, which depends on their relative price.
	
	\item Shocks: The model features 4 shocks, i.e. TFP shock, global energy price shock, price markup shock and monetary policy shock.
	
	\item Calibration: The model is calibrated for the U.K.. However, the majority of parameters is very close to the literature.
	
	\item Replication: We replicated the impulse response functions to a 100\% increase in the global energy prices and a negative shock to TFP.
	\end{itemize}
	
	\subsection{NK\_CFP10: \texorpdfstring{\cite{carlstrom2010optimal}}{Carlstrom et al. (2010)}}
	\label{NKCFP10}
	\cite{carlstrom2010optimal} build a small-scale calibrated New Keynesian DSGE model with agency costs, which are modelled as constraint on the firm's hiring of labour as in the holdup problem of \cite{KiyotakiMoore1999}. 
	\begin{itemize}
		\item Aggregate demand: Households maximize their lifetime utility, where the per-period utility function is separable in consumption and two types of labour. They can buy standard one-period bonds and firm shares, with the latter paying of dividends. 
		\item Aggregate Supply: Entrepreneurs have linear consumption preferences and operate the intermediate good firms. These firms combine both types of labour into the intermediate good using a Cobb-Douglas production function. Due to a hold-up problem, entrepreneurs face a collateral constraint on their hiring of one labour input, in that the wage bill cannot exceed  a Cobb-Douglas combination of net worth and profits. This introduces a credit friction. Monopolistically competitive final goods firms purchase intermediate goods from entrepreneurs and create final goods using a linear production function. Final goods pricing is subject to Rotemberg quadratic adjustment costs. The final goods are aggregated to an output bundle according to a CES function. 
		\item Shocks: A productivity shock, a mark-up shock, a net worth shock and a monetary policy shock. 
		\item Calibration/Estimation: The model is calibrated using standard values in the literature, in particular following \cite{Woodford2003}.  Credit-related parameters are calibrated using the average spread between BB+ and 10-year Treasury bonds from 1996 to present. 
		\item Replication: We simulated the impulse response functions to a monetary policy shock and a technology shock under a simple Taylor rule, Figure 1 and Figure 2 in the paper. 
		
	\end{itemize}
	
	\subsection{NK\_CGG99: \texorpdfstring{\cite{ClaridaGaliGertler1999}}{Clarida et al. (1999)}, hybrid model}
	\label{NKCGG99}
	The model is similar to NK\_RW97 but it features a hybrid
	Phillips curve with endogenous persistence in inflation. Also, government spending is
	not treated explicitly. The
	model and its implications for monetary policy are discussed in
	detail in \cite{ClaridaGaliGertler1999} from page 1691 onwards.
	
	\begin{itemize}
		%\item Purpose of the Model: New Keynesian model with inflation persistence to derive optimal monetary policy.
		\item Aggregate Demand: Hybrid New Keynesian IS curve.
		\item Aggregate Supply: Hybrid New Keynesian Phillips curve.
		%\item The Foreign Sector: no foreign sector
		%\item Microeconomic foundation: yes
		\item Shocks: A cost-push shock, a demand shock and the common monetary policy shock.
		%\item Variable dimension: The model is log-linearized around the steady state. Variables are expressed as percentage deviations from steady state.
		\item Calibration/Estimation: We use the same parametrization as in in NK\_RW97, however %IS THAT CORRECT? WHAT PARAMETERS ARE MEANT HERE?
		expected inflation enters the Phillips curve with a weight of 0.52 and lagged inflation with a weight of 0.48. In the IS curve the expected output gap has a weight of 0.56 and the lagged output gap has a weight of 0.44. % Shocks from DYNARE course file? %All other parameters are the same as in the baseline model.
		%\item Impulse responses: Figure \ref{img:NK_CGG99}.
		%\item Impulse responses: The first row of figure \ref{img:CGG99} shows impulse responses to a one unit monetary policy shock. Using the Taylor rule, one can see the effect of the additional persistence compared to RW97. The second row shows impulse responses to a one unit demand shock that is simply added to the IS curve. One way to interpret this shock is regarding it as transitory changes in monopoly power or tax distortions.
	\end{itemize}
	
	
	\subsection{NK\_CGG02: \texorpdfstring{\cite{ClaridaGaliGertler2002}}{Clarida et al. (2002)}, two-country model}
	
	\label{NKCGG02}
	
	\cite{ClaridaGaliGertler2002} derive a small-scale, two-country, sticky-price model to analyse optimal monetary policy. The two countries are symmetric in size, preferences and technology.
	
	\begin{itemize}
		\item Aggregate Demand: Households maximize their lifetime utility, where the utility function is separable in consumption and leisure, subject to an intertemporal budget constraint. They own the firms, are a monopolistically competitive supplier of labor to the intermediate firms and additionally hold their financial wealth in the form of one-period, state-contingent bonds, which can be traded both domestically and internationally.
		\item Aggregate Supply: Domestic production takes place in two stages. First there is a continuum of intermediate goods firms, each producing a differentiated material input under monopolistic competition using a production function that is linear in labor input and includes an exogenous technology parameter. They set nominal prices on a staggered basis \`{a} la Calvo and receive a subsidy in percent of their wage bill to achieve an undistorted steady state. Final goods producers then combine these inputs into output, which they sell to households under perfect competition. Wages are perfectly flexible. Thus, all workers will charge the same wage and work the same amount of hours. \cite{ClaridaGaliGertler2002} introduce an exogenous time-varying elasticity of labor demand to vary the wage-mark-up over time. The system of equations is collapsed into an IS equation and a Phillips curve, which determine the output gap and inflation, conditional on the path of the nominal interest rate both for the domestic and the foreign economy.
		\item Foreign sector:
		Producer currency pricing is assumed so that the Law of one price holds for the final consumption good and the CPI based real exchange rate is unity. Together with the assumption of complete markets this ensures that the consumption levels are equal in both countries at any point in time.
		\item Shocks: A cost push shock and the common monetary policy shock.
		%\item Variable dimension: The model is log-linearized around the steady state. Variables are expressed as percentage deviations from steady state.
		\item Calibration/Estimation: We take the parametrization of the small open economy model in \cite{GaliMonacelli2005} to calibrate the model. \cite{GaliMonacelli2005} calibrate the stochastic properties of the exogenous driving forces by fitting AR(1)
		processes to log labor productivity in Canada, which is their proxy for the domestic country, and log
		U.S. GDP, which they use as proxy for world output. The sample period comprises 1963:1--2002:4.
		%\item Impulse responses: Figure \ref{img:NK_CGG02}.
		%\item Impulse Responses: The first row of figure \ref{img:CGG03} shows impulse responses to a one unit monetary policy shock. Using the Taylor rule, one can see that there is no persistence in the model as in RW97. The second row shows impulse responses to a one unit demand shock that is simply added to the IS curve.
	\end{itemize}
	
	\subsection{NK\_CK08: \texorpdfstring{\cite{ChristoffelKuester2008}}{Christoffel and Kuester (2008)}}
	\label{NKCK08}
	\cite{ChristoffelKuester2008} incorporate search and matching frictions \`{a} la \cite{MortensenPissarides1994} into an otherwise standard New Keynesian business cycle model.
	
	\begin{itemize}
		
		\item Aggregate Demand: There is a large number of identical families in the economy. Each family consists of unemployed and employed members with time-additive expected utility preferences and an external habit. The representative family pools the labor income of its working members, unemployment benefits of the unemployed members and financial income. The family maximizes its welfare function by choosing consumption and nominal bond holdings subject to its budget constraint.
		
		\item Aggregate Supply: There are three sectors of production in the economy. Firms in the first sector produce a homogeneous intermediate good where labor is the only production input. The production process is subject to matching frictions. Nominal wages in the labor sector are Calvo staggered. The wholesale sector demands labor as the only production input in a perfectly competitive market to produce differentiated goods using a constant-return-to-scale production technology. Subject to price-setting impediments \`{a} la Calvo, the intermediate good is sold under monopolistic competition to a final retail sector. Retailers bundle differentiated goods into a homogeneous consumption/investment basket. These goods are then sold to consumers and government.
		
		\item Shocks: Three shocks: a serially correlated shock to the risk premium that drives a wedge between the return on bonds held by the families and the interest rate set by the central bank, an AR(1) labor sector-wide technology shock process, and a government spending shock.
		
		\item Calibration/Estimation: The model is calibrated to US data from 1964:Q1 to 2006:Q3. The underlying data set used covers data on output, hours worked, total wages, wages per employee, real hourly wages, vacancies, the civilian unemployment rate, the inflation rate and the interest rate.
		
		%\item Implementation: The model is implemented in the database with quarterly frequency.
		
	\end{itemize}
	
	
	\subsection{NK\_CKL09: \texorpdfstring{\cite{ChristoffelKuesterLinzert2009}}{Christoffel et al. (2009)}}
	\label{NKCKL09}
	%\label{EACKL09}
	
	\cite{ChristoffelKuesterLinzert2009} explore the role of labor markets for monetary policy in the Euro Area in a closed-economy, single-country New Keynesian model with \cite{MortensenPissarides1994} type of matching frictions. To allow for a direct channel from wages to inflation, the model builds on the right-to-manage framework of \cite{Trigari2006}. Moreover, \cite{ChristoffelKuesterLinzert2009} incorporate staggered wage-setting \`{a} la Calvo and account for job-related fixed costs as in \cite{ChristoffelKuester2008}. The aim of the paper is to investigate to which extent a more flexible labor market would alter the business cycle behavior and the transmission of monetary policy, employing a genuine Euro Area calibration (NK\_CKL09). Second, by estimating the model with Bayesian techniques (EA\_CKL09, see section \ref{EACKL09}) they analyze to which extent labor market shocks are important determinants of business cycle fluctuations. The results support current central bank practice to put considerable effort into monitoring Euro Area wage dynamics and treat some of the other market information as less important for monetary policy.
	
	\begin{itemize}
		
		\item Aggregate Demand: The demand as well as the supply structure follow closely the one described in \cite{ChristoffelKuester2008}. The economy consists of a large number of identical families that comprise unemployed and employed members with time-additive expected utility preferences that exhibit an external habit. The representative family pools the labor income of its working members, unemployment benefits of the unemployed members and financial income from assets that family members hold via a mutual fund. Each household also owns representative shares of all firms in the economy. It maximizes the sum of unweighted expected utilities of its individual members, by taking consumption, saving, vacancy posting, and labor supply decisions on their behalf.
		
		\item Aggregate Supply: The economy consists of three production sectors. The labor packers use exactly one worker as input to produce a homogeneous intermediate good labeled labor good. The process of labor bargaining is governed by wage rigidities. The wholesale sector buys the labor good from the labor packers in a perfectly competitive market and produces differentiated goods using a constant-return-to-scale production technology. These goods are sold under monopolistic competition to a final retail sector at a price that is subject to impediments \`{a} la Calvo and to a partial indexation rule. Retailers bundle the differentiated goods into a homogeneous consumption/investment basket and sell it to the consumers and to the government.
		
		\item Shocks: Three labor market shocks: a shock to the costs of posting a vacancy, a shock to the rate of separation, and a shock to the bargaining power of workers; a government spending shock; a wholesale sector cost-push shock.
		
		\item Calibration/Estimation: For the calibration exercise (NK\_CKL09) a quarterly Euro Area data set from 1984:Q1 to 2006:Q3 is used. The model is also estimated with Bayesian techniques (EA\_CKL09) employing output, year-on-year inflation, the nominal interest rate, wages per employee, unemployment and proxies for total hours worked and vacancies as observable variables.
		
		%\item Implementation:
		
	\end{itemize}
	
	\subsection{NK\_CW09: \texorpdfstring{\cite{CurdiaWoodford2009}}{Crudia and Woodford (2009)}}
	\label{NKCW09}
	\cite{CurdiaWoodford2009} extend the basic representative-household New-Keynesian model as in \cite{Woodford2003} to allow for a spread between the interest rate available to savers and borrowers. The spread can vary for endogenous or exogenous reasons (the implemented version in the MMB uses endogenous variation). The authors investigate how much of a difference the inclusion of financial frictions (relative to the frictionless baseline) makes for the model's predictions of the response of the economy to various types of shocks under a given monetary policy rule.
	\begin{itemize}
		\item Aggregate demand: Households maximize their lifetime utility, where the utility function is separable in consumption and leisure, subject to an intertemporal budget constraint. Households are either savers or borrowers, which differ in the utility that they can obtain from current expenditure. They own the firms and the financial intermediary. Households are monopolistically competitive suppliers of labor to the firms. Savers and borrowers hold their financial wealth in the form of one-period, riskless nominal contracts with the financial intermediary. The government also consumes a part of the composite good produced by the firms.
		\item Aggregate supply: The production side consists of the firms and the financial intermediary. A continuum of firms uses labor to produce differentiated goods. Price stickiness is introduced via the Calvo framework. The financial intermediary produces loans. He faces intermediation costs, which determine the interest rate spread between the borrowing rate and the savings rate. In addition to costly loan origination, part of the spread is due to the fact that some borrowers are fraudulent and do not plan to repay their loans. Both frictions are increasing in the amount of lending. As these intermediation costs vary, so does the spread between the lending and the borrowing interest rate.
		\item Shocks: The model features one shock on consumption expenditure of savers and another one on consumption expenditure of borrowers. In addition, the model also includes shocks on government purchases of the composite good, labor supply, the wage markup, distortionary tax, technology, government debt, monetary policy. Finally, the model has financial disturbances to the real resource cost of loan origination and monitoring as well as to the costs of fraudulent borrowing.   
		\item Calibration: Many of the model parameters follow those of standard  New Keynesian models such as those in \cite{Woodford2003}. The new parameters needed for the present model are those relating to heterogeneity or to the specification of the credit frictions.
	\end{itemize}
	
	
	
	\subsection{NK\_DEFK17: \texorpdfstring{\cite{delnegro2017eggertson}}{Del Negro et al. 2017} }
	\label{NKDEFK17}
	
	\cite{delnegro2017eggertson} introduce a liquidity friction into an otherwise standard DSGE model. This friction comes in two forms. First, a borrowing constraint for entrepreneurs so that they can only borrow up to a fraction of the value of their current investment. Second, a resaleability constraint that limits the amount of "illiquid" assets that can be sold. With this setup, the importance of a shock to the liquidity of private paper on the economy is examined. Can it generate a shock similar to the one in 2008 and can the government, through an increase in liquidity, effectively intervene in the economy? They find that the financial shock and the liquidity policy can have quantitatively large effects.
	
	\begin{itemize}
		
		\item Aggregate Demand: Households consist of a continuum of members each drawn to be either a worker supplying labor or an entrepreneur with an opportunity to invest. At the end of each period they share their consumption purchases and assets. Entrepreneurs want to sell as much equity and government bonds as possible to finance new capital, which yields a higher return. Hence, a negative liquidity shock affects entrepreneurs, who are not able to sell their equity anymore, thus reducing investment. 
		
		\item Aggregate Supply: Intermediate good producers combine labor and capital services to produce their goods while paying a fixed cost of production. Prices are set in a staggered way, following \cite{Calvo1983}. The goods market is characterized by monopolistic competition. Labor unions set the wage for each type of labor on a staggered basis. Competitive final good producers combine intermediate goods to sell a homogeneous final good. Finally, perfectly competitive  capital producers produce investment goods that are sold to the entrepreneurs.
		
		\item Shocks: The liquidity shock comes through a change in the parameter of the resaleability constraint.
		
		\item Estimation: The model is calibrated at quarterly frequency with U.S. data from 1953:Q1 to 2008:Q3.
	\end{itemize}
	
	
	
	\subsection{NK\_DT12: \texorpdfstring{\cite{defiore2012tristani}}{De Fiore and Tristani (2013)}}
	\label{NKDT12}
	\cite{defiore2012tristani} extend an otherwise standard New Keynesian model by introducing financial market imperfections: (wholesale) Firms need to pay wages prior to production, thus external financing is required. Asymmetric information and costly state verification between borrowers and lenders generate financial frictions in nominal terms. These frictions contribute to the propagation of the response of macroeconomic key variables to real and nominal shocks. The economy is populated by households owning retail sector firms, and entrepreneurs owning wholesale sector firms.
	
	\begin{itemize}
		\item Aggregate Demand: Households gain utility from consumption and leisure. Optimization leads to a standard forward-looking IS-Curve, augmented by a feed-back term on expected future spread increases. Additionally, a term including the current nominal rate is added, since this increases the financial mark-up and thus entrepreneurs' consumption.
		\item Aggregate Supply: The wholesale sector produces a homogeneous good under perfect competition, but subject to asymmetric information and monitoring costs. The retail sector uses the wholesale good to sell differentiated goods under monopolistic competition and Calvo Pricing.
		\item Shocks: The model exhibits a technology shock, the common monetary policy shock, a shock to the endowment of wholesale firms.
		\item Calibration: The model is calibrated at quarterly frequency, following the calibration of \cite{Woodford2003}.
	\end{itemize}
	
	
	\subsection{NK\_ET14: \texorpdfstring{\cite{EllisonTischbirek2014}}{Ellison and Tischbirek (2014)}}
	\label{NKET14}
	\cite{EllisonTischbirek2014} develop a small scale New-Keynesian model with a banking sector and include unconventional monetary policy in the form of asset purchases by the central bank. The aim of the paper is to investigate whether allowing for (always active) unconventional monetary policy as an addition to conventional interest rate policy can be welfare-increasing. The authors find that asset purchases have a stabilizing and welfare-enhancing effect on the economy. The optimal monetary policy mix prescribes that conventional interest rate policy react to inflation only, while unconventional asset purchases should be used to stabilize output.
	
	\begin{itemize}
		\item Aggregate Demand: Households maximize expected lifetime utility by choosing consumption of final goods and labor supply. They obtain funds from labor services, interest on deposits and dividend payments from firms. Each household is subject to dividend and lump-sum taxes.
		\item Aggregate Supply: Monopolistically competitive firms produce consumption goods employing household labor with a decreasing-returns-to-scale production function and subject to Calvo-style price rigidities.
		\item Financial Sector: Perfectly-competitive banks take deposits from households and purchase short- and long-term government bonds. In choosing the composition of the aggregate savings device offered to the household sector, banks perceive households as heterogeneous with regard to their desired investment horizon and assets of different maturities are considered imperfect substitutes. The price of single assets is thus influenced by supply and demand effects specific to that maturity. The central bank sets the short-term interest rate and can influence yields at different maturities by purchasing of government bonds. The treasury issues short-term bonds in a quantity consistent with the interest rate set by the central bank and long-term bonds following a rule linking the real quantity of long-term bonds to steady-state output.
		\item Shocks: The model features seven shocks: to the interest rate, asset purchases, consumption preference, labor supply preference, technology, intra-temporal elasticity of substitution, and government expenditure. All shocks are AR(1) processes.
		\item Calibration/Estimation: Calibration is based on \cite{gali2008monetary} and \cite{SmetsWouters2003,SmetsWouters2007}.
	\end{itemize}
	
	
	\subsection{NK\_FLMF18: \texorpdfstring{\cite{filardo2018monetary}}{Filardo et al. (2018)}}
	\label{NKFLMF18}
	\cite{filardo2018monetary} analyse the implications of monetary policy reacting to commodity prices in the presence of the risk of misdiagnosing the drivers of commodity price developments. They use a global economic model that builds on \cite{nakov2010monetary}, and in which the global economy is split into commodity importers and exporters and commodity prices are determined endogenously by global supply and demand. The economic performance of monetary authorities depends on their ability to identify whether commodity prices are driven by global supply or demand shocks.
	
	\begin{itemize}
		\item Aggregate Demand: The representative household in the commodity-importing countries maximizes lifetime utility over consumption and labor subject to a standard budget constraint. The representative household in the commodity-exporting countries owns the exporting firm wholly and its utility function depends only on the consumption of final goods subject to the constraint that consumption expenditures equal dividends from commodity production. Cross-border financial autarky is assumed.
		\item Aggregate Supply: The commodity supply stems from two types of commodity-exporting countries: a competitive and a monopolistic one. The latter one sets prices above marginal costs, the competitive ones take prices as given. The commodity is produced using final goods sold by commodity-importing countries. There are no nominal rigidities in the commodity-exporting countries. Final goods are produced in the commodity-importing country using labor and commodities as inputs. Final-good producers act under monopolistic competition and set prices according to \cite{Calvo1983}. Monetary policy is conducted only from the perspective of the commodity-importing country.
		\item Shocks: The model features aggregate demand and supply shocks..
		\item Calibration: The model is calibrated at a quarterly frequency and mostly in line with the literature. The commodity share in the consumption basket matches the share of primary commodity inputs in the US CPI (10\%). The share of commodities in the production function is set to 10\%, following \cite{arseneau2013commodity}. The competitive commodity production sector has a size of 10\% relative to GDP.
	\end{itemize}
	
	\subsection{NK\_FNL23: \texorpdfstring{\cite{ferraria2023toward}}{Ferrari and Nispi Landi (2023)}}
	\label{NKFNL23}
	\cite{ferraria2023toward} develop a medium-scale closed economy environmental DSGE model and calibrate it for the Euro Area. The model features a polluting sector and examines to what extend Green Quantitative Easing is a useful policy tool to reach a net zero target in emissions.
	
	\begin{itemize}
	\item Aggregate Demand: Households maximize their lifetime utility, where the choice variables are consumption, hours worked, as well as green, brown and public bonds. Households feature a greenium, i.e. they derive utility from investing in ”green” (non-polluting) firms, and disutility from investing in ”brown” (polluting) firms.
	
	\item Aggregate Supply: The representative final-goods firm uses a CES bundle to produce the final good with input from an intermediate firm. There are intermediate firms in monopolistic competition that produce bundles of green and brown production. They set their prices subject to the demand of the final good firm and pay quadratic adjustment costs. They use input from Green and Brown firms, which they acquire for a type-specific price. Green and Brown firms issue bonds to households and the central bank and buy capital from a capital producer. Polluting firms have to pay a tax for every unit of emissions they produce. For each unit of brown output, there are carbon units released in the atmosphere and complied in the atmospheric stock of carbon. A fraction of emissions is abated.
	
	\item Shocks: The model contains a total factor productivity shock and an expansionary monetary policy shock.
	
	\item Calibration/Estimation: The model is calibrated to the Euro Area, following the New Area- Wide Model (NAWM-II). 
	
	\item Replication: We replicated the impulse response functions to a positive TFP and an expansionary Monetary Policy shock. The variables include output, consumption, and inflation.
	\end{itemize}
	
	\subsection{NK\_GHP16: \texorpdfstring{\cite{gnocci2016housework}}{Gnocci and Pappa (2016)}}
	\label{NKGHP16}
	\cite{gnocci2016housework} introduce housework in an otherwise standard business cycle model. Introducing substitutability between home-produced and market goods, they generate com-plementarity between market consumption and hours worked and analyse how it affects the size of the fiscal multiplier.
	
	\begin{itemize}
		\item Aggregate  demand: Households maximize their lifetime utility, subject to an intertemporal budget constraint, where the utility function is increasing in both consumption and leisure and concave. They can consume home produced goods and market goods. Leisure is the residual time after subtracting hours worked at home and on the market from the time endowment.
		\item Aggregate  supply: There are infinitely many monopolistically competitive firms that buy market capital and hours worked to produce varieties of the market good. Prices are set following \cite{Calvo1983}.
		\item Shocks: This paper presents responses to a government spending shock.
		\item Calibration: The model is calibrated at quarterly frequency in order to match especially the sample averages of the ratio of investment to the capital stock, the capital-output ratios, the hours worked at home and on the market, and the share of government expenditure in GDP. The time series used refer to the time period 1950:Q1 to 2007:Q2, excluding the financial crisis.
	\end{itemize}
	
	
	
	\subsection{NK\_GK11: \texorpdfstring{\cite{GertlerKaradi2011}}{Gertler and Karadi (2011)}}
	\label{NKGK11}
	\cite{GertlerKaradi2011} build a quantitative monetary DSGE model with financial intermediaries that face endogenously determined balance sheet constraints. The authors use the model to analyse unconventional monetary policy measures.
	
	\begin{itemize}
		
		\item Aggregate Demand:  The representative household's utility is separable in consumption and leisure and allows for habit formation in consumption. Households postpone their consumption by holding deposits with the financial intermediaries. The amount of deposits is determined in such a way as to guarantee that the bankers' incentive constraint is satisfied. Expected-lifetime utility is maximized by choosing consumption and labor supplied to intermediate firms.
		
		\item Aggregate Supply: The financial intermediaries issue contingent claims to firms, financed by  deposits. An agency problem between the intermediaries and the depositors generates an endogenous leverage constraint with respect to the leverage ratio of the financial intermediaries. Competitive firms produce intermediate goods using labor services and capital, the latter of which is produced by a capital producer. Retail firms have monopolistic power and re-package intermediate output. Nominal frictions are introduced in the form of Calvo sticky prices. Non-reoptimizing firms index their prices to the previous period's inflation rate.
		
		\item Shocks: Capital quality shock, which affects the effective quantity of the capital stock.
		
		\item Calibration/Estimation: The financial sector parameters are chosen to satisfy a steady state interest rate spread of 100 basis points, a steady state leverage ratio of four, and an average lifetime horizon of bankers of a decade. The calibration of the conventional parameters mostly follows \cite{ChristianoEichenbaumEvans2005}.
		
		%\item Replication: We replicate the impulse responses of 12 key variables to a 5\% negative capital quality shock in Figure 2 of \cite{GertlerKaradi2011}.
		
	\end{itemize}
	
	
	
	\subsection{NK\_GK13: \texorpdfstring{\cite{GertlerKaradi2013}}{Gertler and Karadi (2013)}}
	\label{NKGK13}
	\cite{GertlerKaradi2013} build a quantitative DSGE model in order to analyze central bank large-scale asset purchases (LSAP). The model features private financial intermediaries that face endogenously determined balance sheet constraints stemming from a moral hazard problem in their deposit financing, giving rise to external finance premia. Unconventional monetary policy in the form of LSAP can reduce these premia and hence stimulate real economic activity.
	\begin{itemize}
		\item	Aggregate Demand: The representative household's utility is separable in consumption and leisure and features habit formation in consumption. Expected-lifetime utility is maximized by choosing consumption and labor supply. Households can hold deposits at financial intermediaries, government bonds, private assets issued by firms and are subject to lump-sum transfers.
		\item Aggregate Supply: Competitive intermediate goods are produced with a Cobb-Douglas technology using capital and labor. Household labor is purchased in competitive markets. Capital goods are bought from competitive capital goods producers (which are subject to adjustment costs) and financed issuing state-contingent securities to banks. Monopolistic retail firms, subject to Calvo-style price stickiness, repackage intermediate goods.
		\item Financial Sector: Banks transfer funds from households to non-financial firms and to the government, while engaging in maturity transformation. They hold long-term government bonds and securities from non-financial firms and fund themselves with short-term liabilities (beyond their net worth). A moral hazard/costly enforcement problem constrains the ability of banks to obtain funds from households, while they are able to perfectly monitor firms and enforce contracts. The central bank that can conduct monetary policy either by adjusting the short-term interest rate or, while facing a zero-lower-bound constraint, by engaging in asset purchases, either of long-term government bonds, private securities or both. Government expenditures are composed of government consumption and net interest payments from an exogenously-fixed stock of long-term debt. Revenues consist of lump-sum taxes and the earnings from central bank intermediation net transaction costs.
		\item Shocks: There are six shocks in the model: a total factor productivity shock, a government consumption shock, a capital quality shock and three monetary policy shocks (interest rate as well as asset purchase shocks to either private assets or government bonds). 
		\item Calibration/Estimation: The financial sector parameters are calibrated to satisfy a steady-state interest rate spread for government bonds of 50 basis points, and for private assets of 100 basis points, and an ad-hoc steady-state leverage ratio for banks. The calibration of the conventional parameters is standard.
	\end{itemize}
	
	
	
	\subsection{NK\_GLSV07: \texorpdfstring{\cite{gali2007understanding}}{Galí et al. (2007)}}
	\label{NKGLSV07}
	\cite{gali2007understanding} extend the standard New Keynesian model to allow for the presence of rule-of-thumb
	consumers for which consumption equals labor income. This enables them to generate an increase in consumption in response to a rise in government spending, in a way consistent with much of the recent evidence. Rule-of-thumb consumers partly insulate aggregate demand from the negative wealth effects generated by the higher levels of (current and future) taxes needed to finance the fiscal expansion, while making it more sensitive to current disposable income. The article considers two labor market structures. Here, the version of the model with imperfect labor markets is replicated and implemented in the MMB.
	
	\begin{itemize}
		\item Aggregate  demand: Households  gain  utility  from  consumption and leisure subject to appropriate budget constraints.  A fraction $(1 - \lambda)$ of households have access to capital markets where they can trade a full set of contingent securities, and buy and sell physical capital (Ricardian households). The remaining fraction $\lambda$ of households do not own any assets nor have any liabilities, and just consume their current labor income (rule-of-thumb households). Additionally, two alternative labor market structures are considered in the paper. The first one assumes a competitive labor market, with each household choosing the quantity of hours supplied given the market wage. Under the second labor market structure wages are set in a centralized manner by an economy-wide union. In that case hours are assumed to be determined by firms (instead of being chosen optimally by households), given the wage set by the union.
		
		\item Aggregate  supply:  Intermediate firms  act  under  monopolistic  competition and set nominal prices in a staggered fashion \`{a} la \cite{Calvo1983}. Their products are used as inputs by firms which produce final goods. Perfectly competitive final-good firms produce with a constant returns technology.
		
		\item Shocks:  This paper presents responses to a government spending shock.
		
		\item Calibration:  The model is calibrated at quarterly frequency.
	\end{itemize}
	
	
	
	\subsection{NK\_GM05: \texorpdfstring{\cite{GaliMonacelli2005}}{Galí and Monacelli (2005)}}
	\label{NKGM05}
	\cite{GaliMonacelli2005} develop a model of a small open economy which is part of a world economy comprised of a continuum of small open economies sharing identical preferences, technology and market structure but facing imperfectly correlated productivity shocks. With this framework, the authors analyze the macroeconomic implications of three different rule-based policy regimes for a small open economy, pointing out the trade-offs the authorities face between the stabilization of the nominal exchange rate, domestic inflation and the output gap.
	
	\begin{itemize}
		
		\item Aggregate Demand: The representative household in a small open economy seeks to optimize its utility separable between consumption and leisure subject to its budget constraint. Consumption is a composite of domestic and foreign goods, weighted by the degree of home bias in preferences, which represents the index of country openness. The dynamic IS equation is similar to that found for a closed economy but with the degree of openness influencing the sensitivity of the output gap to interest rate changes. Furthermore, the natural interest rate depends on the expected growth of world output.
		
		\item Aggregate Supply: Differentiated goods are produced from a typical firm using a linear technology with labor as input. Firms face price stickiness \`{a} la Calvo as in the case of a closed economy. Importantly, marginal costs are increasing in the terms of trade and in world output. The degree of country openness affects the slope of the New Keynesian Phillips curve of the small open economy, thus affecting the response of inflation to variations in the output gap.
		
		\item The Foreign Sector: Purchasing Power Parity and the law of one price hold. There is prefect exchange rate pass-through. Under the assumption of complete international financial markets, an international risk sharing in the form of the uncovered interest rate parity is obtained.
		
		\item Shocks: A domestic productivity shock and a world demand shock.
		
		\item Calibration/Estimation: The model is calibrated mostly to fit some characteristics of the Canadian economy. In order to calibrate the stochastic properties of the exogenous driving forces, AR(1) processes are fitted, using quarterly, HP-filtered data over the sample period 1963:Q1--2002:Q4.
		
		%\item Replication: Check the model in the Modelbase!
		
	\end{itemize}
	
	\subsection{NK\_GM07: \texorpdfstring{\cite{goodfriend2007banking}}{Goodfriend and McCallum (2007)}}
	\label{NKGM07}
	\cite{goodfriend2007banking} develop a small New Keynesian model with a banking sector and several interest rates to analyze the role of the banking sector for the evaluation of monetary policy. There are two versions of the model: one in which monetary policy is represented by a money supply rule, and one in which it is represented by a rule for the short-term nominal interest rate. Here, we focus on the latter. The model in the original article contains a mistake, which is detailed in the replication package. We implement the corrected version of the model in the MMB.
	\begin{itemize}
		\item Aggregate Demand: A representative household-firm maximizes expected utility derived from consumption and labor. The utility function is additive separable. Its budget constraint features capital accumulation, real balances and government bonds. Furthermore it receives the income derived from wages and (in its function as a monopolistically competitive firm) from selling its good on the market. Furthermore, consumption is constrained via a transaction constraint by the amount of deposits in the bank.   
		\item Aggregate Supply: Goods are produces with a standard Cobb-Douglas production function, featuring labor and capital. In the linearized system, the capital stock is held constant. The model features a standard New Keynesian Phillips Curve.
		\item Banking System and Interest Rates: The balance sheet of the bank features loans and money holdings on the asset side and deposits on the liability side. The bank produces loans with a Cobb-Douglas production function featuring labor for monitoring loans as well as collateral. Collateral is a function of the amount of capital and bonds in the economy. The model features interest rates on a hypothetical riskless bond without collateral value, the rate on bonds, on capital (both determined by the respective collateral values of bonds and capital), on loans, and the interbank rate, which serves as a policy instrument in the interest rate rule.
		\item Shocks: The model features an interest rate shock, a TFP shock, a shock to the collateral value of capital, and a shock to the monitoring efficiency of banks.
		\item Calibration/Estimation: The model is calibrated to match features of US data. The non-banking parameters are standard. The parameters in the banking system are set such as to match interest rate spreads, the reserve ratio and the velocity of money in the US.
	\end{itemize}
	
	
	\subsection{NK\_GM16 \texorpdfstring{\cite{gali2016understanding}}{Galí and Monacelli (2016)}}
	\label{NKGM16}
	\cite{gali2016understanding} study the gains from increased wage flexibility using a small open economy model with staggered price and wage setting and comparing the cases of an independent monetary policy versus a currency union. The model builds on the framework developed in \cite{GaliMonacelli2005}, extended by introducing sticky nominal wages (in addition to sticky prices) and additional shocks (labor tax shock, domestic demand, exports, and world interest rate). Two results stand out: (i) the effectiveness of labor cost reductions as a means to stimulate employment is much smaller in a currency union, and (ii) an increase in wage flexibility often reduces welfare, more likely so in an economy that is part of a currency union or with an exchange rate-focused monetary policy.
	
	\begin{itemize}
		\item Aggregate Demand: The representative household (that has a continuum of members) in a small open economy seeks to optimize its utility separable between consumption and leisure subject to its budget constraint. Consumption is a composite of domestic and foreign goods, weighted by the degree of home bias in preferences, which represents the index of country openness. Each household member is specialized in a differentiated occupation and supplies labor. Workers specialized in each occupation (or a union representing them) set the corresponding nominal wage, subject to an isoelastic demand function for their services. Each period only a fraction of labor types, drawn randomly from the corresponding population, have their nominal wage reset.
		
		\item Aggregate Supply: Differentiated goods are produced from a typical firm using a technology and constant elasticity of substitution (CES) function of the quantities of the different types of labor services employed. Firms face price stickiness \`{a} la \cite{Calvo1983}. Employment is subject to a proportional payroll tax, common to all labor types.
		
		\item The Foreign Sector: As in \cite{GaliMonacelli2005}, the size of the home economy is presumed to be negligible relative to that of the world economy, which allows taking world aggregates as exogenous. Furthermore, it is assumed that the law of one price holds and that financial markets (both domestic and international) are complete.
		
		\item Monetary Regime: Equilibrium behavior of the small open economy under two monetary policy regimes is considered. Under the first, which they refer to as inflation targeting, the central bank focuses on stabilizing domestic inflation. Under the second monetary regime, the home economy is assumed to be part of a world currency union, where domestic nominal interest rate will move one-for-one with the world interest rate, independent of domestic economic conditions.
		
		\item Shocks: Domestic productivity and demand shock, two types of global shocks (export shock and world interest rate shock) and the labor tax shock.
		
		\item Calibration: The model is calibrated mostly to fit some characteristics of the Euro Area and its peripheral countries.
	\end{itemize}
	
	
	
	\subsection{NK\_GS14: \texorpdfstring{\cite{gambacorta2014should}}{Gambacorta and Signoretti (2014)}}
	\label{NKGS14}
	\cite{gambacorta2014should} simplify the model by \cite{Geralietal2010}, which introduc-es a monopolistically competitive banking sector into a DSGE model with financial frictions \'{a} la \cite{Iacoviello2005}. This simplified version focuses on two frictions: a borrowing constraint, depending on the collateral's value, and a bank leverage constraint.
	
	\begin{itemize}
		\item Aggregate demand: Additionally to banks, there are two types of agents in the model: patient households and impatient entrepreneurs. Subject to a budget constraint, the households maximize their lifetime utility, choosing the levels of consumption, labor supply and bank deposits. Entrepreneurs are net borrowers and maximize their lifetime utility, choosing levels of consumption, labor demand and bank loans, subject to budget and borrowing constraints.
		
		\item Aggregate  supply: Entrepreneurs produce a wholesale good using household's labor and own physical capital. Retailers buy the intermediate goods, brand them and sell the differentiated goods at a price including a mark-up over the purchasing cost. Sticky prices \'{a} la \cite{Rotemberg1982} imply a New Keynesian Phillips curve.
		
		\item Banking sector: Each bank consists of two units: a wholesale and a retail branch. The wholesale unit collects deposits from households and issues loans, paying the interest rate set by the central bank and earning a wholesale loan rate. There exists a target leverage ratio and for deviating the bank has to pay a cost. The retail unit acts under monopolistic competition. It buys wholesale loans, differentiates them and resells them, applying a constant mark-up.
		
		\item Shocks: There is a technology shock and a cost-push shock.
		
		\item Calibration: The calibration is based on \cite{Geralietal2010}.
	\end{itemize}
	
	
	\subsection{NK\_GSSZ17: \texorpdfstring{\cite{gilchrist2017inflation}}{Gilchrist et al. (2017)}}
	\label{NKGSSZ17}
	\cite{gilchrist2017inflation} present a small-scale DSGE models with financial frictions to explain inflation dynamics during the financial crisis. In response to contractionary financial or demand shocks, financial frictions create incentives for firms to raise prices, therefore mitigating the deflationary effects of shocks. 
	
	\begin{itemize}
		\item Aggregate demand: Households maximize their lifetime utility, where the per-period utility function is separable in consumption and labour. Household utility from consumption is subject to good-specific external habits \'{a} la \cite{ravn2006deep}.
		
		\item Aggregate  supply: Intermediate goods production is done by a continuum of monopolistically competitive firms using a production function with decreasing returns to scale and fixed operating costs. Firms maximize the present value of discounted dividends and must commit to pricing and production decisions prior to realizations of their idiosyncratic shock. Depending on the shock realization, firms must raise external funds in order to pay workers. Firms can obtain external funds by issuing new equity subject to dilution costs reflecting agency problems in the financial markets. Firms also face \cite{Rotemberg1982} quadratic adjustment costs when changing nominal prices.
		
		\item Shocks: Technology shock, demand shock, financial shock and a monetary policy shock.
		
		\item Calibration: The model is calibrated for the US using standard values for the core block and following previous literature for the deep habits, the elasticities of substitution and financial market frictions.
		
		\item Replication: We simulated the impulse response functions to a demand shock in the economy with financial frictions and nominal rigidities, Figure 5 (red line) in the paper.
	\end{itemize}
	
	\subsection{NK\_IR04: \texorpdfstring{\cite{Ireland2004}}{Ireland (2004)}}
	\label{NKIR04}
	\cite{Ireland2004} develops a small New Keynesian model with real money balances entering both the forward-looking IS curve and the Phillips curve. The model is used to study the role of money in the U.S. business cycle.
	\begin{itemize}
		\item Aggregate Demand: A representative household maximizes expected utility, nonseparable between consumption and real money balances while separable in leisure, subject to a budget constraint. The optimizing behavior of this household leads to a forward-looking IS curve with real money balances entering the specification. This is due to the non-separability of real balances to consumption in the utility function, as real balances affect the marginal rate of intertemporal substitution.
		
		\item Aggregate Supply: A representative firm produces final goods according to a constant-returns-to-scale technology, using labor and intermediate goods as inputs. On the other hand, intermediate goods are produced under a linear technology using labor as input. The representative intermediate goods-producing firm has monopolistic power in the market, therefore acting as a price-setter. However, price setting is subject to Rotemberg quadratic adjustment costs. The optimizing behavior of this firm leads to a forward-looking Phillips curve with real money balances entering the specification.
		
		\item Shocks: An overall preference shock, a real money balances preference shock, a productivity shock and a monetary policy shock.
		
		\item Calibration/Estimation: Estimated via maximum likelihood using U.S. quarterly data over the period 1980:Q1--2001:Q3.
		
		%\item Replication: Check the model in the Modelbase!
		
	\end{itemize}
	
	
	\subsection{NK\_JO15ht, NK\_JO15lt: \texorpdfstring{\cite{jang2015okano}}{Jang and Okano (2015)}}
	\label{NKJO15}
	\cite{jang2015okano} examine the effects of foreign productivity shocks on monetary policy in a symmetric two-country New Keynesian model, following \cite{GaliMonacelli2005}. In response to asymmetric productivity shocks, firms in one country achieve a more efficient level of production and the terms of trade are directly affected by changes in both economies' output levels. International trade creates a transmission channel for inflation dynamics to which domestic monetary authority should react. Authors conclude that duration of output and inflation responses to changes in the level of foreign productivity is strongly affected by trade openness and that a monetary authority should be cautious about changes in foreign productivity level. Moreover, open economies should coordinate their policy responses to asymmetric shocks. 
	
	\begin{itemize}
		\item Aggregate Demand: The representative household in both economies seeks to optimize its utility that is separable in consumption and leisure, subject to its budget constraint. Consumption is a composite of domestic and foreign goods, weighted by the degree of home bias in preferences, which represents the index of country openness. The dynamic IS equation is similar to that found for a closed economy but with the degree of openness influencing its coefficients.
		
		\item Aggregate Supply: Differentiated goods are produced from a typical firm using a linear technology with labor as input. Firms face price stickiness \`{a} la Calvo as in the case of a closed economy. The degree of country openness affects the slope of the New Keynesian Phillips curves in both economies. Additionally, firms borrow from households at the gross nominal interest rate in order to pay wages. Nominal wage, therefore, corresponds to the discounted value of the nominal payoff in period t+1 generated by the portfolio held by households (\cite{RavennaWalsh2006}).
		
		
		\item Shocks: Foreign productivity shock, but domestic productivity shock, as well as foreign and domestic monetary policy shocks, could be considered.
		
		\item Calibration: The model calibration is based on open-economy DSGE literature, with the parameter values as in Smets and Wouters (2002), \cite{faia2008optimal}, and \cite{rabanal2010euro}. 
	\end{itemize}
	
	
	
	\subsection{NK\_KM16: \texorpdfstring{\cite{krause2016public}}{Krause and Moyen (2016)}}
	\label{NKKM16}
	The aim of the authors is to study the effects of an inflation target increase on real public debt. For this purpose, they employ a standard New Keynesian model augmented with long term debt with stochastic maturity.
	
	\begin{itemize}
		\item Aggregate Demand: As in \cite{RotembergWoodford1997}, households maximize lifetime utility from consumption leisure and money holdings subject to an intertemporal budget constraint while they own the firms. They have access to one-period bonds payed at the policy interest rate rule and to long term bonds with stochastic maturity.
		
		\item Aggregate Supply: Final good producing firms operate under perfect competition, combining the intermediate goods in final good. Intermediate firms are monopolistic competitors with a linear production function on labour, facing price rigidity \`{a} la Calvo.
		
		\item Financial Authority: The government follows a tax rule that reacts to deviations of real debt from its steady state level. Revenues come from a labour tax and newly issued debt while expenditures consist of exogenous government spending, interest payments on bonds and principal payments of the redeemed bonds.
		
		\item Monetary Authority: The Central Bank follows a Taylor interest rate rule with high persistence, that responds to output gap as well as to deviations of the inflation target from its steady state.
		
		\item Shocks: The model incorporates various shocks such as a monetary policy shock, an inflation target shock, a government spending shock and finally a debt shock. A debt shock is assumed to increase debt by 65\% from the current debt- to GDP ratio. All variables responses are expressed as percentage deviations from the steady state values apart from inflation and interest rates which are reported in annualized absolute deviations.
		
		\item Calibration: The model is parametrized at quarterly frequency. Basic parameters values follow \cite{SmetsWouters2007}. The stochastic maturity of bonds is set to 0.0472 so as to match the average maturity of US debt accounting to 5.3 years..
		
		%\item Replication: Check the model in the Modelbase!
		
	\end{itemize}
	
	
	
	\subsection{NK\_KRS12: \texorpdfstring{\cite{KannanRabanalScott2012}}{Kannan et al. (2012)}}
	\label{NKKRS12}
	The NK\_KRS12 model is a DSGE model with housing in the spirit of \cite{Iacoviello2005} and \cite{IacovielloNeri2010} to study the role of monetary policy in mitigating the effects of house price booms. They find that a monetary policy rule with credit aggregates can help counter accelerator mechanisms that push up credit growth and house prices. They also study the effect of macroprudential polices on welfare.
	
	\begin{itemize}
		\item Aggregate Demand: There are two types of households, the borrowers (``impatient'') and the savers (``patient''). They both derive utility from consumption, holdings of housing and leisure. The impatient households discount the future more heavily. This specification induces the impatient households to face borrowing constraints, which is consistent with standard lending criteria used in the mortgage market where the borrowing is limited to a fraction of the housing value. For both types of households, the holdings of housing are subject to housing adjustment costs.
		
		\item Aggregate Supply: There are two type of producers: Final good producers operating under perfect competition and intermediate good producers that supply their goods imperfectly. Price is set \`{a} la  Calvo-type of restrictions.
		
		\item Financial Sector: Borrowers and savers can meet only through financial intermediaries, which charge a spread that depends on the net worth
		of borrowers. The determination of the spread follows the financial accelerator idea of \cite{BernankeGertlerGilchrist1999}.
		
		\item Shocks: A housing demand shock, a financial shock and a technology shock.
		
		\item Calibration/Estimation: A mixture of calibrated and estimated parameters for the US economy. Parameters governing real and nominal rigidities are taken from \cite{IacovielloNeri2010}.
	\end{itemize}
	%\item Replication: Check the model in the Modelbase!
	
	\subsection{NK\_KW16: \texorpdfstring{\cite{kirchner2016fiscal}}{Kirchner and van Wijnbergen (2016)}}
	\label{NKKW16}
	Kirchner and van Wijnbergen extend the model by \cite{GertlerKaradi2011} such that banks are allowed to hold government bonds in addition to capital assets. In this model, the authors analyze the effects of a government spending shock. The main point of the paper is that when banks are balance sheet constrainted, debt-financed fiscal expansions trigger a crowding out of loans to private firms on the banks' balance sheet and reduce the government spending multiplier. Additionally the effects of equity injections into the banking system by the government are evaluated. 
	\begin{itemize}
		\item Aggregate Demand: as in NK\_GK11
		\item Aggregate Supply: as in NK\_GK11
		\item Financial Sector: similar to NK\_GK11, but in addition to loans to private firms, banks hold government bonds on their balance sheet. Thus fiscal policy becomes relevant and enters the model. 
		\item Shocks: The model features a government spending shock, a capital quality shock, a monetary policy shock, and a TFP shock.
		\item Calibration/Estimation: The calibration of most parameters in the paper follows NK\_GK11. The divertibility parameter for government bonds is the same as for capital assets. The debt-to-GDP ratio is set to 60\% of annualized GDP. 
		
	\end{itemize}
	
	
	\subsection{NK\_LWW03: \texorpdfstring{\cite{LevinWielandWilliams2003}}{Levin et al. (2003)}}
	\label{NKLWW03}
	This model is used for comparison in the robustness analysis of monetary policy rules by \cite{LevinWielandWilliams2003}. Its structure is similar to the NK\_RW97 model presented above, but without explicit treatment of government spending.
	\begin{itemize}
		\item Aggregate Demand: Standard New Keynesian IS curve.
		\item Aggregate Supply: Standard New Keynesian Phillips curve.
		\item Shocks: A cost-push shock, a shock to the real interest rate and the common monetary policy shock.
		%\item Variable dimension: The model is log-linearized around the steady state. Variables are expressed as percentage deviations from steady state.
		\item Calibration/Estimation: In calibrating the model, the parameter values of \cite{Woodford2003} adjusted for annualized variables as in     \cite{LevinWielandWilliams2003} are used.
		%\item Impulse responses: Figure \ref{img:NK_LWW03}.
	\end{itemize}
	
	
	
	\subsection{NK\_MCN99cr: \texorpdfstring{\cite{McCallumNelson1999}}{McCallum and Nelson (1999)}}
	\label{NKMCN99cr}
	The model in \cite{McCallumNelson1999} is used to monitor the performance of operational monetary policy rules. Two distinct variants of the model are used, mainly differing in the choice of the aggregate supply setup. In the first setup, aggregate supply is based on a standard Calvo-Rotemberg (NK\_MCN99cr) specification of the Phillips curve where inflation is linked to expected inflation and the output gap. In the second setup of the model, the authors introduce the so-called P-bar price adjustment (NK\_MCN99pb) where price changes occur in order to gradually eliminate deviations of actual from market clearing values of output.
	\begin{itemize}
		\item Aggregate Demand: Standard New Keynesian IS and LM curve.
		\item Aggregate Supply: Two setups: (i) Standard New Keynesian Phillips curve (NK\_MCN99cr), (ii) P-bar price adjustment (NK\_MCN99pb).
		\item Shocks: A shock to the IS curve which follows an AR(1) process, a shock to the LM curve, an investment shock, a shock on capacity output and the common monetary policy shock.
		%\item Variable dimension: The model is log-linearized around  the steady state. Variables are expressed as percent/100.
		\item Calibration/Estimation: The model equations are estimated individually by ordinary least squares and instrumental variable estimation for U.S. data. The sample period comprises 1955--1996.
		%\item Replication: HAS TO BE FILLED IN
		%\item Impulse responses: Figure \ref{img:NK_MCN99}
	\end{itemize}
	
	
	% IRELAND (2004) model: Part of Modelbase version 1? (not yet listed in table 6 of the main part)
	
	\subsection{NK\_MI14: \texorpdfstring{\cite{michaillat2014atheory}}{Machaillat (2014)}}
	\label{NKMI14}
	\cite{michaillat2014atheory} embeds a search-and-matching model into a New Keynesian model to analyse the effects of an increase in public employment at different stages of the business cycle. In this model, the public-employment multiplier is positive and countercyclical.
	
	\begin{itemize}
		\item Aggregate Demand: A representative large household maximizes expected lifetime utility by choosing an optimal consumption stream subject to a budget constraint. Workers in a household pool their income before choosing their consumption of the final good and how much to save via one-period bonds. The government does not consume in the form of purchasing goods from the private sector but compensates public employees.
		
		\item Aggregate Supply: Final-Good firms produce the final good using intermediate goods and sell it on a perfectly competitive market. The intermediate good is produced by a monopolist using labor as the sole input. The monopolist faces a price-adjustment cost following \cite{Rotemberg1982} and needs to pay a hiring cost.
		
		\item Labour Market: The labor market has a search-and-matching structure in which the number of matches is given by a Cobb-Douglas function of vacancies and unemployment. The probability of finding a job and the rate of filling vacancies both for the private and the public sector depend on the labor market tightness. The fraction of destroyed worker-job matches is constant and exogenous and the real wage is a function of technology.
		
		\item Shocks: In the model there is a technology shock that directly affects real wages.
		
		\item Calibration: The model is calibrated to a weekly frequency to US data. The calibration in the replication file remains weekly, but the model that is implemented in the MMB is calibrated to a quarterly frequency.
	\end{itemize}
	
	
	
	
	
	\subsection{NK\_MM10: \texorpdfstring{\cite{MehMoran2010}}{Meh and Moran (2010)}}
	\label{NKMM10}
	
	The NK\_MM10 model is a medium-scale DSGE model with a banking sector where bank capital plays a crucial role in mitigating the moral hazard problem between bankers and their creditors. The model is developed to see whether or how significant the capital position of bank influences the business cycle through a bank capital channel. There are three groups of agents in the model: households, entrepreneurs, and banks.
	\begin{itemize}
		\item Aggregate Demand: Households consume, allocate savings between currency and bank deposits, provide the differentiated labor services, choose a capital utilization rate, and buy capital goods. Entrepreneurs and bankers are risk neutral and they consume their accumulated wealth when exiting the economy. New agents with zero assets replace exiting ones. Monetary authority sets the short-term interest rate according to the Taylor rule.
		
		\item Aggregate Supply: Monopolistically competitive firms manufacture intermediate goods subject to nominal rigidities. Competitive firms produce final goods aggregating intermediate goods. Entrepreneurs produce capital goods with a technology that uses final goods as inputs and faces idiosyncratic uncertainty.
		
		\item Financial Contract: The optimal financial contract among three parties (an entrepreneur, a bank, an investor (household)) borrowed from \cite{holmstromTirole1997} represents the financial sector in the model economy. There are two moral hazard problems presented in this framework due to the imperfect and costly monitoring technology of a bank. The certain levels of banks' net worth as well as entrepreneurial net worth are needed in order to incentivize banks and investors.
		
		\item Shocks: A technology shock, a monetary policy shock, a shock to bank capital
		
		\item Calibration/Estimation: Many of model parameters are calibrated following the previous DSGE literature such as \cite{ChristianoEichenbaumEvans2005}. Parameters related to financial contract are calibrated such that the model's steady state meets several counterpart empirical moments.
	\end{itemize}
	
	\subsection{NK\_MPT10: \texorpdfstring{\cite{monacelli2010unemployment}}{Monacelli et al. (2010)}}
	\label{NKMPT10}
	\cite{monacelli2010unemployment} employ a model with search and matching frictions in the labor markets to analyze the effects of government spending on the unemployment rate in the US. While the main analysis in the model is conducted in an RBC model, the replicated and implemented model in the MMB is the version of the model with sticky prices that is discussed in section 7 of \cite{monacelli2010unemployment}. While in most versions of the model, which the authors discuss, the unemployment multiplier is small, they show that large effects of government spending on unemployment can be obtained, when complementarities between consumption and leisure in the utility function is coupled with price stickiness. 
	\begin{itemize}
		\item Aggregate Demand: The representative household is modelled as a large family with a continuum of members. They pool income and consumption and maximize a common utility function. There are complementarities between consumption and leisure in the utility function. Households consume, work, and invest in bonds and capital assets, where investment in capital is subject to adjustment costs. They search for vacant jobs and engage in wage bargaining with hiring firms.
		
		\item Aggregate Supply: Firms produce output goods with a Cobb-Douglas production function featuring capital and labor. The model version implemented here, additionally features monopolistically competitive retailers, which are subject to nominal rigidities \`{a} la Calvo. Firms engage in wage bargaining as well. 
		
		\item Labor market: Matches in the labor markets are produced with a Cobb-Douglas function of unemployed workers and vacancies. The probabilities of finding a job and of filling a vacancy are endogenous. The separation rate is exogenous. Wages are the result of Nash bargaining between households and firms. The respective reservation wages for households and firms are functions of the disutility of labor, the marginal product of labor and the respective search costs for households and firms.
		
		\item Shocks: The model features a government spending shocks and an interest rate shock.
		
		\item Calibration/Estimation: In the replication file, the model is calibrated to monthly frequency. In file that is implemented in the MMB, the calibration is quarterly. The model is calibrated to US data. The parameters specific to the labor market are chosen such that is matches the average job finding probability and the average tightness in the data, and to satisfy the Hosios condition. 
	\end{itemize}
	
	
	
	\subsection{NK\_NS14 : \texorpdfstring{\cite{NakamuraSteinsson2014}}{Nakamura and Steinsson (2014)}} 
	\label{NKNS14}   
	\cite{NakamuraSteinsson2014}  analyze the effects of government spending in a monetary and fiscal union like the United States. They estimate so-called Open Economy Multiplier (OEM), the effect an increase in government spending of one region has on relative output and employment in another region of the union.  Their model explores three different types of New Keynesian DSGE open economy models consisting of two regions within one country to estimate the OEM. The authors find that the model with firm-specific capital replicates the empirical estimates the best. This model is implemented in the MMB. 
	\begin{itemize}
		\item Aggregate Demand: Households maximize \cite{greenwood1988investment} (GHH) type utility in both region subject to their budget constraints. The GHH utility function emphasizes the complementarity of consumption and labor supply. Both regions have integrated goods market and each household consumes both home and foreign goods. 
		The solutions to the HHs' problem in each region, the Euler equations, represent the aggregate demand. They show that the change in	consumption is a function of the expected inflation, interest rate and the change in labor supply. 
		Furthermore, consumption in the two regions are linked by relative labor supply and the real exchange rate capturing relative purchasing power.
		\item Aggregate Supply: Firms in each region produce their own goods using labor and capital. Each firm maximizes its profit subject to the demand constraint from home and foreign consumption and government spending in their own region. Regarding the price setting, firms keep their price unchanged with the probability of $\alpha$ according to \cite{Calvo1983}. For the capital stock, each firm has its own capital and decides how much to invest in each period. Firms face convex capital adjustment costs, leading to smooth capital formation. Consequently, the aggregate supply, represented by the Phillips curve, is standard.. The current inflation rate depends on discounted future inflation and firm marginal cost.
		\item Policies: Since this model assumes two regions within a monetary union, there exists one central bank whose policy affects both regions simultaneously. However, fiscal spending is carried out on the regional level, tax is levied on the federal level. Taxes are assumed to be non-distortionary.
		\item Shocks: There is a government spending shock in each region and a monetary policy shock.
		\item Calibration/Estimation: The parameters are calibrated to match the U.S. economy.
	\end{itemize}
	
	\subsection{NK\_PP17: \texorpdfstring{\cite{paoli2017coordinating}}{Paoli and Paustian (2017)}}
	\label{NKPP17}
	\cite{paoli2017coordinating} study optimal monetary and macroprudential policies in a small-scale calibrated New Keynesian DSGE model with a moral hazard problem between banks and depositors in the spirit of \cite{GertlerKaradi2011}. The possibility of banks diverting funds from depositors implies that banks are constrained in the amount they can lend to firms. This financial friction motivates the use of macroprudential instruments.
	\begin{itemize}
		\item Aggregate demand: Households maximize their lifetime utility, where the per-period utility function is separable in consumption and two types of labour. They can hold deposits at financial intermediaries. 
		\item Aggregate Supply: Intermediate firms combine both types of labour into the intermediate good using a Cobb-Douglas production function. The entrepreneurs operating the intermediate firms must pay the wage bill associated with one of the inputs before production. Monopolistically competitive final goods firms purchase intermediate goods from entrepreneurs and create final goods using a linear production function. Final goods pricing is subject to Rotemberg quadratic adjustment costs. The final goods are aggregated to an output bundle according to a CES function. 
		\item Financial intermediaries: Banks lend to intermediate goods producers and collect deposit. They also receive a direct subsidy from the macroprudential authority. Bankers maximize terminal net wealth and have the possibility to divert a certain fraction of assets. This  yields an endogenous leverage constraint such that the incentive compatibility constraint is satisfied. Together with the borrowing-in-advance constraint, this introduces a credit friction. 
		\item Shocks: A productivity shock, a mark-up shock, a net worth shock, a moral hazard shock and a monetary policy shock.
		\item Calibration/Estimation: The parameters are calibrated to match the U.S. economy at quarterly frequency.
		
	\end{itemize}
	
	\subsection{NK\_PSV16: \texorpdfstring{\cite{pancrazi2016price}}{Pancrazi et al. (2016)}}
	\label{NKPSV16}
	\cite{pancrazi2016price} consider the so-called financial accelerator mechanism used in many articles since \cite{BernankeGertlerGilchrist1999} and show that the procedure of approximating the price of old capital by the net-of-depreciation price of new capital has profound implications when the capital depreciation rate is positive. When accounting for the appropriate price of capital, the effects of the financial accelerator are even stronger than originally assessed. Since the setup is the same as in \cite{BernankeGertlerGilchrist1999} where entrepreneurs borrow in credit markets to finance their investment in capital, the strength of the financial accelerator turns out to depend crucially on the dynamics of the price of capital. This conclusion has important first-order effects on the solution of a model that assumes a positive depreciation rate of capital together with investment adjustment costs.
	\begin{itemize}
		\item Aggregate  demand:  Households  gain  utility  from  consumption,  leisure  and  real  money  balances. They work, consume, pay taxes, hold money, and invest their savings, in form of deposits, in a financial intermediary that pays the riskless rate of return. These deposits are transferred to entrepreneurs in the form of loanable funds.  Entrepreneurs  use  capital  and  labor  to  produce  wholesale goods  that  are  sold  to  the  retail  sector.  Each  period,  entrepreneurs  have  to  accumulate  capital  that  becomes available  for  production  in  the  subsequent  period.  Entrepreneurs  have  to  borrow  from  households  via  a  financial intermediary  to  finance  capital  purchases.  Since  the  financial  intermediary  has  to  pay  some  auditing  costs  to observe  the  idiosyncratic  return  to  capital,  an  agency  problem  arises.  The  optimal  contract  leads  to  an  aggregate relationship  of  the  spread  between  the  external  finance  costs  and  the  risk-free  rate  and  entrepreneurs'  financial conditions  represented  by  the  leverage  ratio.
		\item Aggregate  supply:  Retail  firms  act  under  monopolistic  competition.  They  buy  wholesale  goods  produced  by entrepreneurs  in  a  competitive  market  and  differentiate  them  at  zero  cost.  Price  stickiness  is  introduced  via the  \cite{Calvo1983} framework.  \cite{BernankeGertlerGilchrist1999}  assume  that  reoptimizing  firms  have  to  set  prices  prior  to  the realization  of  shocks  in  that  period,  so  that  previous  period's  expectations  of  the  output  gap  and  future  inflation enter  the  New  Keynesian  Phillips  curve.
		\item Shocks:  This  paper  presents responses to  a  technology  shock,  as well as to a monetary  policy  shock. 
		\item Calibration/Estimation:  The  model  is  calibrated  at  quarterly  frequency.
	\end{itemize}
	
	\subsection{NK\_RA16: \texorpdfstring{\cite{rannenberg2016bank}}{Rannenberg (2016)}}
	\label{NKRA16}
	\cite{rannenberg2016bank} develops a model, which combines the financial frictions developed by \cite{BernankeGertlerGilchrist1999} and by \cite{GertlerKaradi2011}, and analyses the effects of contractionary shocks, to capture features of the Great Recession. The role of both financial frictions are illustrated by comparing model variants, in which one, none or both frictions are turned off. The model matches the relative volatility of the external finance premium and the procyclicality of bank leverage observed in US data. Here, the full model is replicated and implemented.
	\begin{itemize}
		\item Aggregate Demand: Representative households consume, work, and invest in riskless one-period bonds. Utility is separable in consumption and leisure. The utility function features habit formation.
		
		\item Aggregate Supply: Perfectly competitive capital good producers invest in new capital, subject to convex investment adjustment costs.  Retailers produce output with a Cobb-Douglas production function featuring capital and leisure. They finance a fraction of their factor costs by working capital loans from banks. They act under monopolistic competition and set their prices subject to Calvo frictions. The model used in the simulation additionally features variable capital utilization. 
		
		\item Financial Sector: Banks extend riskless loans to retailers and risky loans to entrepreneurs. When bankers exit the sector, they consume a fraction of their net worth. The initial net worth that new bankers receive is a constant. In all other features they are identical to banks in \cite{GertlerKaradi2011}. While banks are risk averse, entrepreneurs are risk-neutral. They accumulate capital, take loans from banks and can default. In all other features they are modelled as in \cite{BernankeGertlerGilchrist1999}. The optimal contract is between the bank and the entrepreneur. 
		
		\item Shocks: The implemented model features a TFP shock, a government spending shock, and an interest rate shock. 
		
		\item Calibration/Estimation: The model is calibrated to US data over the period from 1990Q1 to 2013Q4. Rannenberg highlights some of the targets for the calibration in the data. Among them the risk free rate, the spread of the loan rate over the risk-free rate, the leverage ratio of the non-financial sector, the quarterly bankruptcy rate of entrepreneurs, and the bank capital ratio.
		
		
	\end{itemize}
	
	\subsection{NK\_RW06: \texorpdfstring{\cite{RavennaWalsh2006}}{Ravenna and Walsh (2006)}}
	\label{NKRW06}
	\cite{RavennaWalsh2006} build a New Keynesian model with a cost channel of monetary transmission and study optimal monetary policy.
	\begin{itemize}
		
		\item Aggregate Demand: The model economy consists of households, firms, the government, and financial intermediaries interacting in asset, goods, and labor markets. Households maximize their expected present discounted value of utility defined over a composite consumption good, a taste shock and leisure. The composite good consists of differentiated products produced by final goods producers. Households enter each period with cash holdings, receive their wage income and use it to make deposits at the financial intermediary. The remaining cash balances are available for the purchase of consumption goods. At the end of a period, households receive profit income from the financial intermediary and firms, and the principal and interest on their deposits at the intermediary.
		
		\item Aggregate Supply: The goods market is characterized by monopolistic competition, and the adjustment of prices follows the Calvo setting. Firms must borrow money from the financial intermediary at the gross nominal interest rate to pay for part of their wage bill.
		
		\item Shocks: A composite demand shock.
		
		\item Calibration/Estimation: The model is calibrated to the US economy.
		
		%\item Implementation: We implement the model under optimal commitment in the database.
		
		
	\end{itemize}
	
	
	\subsection{NK\_RW97: \texorpdfstring{\cite{RotembergWoodford1997}}{Rotemberg and Woodford (1997)}}
	\label{NKRW9}
	The model and the estimation strategy is discussed in detail in
	\cite{RotembergWoodford1997}.  The equations of this model can be derived from the
	behavior of optimizing agents. The expectational IS equation and the
	policy rule together can be viewed as determining aggregate demand,
	while the New-Keynesian Phillips curve equation determines aggregate supply.
	The Phillips curve equation can be obtained as a log-linear approximation
	to the first-order condition of optimizing firms with either
	Calvo-style staggered price contracts \citep{Yun1996} or convex
	costs of price adjustment \citep{Rotemberg1982}. The IS equation
	can be obtained as a log-linear approximation of the
	representative household's first-order equation in a model in which
	consumption, leisure, and real money balances are each additively
	separable in the utility function, and total consumption demand
	(private and government consumption) is equal to aggregate output.
	
	\begin{itemize}
		%\item Purpose of the Model: Develop a method to compute optimal monetary
		%policy based on a utility function based welfare criterion in a model with
		%optimizing agents that fits US data nearly as well as unrestricted VARs.
		\item Aggregate Demand: Standard New Keynesian IS curve.
		\item Aggregate Supply: Standard New Keynesian Phillips curve.
		%\item The Foreign Sector: no foreign sector
		%\item Microeconomic foundation: yes
		\item Shocks: A cost-push shock following an AR(1) process, the common monetary policy shock, a government spending shock representing the common fiscal policy shock.
		%\item Variable dimension: The model is log-linearized around the steady state. Variables are expressed as percentage deviations from steady state.
		\item Calibration/Estimation: \cite{RotembergWoodford1997} match the empirical impulse
		response functions to a monetary policy shock in a VAR (detrended
		real GDP, inflation, funds rate) and the empirical variances with
		the variances and the theoretical impulse responses from the model
		to all three shocks. Quarterly U.S. data for the period 1980:Q1--1995:Q2 is
		used. The estimated parameters are taken from \cite{Woodford2003}
		table 6.1. However, we do not have information on the calibration of the shock processes. Hence, we employ the estimation results from \cite{adam2006optimal} for the NK\_RW97 shock specifications.
		% and in \cite{AdamBilli2006}.
		%\item Impulse responses: Figure \ref{img:NK_RW97}.
		%\item Impulse responses: The first row of figure \ref{img:RW97} shows impulse
		%responses to a one unit monetary policy shock. Using the Taylor rule, there is no
		%persistence in the model, and hence all variables return back to steady state after
		%one period. The second row shows impulse responses to an increase in government
		%spending by one percent of GDP.
	\end{itemize}
	
	\subsection{NK\_ST13: \texorpdfstring{\cite{stracca2013inside}}{Stracca (2013)}}
	\label{NKST13}
	Stracca develops a New Keynesian model with money endogenous and exogenous money. While exogenous money is base money supplied by the central bank, endogenous money is equivalent to bank deposits that affect macroeconomic dynamics due to a deposit in advance constraint for households. In the model, the presence of inside money attenuates the effects of technology and monetary policy shocks. 
	\begin{itemize}
		\item Aggregate Demand: A representative households chooses consumption, labor supply, bond holdings and deposit holdings. The utility function is additive separable in consumption, labor and deposit holdings, where quadratic adjustment costs for deposit holding are introduced into the utility function. Next to its budget constraint, it faces a deposit-in-advance constraint that generates a motive for deposit supply.
		
		\item Aggregate Supply: Intermediate good producers produce output with a Cobb-Douglas function featuring labor and capital. They finance the wage bill and investments with loans from banks. They are monopolistically competitive and set prices. Price setting and capital accumulation are subject to quadratic adjustment costs. Final good producers repackage the intermediate goods and sell them as final goods.
		
		\item Financial Sector: The bank finances itself with deposits, bonds and central bank credit. It extends loans and holds base money. The two components of the cost of financial intermediation are proportional to the amount of loans and the amount of deposits, respectively.
		
		\item Shocks: The model features a policy rate shock, a TFP shock, a shock to the demand for inside money and a shock to the supply of inside money.
		
		\item Calibration/Estimation: The model is calibrated to US data. 
	\end{itemize}
	
	
	\subsection{RBC\_DTT11: \texorpdfstring{\cite{DeFioreetal2011}}{De Fiore et al. (2011)}}
	\label{RBCDTT11} 
	\cite{DeFioreetal2011} introduce financial frictions into an otherwise standard RBC model without price stickiness. Specifically, they assume that total funds, which are required for production, are nominal and predetermined. Monetary policy can therefore affect the real value of funds. Furthermore, the amount of internal funds is limited, such that entrepreneurs always have to borrow external funds via financial contracts with banks. A state verification problem arises due to asymmetric information since idiosyncratic productivity shocks are firms' private information. Banks have to pay monitoring costs to reach perfect information. 
	\begin{itemize}
		\item Aggregate Demand: Households gain utility from consumption, leisure and real
		money balances. Households can hold deposits at financial intermediaries, government bonds and money.
		They are subject to lump-sum transfers. Government consumption is assumed to
		be a share of production net of the monitoring costs.
		\item Aggregate Supply: Firms act under perfect competition. Entrepreneurs use labor
		with a linear technology and aggregate productivity is stochastic. Additionally, each
		firm faces an idiosyncratic shock whose realization is private information.
		\item Financial Sector: Each period, entrepreneurs have to accumulate funds that become
		available for financing production in the subsequent period. They have to borrow
		from households via financial intermediary as internal funds are limited. Since
		the financial intermediary has to pay monitoring costs to observe the idiosyncratic
		shock, a state verification problem arises.
		\item Shocks: The model exhibits a technology shock, three different financial shocks
		(exogenous reduction in the level of internal funds, shock to the standard deviation
		of idiosyncratic technology shocks, increase in the monitoring cost parameter) and
		the common monetary policy shock.
		\item Calibration/Estimation: The model is calibrated at quarterly frequency. The
		volatility of idiosyncratic productivity shocks and the steady-state death probability of entrepreneurs are chosen, so as to generate an annual steady-state credit
		spread of approximately 2 percent and a quarterly bankruptcy rate of approximately
		1 percent following \cite{CarlstromFuerst1997}. The monitoring cost parameter
		is set at 0.15 according to \cite{Levinetal2004}. The calibration of the conventional
		parameters mostly follows \cite{ChristianoEichenbaumEvans2005}.
	\end{itemize}
	
	
	
	\section{Estimated U.S. Models}
	\subsection{US\_ACEL: CEE/ACEL by \texorpdfstring{\cite{AltigChristianoEichenbaumLinde2005}}{Altig et al. (2005)}}
	\label{USACELm}
	The purpose of the authors is to build a model with optimizing agents that can account for the observed inertia in inflation and persistence in output \citep{ChristianoEichenbaumEvans2005}. In the version by \cite{AltigChristianoEichenbaumLinde2005} firm-specific capital is introduced to get a Calvo parameter consistent with the microeconomic evidence of price re-optimizations on average once every 1.5 quarters. The Modelbase contains four different specifications of the CEE/ACEL model, labeled by m = monetary policy shock, t = technology shock and sw = SW assumptions, i.e. no cost channel and no timing constraints as in \cite{taylor2009surprising}.
	\begin{itemize}
		\item Aggregate Demand: The representative household's utility is separable in consumption and leisure and allows for habit formation in consumption. Expected-lifetime utility is maximized, choosing optimal consumption and investment, as well as the amount of capital services supplied to the intermediate firms (homogenous capital model) and portfolio decisions. Investment adjustment costs are introduced. Furthermore, the household determines the wage rate for its monopolistically supplied differentiated labor services whenever it receives a Calvo signal. In those periods, in which it does not receive a signal, the wage is increased by the lagged inflation rate augmented by the steady state growth rate of a combination of the neutral technology shock and the shock to capital embodied technology. Labor services are sold to a competitive firm that aggregates the differentiated services and supplies the resulting aggregated labor to the intermediate goods firms. \\
		In the firm-specific capital model, the capital stock is owned by the firms.
		\item Aggregate Supply:
		The final consumption good is produced under perfect competition using differentiated intermediate goods as inputs. Each intermediate good is produced by a monopolist employing capital (which is firm-specific in one variant of the model) and labor services. The production function is augmented by a technology shock. Capital is pre-determined. Hence, if capital is firm-specific, marginal costs depend positively on the firm's output level. Furthermore, it is assumed that the monopolistic firms have to pay the wage bill in advance which requires borrowing from a financial intermediary. Nominal frictions are introduced in the form of Calvo sticky prices. Non-reoptimizing firms index their prices to previous periods inflation.
		%\item The Foreign Sector: no foreign sector
		%\item Microeconomic foundation: yes
		\item Shocks: The common monetary policy shock, a neutral technology shock and an investment specific technology shock. % and the common fiscal policy shock which is added to the resource constraint.
		%\item Variable dimension: The model is log-linearized around the steady state. Variables are expressed in terms of percentage deviations from steady state.
		\item Calibration/Estimation: The model has been estimated by matching the empirical impulse response functions to a monetary policy shock in a ten variable VAR with the theoretical impulse responses from the model to a monetary policy shock. Quarterly U.S. data from 1959:Q2--2001:Q4 is used.
		\item Replication: Using the US\_ACELm model we replicated the impulse response functions for annualized quarterly inflation, output, annualized quarterly money growth and the annualized quarterly interest rate to a one standard deviation monetary policy shock.
		
		%\item Replication: Using the original money supply rule in \cite{AltigChristianoEichenbaumLinde2005} we can replicate the impulse responses generated with the code on L. J. Christianos website.\footnote{http://faculty.wcas.northwestern.edu/~lchrist/research/ACEL/acelweb.htm}
		
		%\item Impulse responses: Figure \ref{img:US_ACELm} (US\_ACELm), figure \ref{img:US_ACELt} (US\_ACELt), figure \ref{img:US_ACELswm} (US\_ACELswm), figure\ref{img:US_ACELswt} (US\_ACELswt).
		
		%\item Impulse responses: the first row of figure \ref{img:ACEL041} shows impulse responses to a one unit monetary policy shock. The solid line is the original model as in \cite{AltigChristianoEichenbaumLinde2005}. The model has an ordering of the variables in line with the recursive identification assumptions in the VAR. Using the Taylor rule, the output gap reacts in the period where the monetary policy shock occurs. Inflation reacts only one period afterwards. As the Taylor rule implies no persistence and  thus the output gap moves back to steady state after one period, there is no inflation reaction. The output gap is even positive rather than negative. Therefore, we plot impulse responses for a version of the model without a variable ordering (figure\ref{img:ACEL042}). This time the output gap reacts as expected. However due the fact that firms have to pay wage bills one period in advance (implies more persistence in inflation and might help to account for the price puzzle often observed in VARs) marginal costs and thus inflation increase on impact. Therefore a third version without this friction is shown (figure\ref{img:ACEL043}). Now all impulse responses look as expected. We observe strong inflation persistence and some persistence in the output gap. The second rows of the three figures show impulse responses to a one unit increase in output. This shock is ad-hoc and has been simply added to the resource constraint.
	\end{itemize}
	
	\subsection{US\_AJ16: \texorpdfstring{\cite{ajello2016financial}}{Ajello (2016)}}
	\label{USAJ16}
	\cite{ajello2016financial} develops a medium-scale model with financial frictions to analyze the role of US firm financing, in particular the financing gap, for business cycles and vice versa. In the model, shocks to financial intermediation play a major role for GDP and investment. 
	\begin{itemize}
		\item Aggregate Demand: The representative household is composed of a continuum of members. Household consume, hold bonds and accumulate net worth. The utility function is separable in consumption and leisure and features habit formation. Each period, all members receive an idiosyncratic shock that determines the productivity of their investments, and thus their investment behavior. In equilibrium, the members of the household can be divided into three subgroups: sellers who take up credit from banks and invest in new capital, keepers, who install new capital, but do not borrow funds, and buyers, who supply labor to firms, forego  investment into own productive capital and buy equity claims on other members' capital stock instead. The leverage of households is limited by a collateral constraint. Additionally, their financial claims are illiquid. Labor markets feature monopolistic supply and sticky nominal wages with wage indexation.
		
		\item Aggregate Supply: Intermediate good producers have a Cobb-Douglas Production function that features capital and labor. They are monopolistically competitive and set price subject to a Calvo friction with price indexation. Final good producers repackage intermediate goods and sell them as final goods. Investment good producers face convex investment adjustment costs.
		
		\item Financial Sector: Banks buy claims from sellers and sell them to buyers. Financial intermediation is subject to resource costs that create a spread between the ask and bid price of financial claims. 
		
		\item Shocks: The model features seven shocks: a TFP shock, a government spending shock, an interest rate shock, a shock on the spread, a discount factor shock, a price markup shock, and a wage markup shock.
		
		\item Calibration/Estimation: The log-linear model is estimated for the U.S. by means of Bayesian techniques for the period 1989Q1$-$2008Q2 using eight variables: GDP, consumption, investment, labor, wage rate, the nominal interest rate, inflation, the spread between BAA corporate bonds and ten-year Treasury notes, and the financing gap share. Measurement errors are introduced into the observation equations for the spread and the financing gap share.
		
	\end{itemize}

	\subsection{US\_BB18: \cite{balke2018oil}}
\label{USBB18}
In \cite{balke2018oil}, the authors consider the role of oil in a medium-sized DSGE framework that models the domestic U.S. economy as well as the rest of the world (ROW). Oil is supplied to and demanded from the U.S., and the U.S. domestic economy possesses an oil-producing sector.

\begin{itemize}
\item Aggregate Demand: Households maximize lifetime utility including consumption habit formation. Households allocate labor to production of final goods, intermediate goods, and oil production. Reallocating labor supply induces adjustment costs. Households use oil to produce capital services and are subject to investment adjustment costs. The model exhibits real wage frictions similar to those in \cite{blanchard2007macroeconomic}.

\item Aggregate Supply: The final good is produced under perfect competition via a CES production function that uses the intermediate good and transportation as inputs. The intermediate good results from the intermediate good producer’s profit maximization using labor and capital services as input. This optimization features nominal rigidities in the form of Rotemberg price adjustment costs. Transportation is equally modeled using a CES-type linear combination depending on labor and capital services, which need oil. The capital good supplier equivalently maximizes profits and uses oil as a direct input. The domestic economy has a stand-alone oil production sector using labor and capital as inputs.

\item Foreign sector: The rest of the world is modeled in a reduced form. It supplies and demands oil and has some economic activity. 

\item Shocks: The model features 13 shocks including domestic and ROW oil supply shocks.

\item Calibration/Estimation: Parameters and steady states in this paper are either calibrated or estimated using Bayesian inference taking advantage of quarterly data from 1991Q1--2015Q4.

\item Replication: In replicating this model, we simulate a world oil supply and a world oil demand shock as well as a shock to U.S. oil productivity in line with the impulse responses presented in the paper.
\end{itemize}
	
	
	\subsection{US\_BKM12: \cite{bils2012reset}}
	\label{USBKM12}
	\cite{bils2012reset} construct a two-sector model based on the model by  \cite{SmetsWouters2007} and re-estimate it on bimonthly data from 1990-2009. This is for comparing the behavior of actual and reset price ination to that for series simulated from the models, as the authors construct an empirical measure of reset price inflation on a bimonthly basis using US CPI micro data. They find that the models generate too much persistence and stability both in reset price inflation and in the way reset price inflation converted into actual inflation.
	
	\begin{itemize}
		\item Aggregate Demand: The same as in US\_SW07, except that the consumption good demanded by the households is now a composite of the goods of the two sectors.
		
		\item Aggregate Supply: The same as in US\_SW07, except that in US\_BKM12 there are two sectors that produce goods. Firms in the one sector faces sticky prices and aggregate their goods according to a Kimball aggregator, firms in the other sector can set their prices flexibly and their goods are aggregated according to a CES aggregator.
		
		\item Shocks: The same as in US\_SW07, except that the price markup shock in US\_SW07 is now replaced by a sector specific productivity shock to the sector with flexible price setting.
		
		\item Estimation: In the paper, the SW model is reestimated with Bayesian methods
		using seven bimonthly US observables over the period 1990:1 to 2009:10. Also, different
		from US\_SW07, the personal consumption deflator is used for price inflation instead of the
		GDP deflator. To be consistent with the MMB system, some parameters are adjusted to
		correspond to quarterly frequency.
	\end{itemize}
	
	
	\subsection{US\_CCF12: \cite{chen2012macroeconomic}}
	\label{USCCF12} 
	
	\cite{chen2012macroeconomic} simulate the Fed's second Large-Scale Asset Purchase program (LSAP II) in a medium-scale DSGE model with bond market segmentation (short- and long-term) estimated on US data. They find modest effects on GDP growth and inflation, but a lasting impact on the level of GDP. The effects would be even smaller absent a credible commitment to hold interest rates at its lower bound for an extended period of time.
	
	\begin{itemize}
		\item Aggregate Demand: Households are divided in unrestricted households that can trade in both, short- and long-term bonds, but face transaction costs for long-term bonds, and restricted households that can only trade in long-term bonds, but do not face transaction costs. Both types of households form habits in consumption and derive disutility from labor they supply to firms.
		
		\item Aggregate Supply: Competitive labor agencies combine differentiated labor inputs from households into a homogeneous labor composite. Competitive capital producers transform the consumption good into capital which they rent to intermediate goods producers. Monopolistic competitive intermediate goods producers hire labor and rent capital to produce intermediate goods. These are packaged into a homogeneous consumption good by competitive final goods producers.
		
		\item Government: The central bank sets the interest rate according to a conventional Taylor rule. The government supplies bonds. LSAP programs are interpreted as shocks to the composition of outstanding government liabilities compared with the historical behavior of these series.
		
		\item Shocks: Preference and labor supply shocks for restricted and unrestricted households, price markup shock, technology shock, investment-specific technology (adjustment cost) shock,  monetary policy shock, government spending shock, long-term bond supply shock, fiscal tax shock, and a risk premium shock.
		
		\item Estimation: The authors use Bayesian methods for estimation. The data is quarterly for the US from the period from 1987:3 to 2009:3 obtained from FRED and includes seven series: real GDP per capita, hours worked, real wages, core personal consumption expenditures deflator, nominal effective Federal Funds rate, the 10-year Treasury constant maturity yield, and the ratio between long-term and short-term US Treasury debt.
	\end{itemize}
	
	
	
	
	\subsection{US\_CCTW10: \cite{CoganCwikTaylorWieland2010}}
	\label{USCCTW10}
	
	\cite{CoganCwikTaylorWieland2010} examine the effect of fiscal policy stimulus using the Smets-Wouters model of the US economy (US\_SW07). They extend \cite{SmetsWouters2007} by introducing to the model rule-of-thumb consumers who spend all their disposable income. As the Ricardian equivalence property does not hold due to the presence of rule-of-thumb consumers, a fiscal policy rule is also included for determining a particular path for taxes.
	
	\begin{itemize}
		
		\item Aggregate Demand: There are two types of consumers. One is rule-of-thumb consumers and the other is forward-looking consumers identical to a representative household in \cite{SmetsWouters2007}. The rest of model is the same as in US\_SW07.
		
		\item Aggregate Supply: As in US\_SW07.
		
		\item Shocks: As in US\_SW07.
		
		\item Calibration/Estimation: The model is reestimated via Bayesian inference method with the same data set on US macroeconomic aggregates as in \cite{SmetsWouters2007}.
		
		%\item Replication: All impulse responses to different fiscal policy shocks, as appearing in \cite{CoenenMcAdamStraub2008}, have been replicated.
		
	\end{itemize}
	
	
	
	\subsection{US\_CD08: \cite{ChristensenDib2008}}
	\label{USCD08}
	\cite{ChristensenDib2008} develop and estimate a DSGE model characterized by price stickiness, capital adjustment costs and financial frictions with the aim of evaluating the importance of the financial accelerator in the amplification and propagation of the effects of the transitory shocks to the U.S. economy. {US\_CD08} is a closed economy model like in \cite{Ireland2003} enriched with financial frictions as in \cite{BernankeGertlerGilchrist1999}. The model is estimated in two versions, with and without the financial accelerator mechanism.
	
	\begin{itemize}
		
		\item Aggregate Demand: The representative household derives utility from consumption, real money balances and leisure. Consumption and real balances are subject to a preference shock and a money demand shock, respectively. The household keeps deposits at the financial intermediary, supplies labor to the entrepreneurs and earns dividends from retailer firms.
		
		\item Aggregate Supply: The production sector is comprised of entrepreneurs, capital producers and retailers. The set up introducing the financial frictions is similar to \cite{BernankeGertlerGilchrist1999}, apart from the fact that the debt contracts in \cite{ChristensenDib2008} are written in terms of the nominal interest rate. This specification allows for debt inflation effects, as unanticipated changes in inflation will affect the real cost of debt payment and the entrepreneurial net worth. Entrepreneurs borrow from financial intermediaries to buy capital from capital producers and produce intermediate goods. Due to asymmetric information between the entrepreneurs and financial intermediaries, the demand for capital is dependent on the entrepreneurs' financial conditions. Capital producers combine efficient investment goods and existing capital to produce new capital, subject to capital adjustment costs, which slow down the response of investment to different shocks. On the other side, retailers buy wholesale goods from entrepreneurs, differentiate them at no cost and sell them in a monopolistic competitive market, subject to price stickiness as in \cite{Calvo1983} and \cite{Yun1996}.
		
		\item Shocks: A preference shock, a money demand shock, a technology shock, an investment shock and a monetary policy shock.
		
		\item Calibration/Estimation: The model is estimated using a maximum-likelihood procedure with Kalman filter on quarterly U.S. data for the period 1979:Q3--2004:Q3.
		
		%\item Replication: Check the model in the Modelbase!
		
	\end{itemize}
	
	\subsection{US\_CET15: \texorpdfstring{\cite{christiano2015eichenbaum}}{Christiano et al. (2015)}}
	\label{USCET15}
	\cite{christiano2015eichenbaum} develop a medium-scale New Keynesian model that entails a detailed labor market and financial friction. They estimate the model, and use it to account for US dynamics in and after the Great Recession.
	
	\begin{itemize}
		\item Aggregate Demand: Households consume, supply labor and hold capital assets, riskless bonds, and money. Their utility function is separable in consumption and money. Consumption is composed of goods from home production and market production. The utility function features habit formation in both consumption types. Next to the labor income and capital income, households derive income from firms' profits, and potentially unemployment benefits. Furthermore, they have to pay lump sum taxes. Households can either be employed, unemployed or out of the work force. Their labor market state is determined in a search and matching framework similar to \cite{MortensenPissarides1994}. Wages are determined in an Alternative Offer Bargaining, developed by \cite{christiano2016unemployment}.
		
		\item Aggregate Supply: The production sector is comprised of final good producers, and retailers and wholesalers. Wholesalers employ labor as determined in the search and matching process and sell their product (intermediate goods) in perfect competition to retailers. Retailers produce their goods via a Cobb-Douglas production function employing capital and intermediate goods. The production function features fixed costs. Retailers are monopolistically competitive and face price stickiness as in the Calvo framework. Nonoptimizing retailers index their prices to inflation. Final good producers act in perfect competition. They buy retailers goods and bundle them to a homogenous final good. Capital accumulation (by the households) is subject to investment adjustment costs. 
		
		\item Shocks: The model features an interest rate shock, a TFP shock, an investment-specific shock and a government spending shock.
		
		\item Estimation: The model is estimated on US data from 1951:1-2008:4 using Bayesian methods such that the IRFs of the monetary policy shock and the two technology shocks match their counterparts derived form a VAR that is estimated on the same dataset.
	\end{itemize}
	
	
	
	\subsection{US\_CFOP14: \cite{Carlstrometal2014}}
	\label{USCFOP14}
	\cite{Carlstrometal2014} assess the importance of contract indexation in business cycle dynamics. The paper develops a mechanism for modeling financial frictions which builds on \cite{BernankeGertlerGilchrist1999} by allowing for contract indexation, and assumes a Costly State Verification framework as introduced by \cite{townsend1979optimal}. This mechanism is then imbedded into the medium-scale new-Keynesian model developed by \cite{Justinianoetal2011} and estimated by Bayesian techniques using US data on real, nominal and financial variables.
	\begin{itemize}
		\item Aggregate Demand: Households maximize their lifetime utility, where the utility function is separable in consumption and leisure and includes habit formation, subject to an intertemporal budget constraint. Households own firms and lenders, offer specialized labor in monopolistic competition, subject to \cite{Calvo1983} wage-stickiness with partial indexation to inflation and to employment agencies. They save in government bonds and deposits taken by lenders, and receive dividend payments and lump-sum transfers. Additionally, it is assumed that they have access to state-contingent securities which they trade among each other. Government spending is exogenous.
		\item Aggregate Supply: Perfectly competitive final good producers purchase intermediate goods from monopolistically-competitive producers and combine them through CES technology into a homogenous final good which can be used for consumption or investment. Intermediate good producers, which are subject to \cite{Calvo1983} price-stickiness with partial indexation to inflation, rent effective capital and purchase homogenous labor units to produce by means of a Cobb-Douglas production function with fixed production costs. Homogenous labor units are produced by perfectly competitive employment agencies which purchase specialized labor from the households. Perfectly competitive capital agencies manage the capital stock by renting out capital services to intermediate producers, while setting the utilization rate, and expanding the capital stock through investment. To invest, capital agencies linearly transform final goods into investment goods and transform investment goods into new capital, subject to adjustment costs. Risk neutral entrepreneurs, who purchase capital from capital agencies at the end of each period and sell it at the beginning of the next, are the sole accumulators of capital. They finance their capital purchase projects, which are subject to idiosyncratic risk, through their net worth and external financing from the lenders. The loan contract between the entrepreneurs and lender allows for the repayment rate to be state-dependent; at the optimum it is indexed to the return on holding capital.
		\item Shocks: An intertemporal preference shock affects households, intermediate firms' neutral technology factor and capital agencies' investment-specific productivity factor are unit root processes, wages and intermediate goods' prices are subject to mark-up shocks, capital agencies' marginal efficiency of investment is subject to an exogenous disturbance, entrepreneurs are subject to net worth and idiosyncratic risk shocks, and both government spending and the monetary policy rate are subject to shocks.
		\item Calibration/Estimation: The model is estimated using U.S. data by means of Bayesian techniques for the period 1972:1-2008:4 with ten macroeconomic variables: employment, inflation, the nominal interest rate, return to capital, the risk premium, real GDP, consumption, investment, real wage and relative price of investment. 
	\end{itemize}
	
	
	\subsection{US\_CFP17exo, US\_CFP17endo: \cite{carlstrom2017targeting}}
	\label{USCFP17}
	\cite{carlstrom2017targeting} build a quantitative DSGE model which features long-term bond purchases by the central bank, in order to analyze the effect of financial market segmentation and of term-premium targeting on the effectiveness of monetary policy. The model features private financial intermediaries within segmented financial markets in which the net worth of financial institutions limits the degree of arbitrage across the term structure. This is caused by a hold-up problem between households and banks. Through portfolio adjustment costs, central bank purchases of long-term bonds have a significant effect on long yields and thereby effect capital investment and the real economy. 
	\begin{itemize}
		\item Aggregate Demand: The representative household's utility is separable in consumption and leisure and features habit formation in consumption. Expected lifetime utility is maximized by choosing consumption and labor supply. The household has two options of intertemporal savings (short-term deposits and accumulation of physical capital). Also short-term government bonds can be held by households but are perfectly substitutable by deposits.
		\item Aggregate Supply: Perfectly competitive capital producers transform investment goods into new capital, facing investment adjustment costs. Monopolistic intermediate goods producers process labor and capital within a Cobb-Douglas production and set their price subject to nominal rigidities (\cite{Calvo1983}). The generated output is sold to final goods producing firms which repackage intermediate output and finally provide a consumption good. 
		\item Financial Sector: Banks engage in fund channeling and maturity transformation, i.e. they buy short- and long-term government bonds which are financed by accumulated net worth and deposits. There is no direct interaction between banks and firms in this model, effects of changes on the intermediaries' balance sheet are thus always channeled to firms via household investment. The bank`s objective is to maximize the stream of dividends to the households. The financial intermediary's net worth, however, and thereby also the size of its portfolio is subject to adjustment costs, which dampens the possibility to react to shocks. In addition, banks face a financial constraint: Their ability to attract deposits is limited by its net worth. A so called hold-up problem is used to implement this leverage constraint into the economy. Before shocks are realized, at the beginning of period t + 1, the bank may decide to default and not to repay its depositors. As a result, the fraction of financial intermediaries' assets which can still be utilized by depositors is limited. Hence, there is a compatibility constraint to ensure repayment of the depositing household. Generally, bank intermediation is required because new household investment must be financed via new debt issuance.
		\item Shocks: There are eight shocks in the model: productivity shock, credit shock, investment shock, monetary policy shock, natural rate shock, wage markup shock, price markup shock, QE shock.
		\item Calibration/Estimation: Several parameters are calibrated to match long-run features of US data. Evidence on interest rate spreads and leverage is used to pin down the steady-state loan-to-deposit spread and the leverage ratio in the model. Parameters are calibrated as to match a term premium of 100 annual basis points and a steady-state leverage ratio of 6. This is the same calibration as in \cite{GertlerKaradi2013}. Government bonds are calibrated to a duration of 40 quarters. The steady state balance sheet of financial intermediaries is calibrated to consist of 40% government securities. The remaining calibration is standard. 
	\end{itemize}
	The difference between the two model versions is rooted in the behaviour of the level of long-term debt on the balance sheet of the financial intermediaries. In US\_CFP17exo, central bank bond purchases and changes to the mix of short-term and long-term debt by the fiscal authority are modeled by exogenous movements in long-term debt. Consequently, the long-term yield will be endognous. Contrary, in US\_CFP17endo, the central bank pegs the term premium and hence the level of long-term debt will be endogenous. 
	
	
	
	\subsection{US\_CMR10, US\_CMR10fa: \cite{Christianoetal2010}}
	% \subsection{: \cite{Christianoetal2010} - small version with financial accelerator} \\
	\label{USCMR10}, \label{USGMR10fa}
	The US\_GMR10 model combines a standard DSGE model like \cite{SmetsWouters2003} and \cite{SmetsWouters2007} with a detailed financial sector based on agency problem borrowing from \cite{BernankeGertlerGilchrist1999} to investigate the role of financial factors in business cycles. Several mechanism are imbedded into the baseline model due to several apparatus for financial frictions: the financial accelerator channel, the Fisher deflation channel and the bank funding channel. The economy consists of households, intermediate-good producing firms, final-goods producing firms, capital producers, entrepreneurs, bank and government. The US\_GMR10fa model considers the financial accelerator channel shutting down the bank funding channel, namely ignoring the bank's supply of liquidity and household's demand for money. \\
	
	\begin{itemize}
		\item Aggregate Demand: Households obtain utility from having consumption and liquidity services and disutility from supplying labor services and adjusting real currency holdings. Household provide labor supply labor under monopolistic competition and make a portfolio decision over high powered money, bank deposits, short-term marketable securities and other financial securities.
		\item Aggregate Supply: Monopolistically competitive intermediate-good producing firms maximize profit using labor and capital (rented from entrepreneurs) subject to a Calvo price setting. They have to pay for working capital in advance of production. Perfectly competitive final-good producing firms aggregate a variety of intermediate goods. The final goods are then converted into consumption, investment and government goods. Capital producers combine investment goods with used capital purchased from entrepreneurs to produce new capital facing the convex investment cost function. Entrepreneurs own the stock of physical capital, buy new capital using their own wealth as well as bank loans, provide capital services while choosing the capital utilization rate. Government spending is modeled as a certain fraction of final good and are financed by lump-sum taxes levied to households.
		\item Financial System: Since a shock on entrepreneurial investments is idiosyncratic and privately-observable, it incurs the monitoring cost to banks.
		
		\item Shocks: The model include 16 shocks: a banking technology shock, a bank reserve demand shock, a term premium shock, a investment specific shock, a money demand shock, a government consumption shock, a persistent productivity shock, a transitory productivity shock, a financial wealth shock, a risk shock, a consumption preference shock, a shock on marginal efficiency of investment, an oil price shock, a price mark-up shock and an inflation target shock. All shocks are assumed to follow AR(1) process but an inflation target shock and a monetary policy shock which are treated as an i.i.d process.
		\item Calibration/Estimation: The model is estimated by standard Bayesian methods using quarterly data from 1985Q1 to 2008Q2 for the Euro Area (EA) and for the United States (US). In the baseline estimation 16 variables are used with consideration of a measurement error: GDP, Consumption, Investment, GDP deflator, real wages, hours worked, the relative price of investment, the relative price of oil, the short-term interest rate, the stock market, a measure of the external finance premium, real credit, two definitions of real money growth, bank reserves and the term spread.
	\end{itemize}
	
	
	\subsection{US\_CMR14, US\_CMR14noFA:\cite{CMR2014}}
	\label{USCMR14}	
	\cite{CMR2014} augment a standard DSGE model such as \cite{SmetsWouters2003} or \cite{SmetsWouters2007} with a financial accelerator mechanism as in \cite{BernankeGertlerGilchrist1999}. In particular, the return on capital of individual entrepreneurs is subject to idiosyncratic uncertainty. The model is fitted to US data, while modeling aggregate risk as the innovation to the variance of the distribution determining the return on capital. The paper's main-finding is that fluctuations in risk are the most important shock driving the business cycle.
	\begin{itemize}
		\item Aggregate Demand: Households maximize expected lifetime utility by choosing consumption of final goods, labor supply and investment. They obtain funds from supplying labor, purchasing long- and short-term bonds, building and selling raw capital, as well as from various lump-sum transfers. Further, each household is subject to taxes on consumption and labor income.
		\item Aggregate Supply: Competitive final-goods producers purchase and combine intermediate goods from monopolistic intermediate-goods producers. These produce by employing labor and renting capital while subject to Calvo-style rigidities. Homogenous labor units are produced by perfectly competitive labor contractors which aggregate differentiated household labor services purchased from monopolistic unions that set wages subject to Calvo-style frictions. Households build raw capital subject to capital-adjustment costs and sell it to entrepreneurs, which they own.
		\item Financial Sector: Risk-neutral entrepreneurs finance their purchases of capital through their net worth and loans from competitive mutual funds. The loan contract between entrepreneurs and mutual funds is as in \cite{BernankeGertlerGilchrist1999}. However, the authors introduce a shock to the variance of idiosyncratic productivity that influences individual entrepreneur's return on capital. It is referred to as a risk shock. With an agency problem between entrepreneurs and
		mutual funds, a positive risk shock increases the required return on borrowing, that
		is, the external finance premium.
		\item Shocks: The model includes shocks to the following 12 variables: price markup, price of investment goods, government consumption, technology growth persistence, technology (transitory), risk, consumption preference, marginal efficiency of investment, term structure, equity, monetary policy, and the inflation target.
		\item The model is estimated by Bayesian techniques using 12 quarterly observables covering the period 1985:Q1 to 2010:Q2. The data set includes 8 macroeconomic and 4  financial variables. 
		\item US\_CMR14noFA is the version of US\_CMR14 where the financial accelerator channel has been muted. The parametrization is however that of the baseline model.      
	\end{itemize}
	
	
	\subsection{US\_CPS10: \cite{Cogleyetal2010}}	
	\label{USCPS10} 
	\cite{Cogleyetal2010} estimate a New Keynesian model based on \cite{RotembergWoodford1997} and \cite{boivin2006has}. The model is applied to provide a structural explanation of the feature that the inflation gap became less persistent after the Volcker disinflation. The key difference to a number of other mainstream models is that it allows for the inflation target to change over time. The main finding of the paper is that the most important factor explaining the change in the inflation gap persistence is the decline in the variance of the inflation target shock. Yet, changes in the non-policy parameters also contributed to the decline in the inflation gap persistence significantly.
	\begin{itemize}
		\item Aggregate Demand: The representative household maximizes lifetime utility subject to an intertemporal budget constraint. Utility from consumption and disutility from labor is separable. Preferences for consumption are subject to habit persistence. The representative household offers a continuum of different types of laborto the firms. Furthermore, the household owns the firms, obtains their profits and receives also income from labor.
		\item Aggregate Supply: There exists a continuum of monopolistically competitive firms. Price stickiness is embedded into the model via the \cite{Calvo1983} framework. Each good is produced using linear technology and the sole production factor is labor. Technology follows a unit root process and its growth rate is modeled by an exogenous AR(1)-process.
		\item Shocks: An intertemporal preference shock, a price markup shock, a technology shock and an inflation target shock.
		\item Calibration/Estimation: The model is estimated using Bayesian methods on quarterly U.S. data for two subsamples: 1960:Q1-1979:Q3 and 1982:Q4-2006:Q4.
	\end{itemize}
	
	
	
	\subsection{US\_DG08: \cite{DeGraeve2008}}
	\label{USDG08}
	\cite{DeGraeve2008} uses a medium-scale New Keynesian model like in \cite{SmetsWouters2007} enriched with financial frictions as in \cite{BernankeGertlerGilchrist1999} to estimate and explore the role of the external finance premium in propagating shocks for the U.S. economy. Conditional on certain shocks, he finds that a framework with financial frictions and investment adjustment costs may give rise to a financial ``decelerator''.
	
	\begin{itemize}
		
		\item Aggregate Demand: As in \cite{SmetsWouters2007}, households maximize their lifetime utility function, non-separable in consumption and leisure, subject to an intertemporal budget constraint. Preferences for consumption are subject to habit persistence. They own firms, hold financial wealth in the form of one-period, state-contingent bonds and supply labor monopolistically. Wage stickiness is introduced via the Calvo framework.
		
		\item Aggregate Supply: Apart from the intermediate and final goods firms as in \cite{SmetsWouters2007}, a financial intermediary, capital goods producers and entrepreneurs are introduced in the model to match the structure as in \cite{BernankeGertlerGilchrist1999} and \cite{ChristianoMottoRostagno2003}. Intermediate goods firms face price rigidity \`{a} la Calvo while capital good producers face convex investment adjustment costs. On the other side, the presence of entrepreneurs and the financial intermediary brings financial frictions into play. Entrepreneurs borrow from financial intermediaries to buy capital (from capital producers), decide on capital utilization, rent capital services to intermediate goods firms and sell non-depreciated capital back to capital producers. However, after the purchase of the capital stock, entrepreneurs are hit by an idiosyncratic shock, observable only by them. This leads to the costly state verification framework \`{a} la \cite{BernankeGertlerGilchrist1999}, giving raise to extra costs, above the risk-free rate. The optimal contract between entrepreneurs and the financial intermediary leads to an aggregate relationship of the spread between the external finance costs and the risk-free rate and entrepreneurs' financial conditions represented by the leverage ratio.
		
		\item Shocks: A preference shock, a labor supply shock, a total factor productivity shock, an investment technology shock, a government spending shock, an inflation target shock, a monetary policy shock, a wage and price mark-up shock.
		
		\item Calibration/Estimation: The model is estimated using Bayesian methods on quarterly U.S. data for the period 1954:Q1--2004:Q4.
		
		%\item Replication: Replication is conducted in terms of variance decomposition and impulse response functions as given in \cite{DeGraeve2008}. (Melanie writes that the replication matches the results in \cite{DeGraeve2008}. From the IRFs given in the report the match is close but for the variance decomposition I cannot say so).
	\end{itemize}
	
	
	
	\subsection{US\_DNGS15 : \cite{del2015inflation}} 
	\label{USDNGS15}
	\cite{del2015inflation} build a medium-scale New Keynesian model that can predict a sharp contraction in economic activity along with a protracted but relatively modest decline in inflation, following the Great Recession. They build upon a standard DSGE model (like in \cite{SmetsWouters2007}) enriched
	with financial frictions and a time-varying target inflation rate.
	\begin{itemize}
		\item Aggregate Demand: As in \cite{SmetsWouters2007}, households maximize
		a nonseparable utility function with two arguments (goods and labor effort)
		over an infinite life horizon, subject to an intertemporal budget constraint.
		Preferences for consumption are subject to habit persistence. Households supply labor monopolistically and wage stickiness is introduced via the Calvo framework. 
		\item Aggregate Supply: Monopolistically competitive firms produce intermediate goods, which a competitive firm aggregates into a single final good that is used for both consumption and investment. The intermediate goods firms decide on labor and capital inputs, and set prices according to the Calvo model.
		\item Financial Sector: Building on the work of \cite{BernankeGertlerGilchrist1999}, \cite{CMR2014}, and \cite{DeGraeve2008}, a financial intermediary, capital producers and entrepreneurs are introduced in the model in addition to the intermediate and final goods firms as in \cite{SmetsWouters2007}. Financial frictions come into play by the presence of entrepreneurs and the financial intermediary. Banks collect deposits from households and lend to entrepreneurs who use these funds as well as their own wealth to acquire physical capital, which is then rented to intermediate goods producers.
		Entrepreneurs are subject to idiosyncratic disturbances that affect their ability to manage capital which leads to the costly state verification framework as in \cite{BernankeGertlerGilchrist1999} and gives rise to a spread, above the risk-free rate. This spread is thus a function of the entrepreneurs' leverage and riskiness.
		\item Shocks: The model features a preference shock, a financial friction shock, a total factor productivity shock, an investment specific technology shock, a government spending shock, an inflation target shock, a monetary policy shock, a wage and price mark-up shock.
		\item Calibration/Estimation: The model is estimated using Bayesian methods on
		quarterly U.S. data for the period 1964:Q1 - 2008:Q3.
	\end{itemize}
	We furthermore include three variants of the model proposed and estimated by DNGS: A version without financial frictions and time-varying inflation target (DNGS15\_SW), a version without financial frictions (DNGS15\_SWpi) and the Smets-Wouters 2007 model estimated using the same variables as the original authors using 2012Q3 data.
	
	\subsection{US\_FGKR15 : \cite{FernandezVillaverdeetal2015}}
	\label{USFGKR15}
	The purpose of the authors is to quantify the effects of fiscal volatility shocks on the dynamics of key macroeconomic variables. They rely on a medium-scale New Keynesian DSGE model in the spirit of \cite{ChristianoEichenbaumEvans2005}, which they augment with fiscal policy. Particularly, the fiscal authority has four instruments at its disposal: government expenditure and taxes on capital income, labor income as well as consumption. Fiscal rules are standard, i.e. the fiscal authority reacts to changes in output and debt-to-GDP ratio. However, the standard deviations of fiscal variables are not constant, but follow exogenous stochastic processes. As a result, this specification allows for a clear distinction between fiscal shocks and fiscal volatility shocks. Since the latter are shocks to the second moment, the model has to be solved using (at least) a third-order perturbation approach. %Currently, the modelbase contains a version of the model that is solved using a first-order perturbation approach. Hence, fiscal volatility shocks are not present in this version of the code. The model features only fiscal (level) shocks.
	\begin{itemize}
		\item Aggregate Demand: The representative household's utility is separable in consumption, government expenditure and leisure and allows for (external) habit formation in consumption. The household can invest in capital and hold government bonds. Investment in capital is subject to adjustment costs. Wages are sticky (\cite{Rotemberg1982} adjustment costs). The household pays consumption taxes, labor income taxes, capital income taxes as well as lump-sum taxes. 
		\item	Aggregate Supply: The final consumption good is produced under perfect competition using differentiated intermediate goods as inputs. Intermediate goods producers operate in monopolistically competitive environment and employ capital and labor services as factors of production. Prices are sticky (\cite{Rotemberg1982} adjustment costs).
		\item Shocks: A preference shock, (a labor-augmenting) productivity shock, a monetary policy shock and (four) fiscal shocks
		\item 	Calibration/Estimation: Some of the model parameters are estimated using simulated method of moments (SMM) to match US quarterly data moments, the others are set to conventional values prior to estimation.
	\end{itemize}
	
	
	\subsection{US\_FM95: \cite{FuhrerMoore1995}}
	\label{USFM95}
	The model is described in \cite{FuhrerMoore1995} and \cite{FuhrerMoore1995a}. We employ the parametrization used in \cite{LevinWielandWilliams2003}.
	Fuhrer and Moore introduce a new wage contracting model where agents care about relative real wages in order to match the strong inflation persistence observed in U.S. data. %Up to that time the Phelps-Taylor model of overlapping wage contracts was used where the inflation rate is more flexible so that disinflations without output losses are possible. The US\_FM95 model implies additional inflation persistence beyond that imparted by the persistence of the driving output process and thus yields a realistic sacrifice ratio (see \cite{FuhrerMoore1995}). Additionally, as remarked in \cite{FuhrerMoore1995a} the model provides a structural explanation of the correlation between real output and the short-term nominal interest rate, by linking long- and short-term rates via an intertemporal arbitrage using a Macaulay duration term.
	\begin{itemize}
		\item Aggregate Demand: The US\_FM95 model represents aggregate spending by a single reduced-form equation corresponding to an IS curve. The current output gap depends on its lagged values over the past two quarters and the lagged value of the long-term real interest rate, which is defined as a weighted average of ex-ante short-term real interest rates with a duration of 40 quarters. %The FM model does not explicitly include trade variables or exchange rates; instead, net exports (and the relationship between real interest and real exchange rates) are implicitly incorporated in the IS curve equation.
		\item Aggregate Supply: The aggregate price level is a constant mark-up (normalized to one) over the aggregate wage rate. The aggregate wage dynamics are determined by overlapping wage contracts. In particular, the aggregate wage is defined to be the weighted average of current and three lagged values of the contract wage rate. The real contract wage, that is the contract wage deflated by the aggregate wage, is determined as a weighted average of expected real contract wages, adjusted for the expected average output gap over the life of the contract. This specification yields a hybrid Phillips curve that depends additionally on current and past demand and expectations about future demand.
		%\item The Foreign Sector: no explicit inclusion of foreign variables
		%\item Microeconomic foundation: no
		\item Shocks: An ad hoc supply shock and the common monetary policy shock.
		%\item Variable dimension: Original variables are expressed in percent/100. The common Modelbase variables are expressed in percent.
		\item Calibration/Estimation: Full-information maximum likelihood estimation on U.S. data from 1966--1994.
		\item Replication: We replicated the impulse response functions for annualized quarterly inflation and the output gap to a 100 basis point innovation to the federal funds rate in Figure 2 of \cite{LevinWielandWilliams2003}.
		%\item Notes: even though today the Calvo mechanism of price rigidities is commonly used the FM model is not outdated. See for example \cite{Guerrieri2006} for its empirical relevance and \cite{DixonKara2006} and \cite{Kiley2002} for a comparison of Taylor contracts and Calvo rigidities.
		%\item Impulse responses: Figure \ref{img:US_FM95}.
		%\item Impulse responses: The first row of figure \ref{img:FM95} shows impulse responses to a one unit monetary policy shock. One can see that the IS curve is backward looking and thus output responses with a lag of one period. Inflation is extremely persistent due to the real wage contracts Phillips curve. Output is also very persistent: the coefficient in the IS curve on lagged output is 1.4447 and the coefficient on the second lag is -0.468. The dynamics of the supply and demand block have only a weak link. Demand enters the contract wage equation with a weight of $0.02$ only. The second row shows impulse responses to a one unit demand shock that is added to the IS curve.
	\end{itemize}
	
	
	
	\subsection{US\_FMS13 : \cite{Feveetal2013}}
	\label{USFMS13} 
	\cite{Feveetal2013} use specifications from \cite{Justinianoetal2011} which is an extension of the standard \cite{SmetsWouters2007} model. In addition to the standard features of the model, such as habit formation, investment adjustment costs, nominal rigidities, etc., they also allow for endogenous public spending and Edgeworth complementarity/substitutability. These two novelties of the model enable to take the countercyclicality of fiscal policy into account and to better match the observed correlation between the growth rates of government spending and private consumption.
	\begin{itemize}
		\item Aggregate Demand: Households maximize their lifetime utility, where the utility function is separable in consumption and leisure, subject to an intertemporal budget constraint and allowing Edgeworth complementarity and substitutability. The model allows for capital accumulation, habit formation in leisure decisions, and multiple shocks. Households own firms, rent capital services to firms and decide how much capital to accumulate given certain capital adjustment costs. They additionally hold their financial wealth in the form of one-period, state-contingent bonds. The endogenous fiscal policy component is countercyclical and the stochastic component is assumed to follow a first-order auto-regressive process.
		\item Aggregate Supply: As in US\_SW07.
		\item Shocks: A monetary policy shock, fiscal shock, mark-up shocks, intertemporal preference shock, investment shock and a technology shock.
		\item Calibration/Estimation: The model parameters are estimated by using Bayesian techniques on the following US quarterly data between 1960q1 and 2007q4: consumption growth, investment growth, real government expenditures growth, growth rate of real hourly compensation, the log of hours worked, the inflation rate and nominal interest rate.
	\end{itemize}
	
	\subsection{US\_FRB03: FRB-US model} 
	\label{USFRB03}
	%Description?
	\subsection{US\_FRB08: FRB-US model}
	\label{USFRB08}
	\label{USFRB08mx}
	The FRB model is a large-scale model of the U.S. economy with a relatively detailed representation of the supply side of the economy. The version US\_FRB03 was linearized by \cite{LevinWielandWilliams2003}.
	%whereas the other two versions were linearized by \cite{BraytonLaubach2008}.
	%The US\_FRB08 model exhibits model-consistent expectations only. The US\_FRB08mx model %allows the user to choose between VAR-based and rational expectations.
	
	\begin{itemize}
		\item Aggregate Demand: Real spending is divided into five components: private consumption, fixed investment, inventory investment, net exports and government purchases. The broad components are disaggregated further i.e. spending on fixed investment is separated into equipment, nonresidential structures and residential construction. Government spending is divided into six sub-components, each of which follows a simple reduced-form equation that includes a counter-cyclical term. The specification of most non-trade private spending equations follows the generalized adjustment cost model due to \cite{Tinsley1993}.
		\item Aggregate Supply: Potential output is modeled as a function of the labor force, crude energy use, and a composite capital stock, using a three-factor Cobb-Douglas production technology. The equilibrium output price is a mark-up over a weighted average of the productivity-adjusted wage rate and the domestic energy price. The specification of the wage and price dynamics follows the generalized adjustment cost framework used in the aggregate demand block. Wage inflation depends on lagged wage inflation over the previous three quarters, as well as expected future growth in prices and productivity, and a weighted average of expected future unemployment rates. Price inflation depends on its own lagged values over the past two quarters, as well as expected future changes in equilibrium prices and expected future unemployment rates. In addition, both wages and prices error-correct to their respective equilibrium levels. A vertical long-run Phillips curve is imposed in estimation. The model contains a detailed accounting of various categories of income, taxes, and stocks, an explicit treatment of labor markets, and endogenous determination of potential output. Long-run equilibrium in the model is of the stock-flow type; the income tax rate and real exchange rate risk premium adjust over time to bring government and foreign debt-to-GDP ratios back to specified (constant) levels.
		\item Foreign sector: The full model includes detailed treatments of foreign variables. Twelve sectors (countries or regions) are modeled, which encompass the entire global economy. In the model used in the Modelbase the full set of equations describing the foreign countries is replaced by two reduced form equations for foreign output and prices, to reduce computational cost.
		\item Shocks: The model exhibits a large range of shocks to which we add the common monetary policy shock and a fiscal shock that equally affects all three components of federal government spending such that a unit fiscal policy shock affects output by 1 percent.
		%\item Variable dimension: Original variables are expressed in percent/100. The common Modelbase variables are expressed in percent.
		%\item Impulse responses: Figure \ref{img:US_FRB03} (US\_FRB03), figure \ref{img:US_FRB08} (US\_FRB08), figure \ref{img:US_FRB08mx} (US\_FRB08mx).
		\item Replication: We replicated the impulse response functions for annualized quarterly inflation and the output gap to a 100 basis point innovation to the federal funds rate in Figure 2 of \cite{LevinWielandWilliams2003}.
		%\item Replication: Using the monetary policy rule by \cite{LevinWielandWilliams2003}, which is included in the modelbase, we can replicate the inflation and output gap impulse response functions and autocorrelation functions of the FRB03-model in \cite{LevinWielandWilliams2003} figure 1 and 2. We compared the impulse response functions of the FRB08 model with and without rational expectations after a monetary policy shock generated by the original AIM-Code with the IRFs generated by Dynare and obtained the same results.
		%\item Impulse responses: The first row of figure \ref{img:FRB03} shows impulse responses to a one unit monetary policy shock for the FRB03 model. The second row shows impulse responses to an increase in federal government consumption of one percent of GDP (WE NEED TO RECHECK THIS! NOT SURE YET, IF THIS IS RIGHT!). Figure \ref{img:FRB08} shows the respective impulse responses for the FRB08 model and figure \ref{img:FRB08mx} for the FRB08mx model which deviates from rational expectations.
	\end{itemize}
	
	
	\subsection{US\_FRB22: FRB-US model}
\label{USFRB22}
\cite{brayton2022linver} build the linear version (LINVER) of the FRB/US model. \cite{brayton2014frb} is a large-scale model of the U.S. economy. For some model parts agents in this model form model-consistent expectations, while for other parts expectations are based on the average historical dynamics of the predictions of estimated limited-information VAR models. Consequently, we include 4 model variants.
\begin{itemize}
\item Aggregate Demand: There are liquidity-constrained and unconstrained households. Liquidity-constrained households spend all their income each quarter. In contrast, unconstrained households consume and invest based on their assessment of their lifetime resources. Future labor and transfer income is discounted at a rate substantially higher than the discount rate on future income from non-human wealth. Unconstrained households face adjustment costs that cause them to adjust their spending gradually in response to changes in expected income and property wealth. As in the national income and product accounts, total spending by households consists of consumption of nondurable goods and non-housing services, purchases of durable consumer goods, and consumption of housing services; movements in these three components of total spending are modeled separately. 

\item Aggregate Supply: The key production sector is the nonfarm business sector plus imported energy. The production function in this sector is Cobb-Douglas with potential output depending on the sustainable full-employment level of labor input, actual capital services, trend energy services, and the trend component of multi-factor productivity. The key inflation measures modeled in FRB/US are for core PCE prices and ECI hourly compensation, following the New Keynesian Phillips curve specification in the presence of nonzero trend inflation developed in \cite{cogley2008trend}. In addition to slack and expectations of future inflation, other important determinants of total consumer price inflation include movements in the relative prices of food, energy, and non-energy imports. The government sector includes disaggregated components of spending and a wide range of tax rates and credits.

\item Foreign sector: The model includes imports and exports of goods and services that depend on real activity in the rest of the world and the terms of trade. The trade-weighted dollar exchange rate is modeled assuming uncovered interest parity.

\item Shocks: The model exhibits a large range of shocks including a fiscal shock that equally affects all three components of federal government spending such that a unit fiscal policy shock affects output by 1 percent.

\item Calibration/Estimation: The model is partly calibrated and includes an involved estimation routine.
\end{itemize}

	
	
	\subsection{US\_FU19: \texorpdfstring{\cite{fratto2019uhlig}}{Fratto and Uhlig} }
	\label{USFU19}
	
	\cite{fratto2019uhlig} investigate on the missing deflation puzzle by estimating versions of the \cite{SmetsWouters2007} on different samples of US data that include or exclude the years after the Financial Crisis. They find that markup shocks account for the almost all of the variation in inflation before and after the crisis. In the MMB, we parametrize the model according to the estimates on 1984-2015 data. 
	
	\begin{itemize}
		
		\item Aggregate Demand: Households maximize their lifetime utility, where the utility function is nonseparable in consumption and leisure, subject to an intertemporal budget constraint. \cite{SmetsWouters2007} include external habit formation to make the consumption response in the model more persistent. Households own firms, rent capital services to firms and decide how much capital to accumulate given certain capital adjustment costs. They additionally hold their financial wealth in the form of one-period, state-contingent bonds. Exogenous spending follows a first-order autoregressive process with an iid-normal error term and is also affected by the productivity shock.
		
		\item Aggregate Supply: The final goods, which are produced under perfect competition, are used for consumption and investment by the households and by the government. The final goods producer maximizes profits subject to a \cite{Kimball1995} aggregator of intermediate goods, which introduces monopolistic competition in the market for intermediate goods and features a non constant elasticity of substitution between different intermediate goods, which depends on their relative price. A continuum of intermediate firms produce differentiated goods using a production function with Cobb-Douglas technology and fixed costs and sell these goods to the final-good sector. They decide on labor and capital inputs, and set prices according to the Calvo model. Labor is differentiated by a union using the Kimball aggregator, too, so that there is some monopoly power over wages, which results in an explicit wage equation. Labor packers buy the labor from the unions and resell it to the intermediate goods producer in a perfectly competitive environment. Sticky wages à la Calvo are additionally assumed. The Calvo model in both wage and price setting is augmented by the assumption that prices that can not be freely set, are partially indexed to past inflation rates.
		
		\item Shocks: A total factor productivity shock, a risk premium shock, an investment-specific technology shock, a wage and a price mark-up shock and two policy shocks: the common fiscal policy shock entering the government spending equation and the common monetary policy shock. 
		
		\item Estimation: The model is estimated for the U.S. with Bayesian techniques for the period 1984Q1-2015Q4 using seven key macroeconomic variables: real GDP, consumption, investment, the GDP deflator, real wages, employment and the nominal short-term interest rate. The replication package additionally contains the baseline version of the model estimated on a shorter sample (1984Q1-2007Q4).
		
	\end{itemize}
	
	
	\subsection{US\_FV10: \cite{fernandez2010econometrics}}
	\label{USFV10}
	\cite{fernandez2010econometrics} employs a canonical medium-scale closed economy DSGE-Model similar to \cite{SmetsWouters2007}, estimated on U.S. data. The model features a deterministic growth rate driven by labor-augmenting technological progress, so that the data do not need to be detrended before estimation. The code is written in non-linearized form.
	\begin{itemize}
		\item Aggregate demand: Households maximize their lifetime utility, where the utility function is separable in consumption, leisure and real money balances, subject to an intertemporal budget constraint. Consumption utility is subject to habit formation. Households own firms, rent capital services to firms and decide on investment given certain investment adjustment costs. 
		\item Aggregate Supply: The final goods, which are produced under perfect competition, are used for consumption and investment by the households. The final goods producer aggregates intermediate goods using a constant elasticity of substitution (CES) production function. A continuum of monopolistically competitive intermediate firms produce differentiated goods using a production function with Cobb-Douglas technology and fixed costs and sell these goods to the final-good sector. They decide on labor and capital inputs, and set prices according to the Calvo model. Labor is differentiated by a union using the CES aggregator, too, so that there is some monopoly power over wages, which results in an explicit wage equation. Labor packers buy the labor from the unions and resell it to the intermediate goods producer in a perfectly competitive environment. Sticky wages \'a la Calvo are additionally assumed. The Calvo model in both wage and price setting is augmented by the assumption that prices that cannot be freely set, are partially indexed to past inflation rates.
		\item Shocks: A total factor productivity shock, an investment-specific technology shock, an intertemporal preference shock, an intratemporal preference shock and a monetary policy shock.
		\item Calibration/Estimation: The model is estimated for the U.S. with Bayesian techniques for the period 1959:1$-$2007:1 using five key macroeconomic variables: relative price of investment, real output per capita growth, real wages per capita, CPI inflation and the federal funds rate. 
		\item Replication: We simulated the impulse response functions to a positive one standard deviation monetary policy shock and technology shock. While FV (2010) does not show any impulse responses, our simulated IRFs are very similar to the impulse responses provided in the technical appendix to FV (2015). 
	\end{itemize} 
	
	
	\subsection{US\_FV15: \cite{fernandez2015estimating}}
	\label{USFV15}
	\cite{fernandez2015estimating} employs a canonical medium-scale closed economy DSGE-Model similar to \cite{SmetsWouters2007}, estimated on U.S. data, but augmented with time-varying volatility in the shocks. The model features a deterministic growth rate driven by labor-augmenting technological progress, so that the data do not need to be detrended before estimation. The code is written in non-linearized form. 
	\begin{itemize}
		\item Aggregate demand: Households maximize their lifetime utility, where the utility function is separable in consumption, leisure and real money balances, subject to an intertemporal budget constraint. Consumption utility is subject to habit formation. Households own firms, rent capital services to firms and decide on investment given certain investment adjustment costs. 
		\item Aggregate Supply: The final goods, which are produced under perfect competition, are used for consumption and investment by the households. The final goods producer aggregates intermediate goods using a constant elasticity of substitution (CES) production function. A continuum of monopolistically competitive intermediate firms produce differentiated goods using a production function with Cobb-Douglas technology and fixed costs and sell these goods to the final-good sector. They decide on labor and capital inputs, and set prices according to the Calvo model. Labor is differentiated by a union using the CES aggregator, too, so that there is some monopoly power over wages, which results in an explicit wage equation. Labor packers buy the labor from the unions and resell it to the intermediate goods producer in a perfectly competitive environment. Sticky wages \`{a}  la Calvo are additionally assumed. The Calvo model in both wage and price setting is augmented by the assumption that prices that cannot be freely set, are partially indexed to past inflation rates.
		\item Shocks: A total factor productivity shock, an investment-specific technology shock, an intertemporal preference shock, an intratemporal preference shock and a monetary policy shock. The standard deviations of the structural innovations are subject to stochastic volatility shocks. The model also includes shocks to the two parameters in the monetary policy rule.  
		\item Calibration/Estimation: The model is estimated for the U.S. with Bayesian techniques for the period 1959:1$-$2007:1 using five key macroeconomic variables: relative price of investment, real output per capita growth, real wages per capita, CPI inflation and the federal funds rate. 
		\item Replication: We replicated the impulse response functions to a positive one standard deviation monetary policy shock and technology shock, as shown in Figure 6.1 and 6.2 of the technical appendix.
	\end{itemize}
	
	
	\subsection{US\_HL16: \texorpdfstring{\cite{hollander2016liu}}{Hollander and Liu (2016)}}
	\label{USHL16}
	\cite{hollander2016liu} analyse the role of the equity price channel in business cycle fluctuations. They incorporate the financial accelerator channel and the bank equity channel into a medium-scale New-Keynesian DSGE model. Through these two channels, the equity price channel amplifies shocks to the real economy. The model reproduces the procyclicality of the equity price found in the data.
	
	\begin{itemize}
		
		\item Aggregate Demand: There are two types of representative households, saver and borrower households. Both maximize their expected lifetime utility that depends on consumption, labour, deposits and equity investments, subject to budget constraints. Households' consumption preferences exhibit habit formation. In addition to the budget constraint, borrower households also face a borrowing constraint. Households supply labour and wages are flexible in the model.
		
		\item Aggregate Supply: Entrepreneurs produce the wholesale good using capital and labour as inputs. They face direct costs of adjusting their capital stock. Retailers buy the intermediate goods at the wholesale price, differentiate them at no cost, and sell the final good with a mark-up. Prices are subject to nominal rigidities \`{a}  la \cite{Calvo1983}.
		
		\item Banking sector: The banking sector in the model builds on \cite{Geralietal2010}. Each bank consists of two monopolistically competitive retail branches (a loan and a deposit branch) and one perfectly competitive wholesale branch that manages the consolidated balance sheet of the respective bank.
		
		\item Shocks: There are nine shocks in the model: a technology, a monetary policy, a deposit, a loan markup (firms), a loan markup (households), households' LTV, an equity price, and a price markup shock.
		
		\item Estimation: The model is estimated with Bayesian techniques using US data from 1982Q1 to 2015Q1 on nine variables: output, inflation, equity price, household loans, entrepreneur loans, deposits, the Fed funds rate, the mortgage rate, and the Baa corporate rate.
		
	\end{itemize}
	
	
	
	\subsection{US\_IAC05: \cite{Iacoviello2005}}
	\label{USIAC05}
	\cite{Iacoviello2005} develops a New Keynesian model with nominal and financial frictions, where debt contracts are written in nominal terms and some agents face collateral constraints tied to housing values. This gives rise to an accelerator effect for demand shocks and a decelerator effect for supply shocks. The model can match the response of the aggregate demand to housing price shocks and the hump-shaped dynamics of output to inflation surprises, observed from U.S. data.
	
	\begin{itemize}
		
		\item Aggregate Demand: There are two types of households, the ``patient'' and the ``impatient'' ones. They both derive utility from consumption, holdings of housing, real money balances and leisure. However they discount the future differently, with the impatient household discounting the future more heavily. This specification induces the impatient household to face borrowing constraints, consistent with standard lending criteria used in the mortgage market where the borrowing is limited to a fraction of the housing value. For both types of households, the holding of housing is subject to housing adjustment costs.
		
		\item Aggregate Supply: Entrepreneurs produce a homogeneous intermediate good using a Cobb-Douglas technology with labor from both types of households, capital and real estate as inputs. Housing and variable capital are subject to adjustment costs. Following \cite{KiyotakiMoore1999}, a limit on the obligation of the entrepreneurs is assumed. Entrepreneurs discount the future more heavily than the patient households. Both assumptions assure that the borrowing constraint is binding for entrepreneurs. In addition there are retailers who buy the intermediate goods from the entrepreneur, differentiate them at no cost and sell them at a price that can be re-optimized every period only with a certain probability. The optimization problem of the retailers yields a forward-looking Phillips curve.
		
		\item Shocks: A housing preference shock, an inflation shock, a technology shock and a monetary policy shock.
		
		\item Calibration/Estimation: A mixture of calibrated and estimated parameters. Estimation of parameters is done by minimizing a measure of the distance between the VAR impulse responses and model responses, using quarterly U.S. data for the period 1974:Q1--2003:Q2.
		
		%\item Replication: Check the model in the Modelbase!
		
		
	\end{itemize}
	
	\subsection{US\_IN10: \cite{IacovielloNeri2010}}
	\label{USIN10}
	
	The US\_IN10 model is based on various dynamic equilibrium models with neoclassical core and real and nominal rigidities (e.g. \cite{SmetsWouters2007} model). The main goal of this model is to explain the development of the price and the quantity side of the housing market and to examine the spillovers from the housing market to the rest of the economy. The model features a multi-sector structure with housing and non-housing goods, financial frictions in the household sector (introduced through a collateral constraint imposed on a fraction of households), and rich set of shocks which take the model to the data. \\
	
	\begin{itemize}
		
		\item Aggregate demand: There are two types of households according to their discount factors: patient (lenders) and impatient (borrowers). A representative household within each group obtain utility from consumption and housing and disutility from supplying labor in an additively-separable way. Habit formation and balanced growth in consumption are considered. Imperfect labor mobility is introduced across sectors. Patient households accumulate housing and capital, make loans to impatient households, rent capital and land to firms, and choose the capital utilization rate. Impatient households work, consume, accumulate housing and borrow against the value of their housing. Impatient households accumulate housing and borrow the maximum possible amount against its collateral value in equilibrium.
		
		\item Aggregate Supply: Wholesale firms consists of two production units. Housing sector produces new houses (using capital, labour, land and intermediate goods). Non-housing sector produces consumption goods, investment goods and intermediate goods (using capital and labour). It is allowed for price rigidities in the consumption sector and for wage rigidities in both the consumption and housing sectors, but there are no price rigidities in the housing market.
		
		\item Shocks: An intertemporal preference shock, a labor supply shock, a housing preference shock, a cost-push shock, a monetary policy shock, a shock on the central bank's inflation target, sectoral productivity shocks (housing, consumption and non-residential sector)
		
		\item Calibration/Estimation: The model is estimated with Bayesian methods using ten US observables over the period 1965:Q1 to 2006:QIV.
	\end{itemize}
	
	
	\subsection{US\_IR11: \cite{Ireland2011}}
	\label{USIR11}
	\cite{Ireland2011} estimates a New Keynesian model for the US economy in order to compare the Great Recession of 2007-09 with its two immediate predecessors, the milder recessions of 1990-91 and 2001.
	
	\begin{itemize}
		\item Aggregate Demand:  The utility function of the representative household is additively separable in consumption, real money balances and hours worked, and features habit formation in consumption. The household enters each period with money and bonds. At the beginning of each period, it receives a lump-sum nominal transfer from the central bank. Moreover, the household decides about the purchase of new bonds, the supply of labor and the consumption of finished goods. At the end of each period, the household receives nominal dividend payments resulting from the ownership of intermediate-goods-producing firms.
		
		\item Aggregate Supply: During each period, the representative intermediate-goods-producing firm hires labor to manufacture intermediate goods according to a constant-return-to-scale technology.  The representative intermediate-goods-producing firm has monopolistic power, acting as a price-setter. However, price setting is subject to Rotemberg quadratic adjustment costs. The intermediate goods are then used by the finished-goods-producing firms to manufacture final goods under perfect competition.
		
		\item Shocks: An AR(1) preference shock, a cost-push shock in form of a shock to the price mark up, a technology shock that follows a random walk with drift and a monetary policy shock.
		
		\item Calibration/Estimation: The model is estimated via maximum likelihood using U.S. quarterly data on output growth, the inflation rate and the short-term nominal interest rate over the period 1930:Q1--2009:Q4.
		
		%\item Replication: We replicate the impulse responses to a preference, a normalized cost-push, a technology and a monetary policy shock as depicted in Figure 1 on page 44 in \cite{Ireland2011}.
		
	\end{itemize}
	
	\subsection{US\_IR15: \cite{ireland2015monetary}}
	\label{USIR15}
	\cite{ireland2015monetary} considers a model of the term structure of interest rates, where bond yields are driven by observable and unobservable macroeconomic factors. Restrictions on the model parameters help identify the effects of monetary policy and other structural disturbances on output, inflation, and interest rates and decompose movements in long-term rates into terms attributable to changing expected future short rates versus risk premia. The model is estimated on US data and highlights a broad range of channels through which monetary policy, risk premia and the economy interact.
	\begin{itemize}
		\item Model: Bond yields in this pricing model get driven by five state variables: two unobservable (a risk variable which governs all variation in bond risk premia and the central bank's inflation target) and three observable (short-term nominal interest rate, the inflation rate, and the output gap). The inflation and output gaps are allowed to depend on their own lagged values and lagged values of the model's other variables, as they would in a more conventional macroeconomic vector autoregression. The structural shocks are identified by restrictions on the parameters implied by a) no-arbitrage considerations, b) the assumption that the risk premium is solely driven by the unobservable risk variable, and c) the assumption of a Taylor-rule type interest rate policy. The estimates of the correlation and volatility parameters, together with an analysis of the impulse responses and forecast error variance decompositions implied by those estimates, are used to assess the extent to which movements in bond risk premia are driven by monetary policy and macroeconomic shocks or whether they reflect, instead, disturbances that appear purely financial in origin.
		
		\item Shocks:  There are 5 structural shocks in the model: on the short-term nominal interest rate, on inflation, output gap, the inflation target and the risk premium. Furthermore there are measurement errors for the one, two and four-year bond yields. 
		\item Calibration/Estimation: The model is estimated with US quarterly data from 1959:1 to 2007:4 for the short-term nominal interest rate, the inflation rate, the output gap, and yields on discount bonds with one through five years to maturity.
		
	\end{itemize}
	
	\subsection{US\_JPT11: \cite{Justinianoetal2011}}
	\label{USJPT11}  
	\cite{Justinianoetal2011} include two investment shocks into an otherwise standard new-Keynesian dynamic stochastic general equilibrium model and estimate it through Bayesian techniques using U.S. data. The first new shock, an investment-specific technology shock, affects the transformation of consumption into investment goods and is identified with the relative price of investment. The second shock affects the production of installed capital from investment goods or, more broadly, the transformation of savings into the future capital input.
	\begin{itemize}
		\item Aggregate Demand: Households maximize their lifetime utility, where the utility function is separable in consumption and leisure and includes habit formation, subject to an intertemporal budget constraint. Households own firms and the capital stock, monopolistically offer specialized labor to employment agencies, set the utilization rate when renting effective capital to intermediate producers, save in government bonds, and receive dividend payments and lump-sum transfers. Additionally, it is assumed that they have access to state-contingent securities traded between each other. 
		\item Aggregate Supply: Perfectly competitive final good producers purchase intermediate goods from monopolistically-competitive producers and combine them through CES technology into a homogeneous final good which can be used for consumption or investment. Intermediate good producers, which are subject to \cite{Calvo1983} price-stickiness with partial indexation to inflation, rent effective capital and purchase homogeneous labor units to produce by means of a Cobb-Douglas production function with a fixed production cost. The homogeneous labor units are produced by perfectly competitive employment agencies which purchase specialized labor from the households. Perfectly competitive investment good producers linearly transform final goods into investment goods, while capital good producers transform investment goods into new capital subject to adjustment costs.
		\item Shocks: An intertemporal preference shock affects households, intermediate firms' neutral technology factor and investment good producers' investment-specific productivity factor are unit root processes. Wages and intermediate goods' prices are subject to mark-up shocks. Capital producers' marginal efficiency of investment is subject to an exogenous disturbance, and both government spending and the monetary policy rate are subject to shocks.
		\item Calibration/Estimation: The model is estimated for the U.S. with Bayesian techniques for the period 1954:3 until 2009:1 using eight macroeconomic variables: employment, inflation, the nominal interest rate, real GDP, consumption, investment, real wage and relative price of investment. For the growth rates of investment-specific technology and the composite trend the authors divide the sample into two subsamples: 1954--1982 and 1983--2008.
	\end{itemize}
	
	
	
	\subsection{US\_KK14: \texorpdfstring{\cite{kliem2014kriwoluzky}}{Kliem and Kriwoluzky (2014)}}
	\label{USKK14}
	\cite{kliem2014kriwoluzky} set up a New Keynesian model that entails a fiscal sector with several instruments, estimate it on US data and analyze the empirical plausibility and welfare properties of feedback rules for labor and capital income tax rates.
	
	\begin{itemize}
		
		\item Aggregate Demand: Households consume, supply labor and invest in capital and riskless bonds. Their utility function is separable in consumption and leisure and features habit formation as well as an exogenous consumption demand shifter. Next to the labor income and capital income (which are both taxed), it derives income from firm's dividends (which are also taxed) and fiscal transfers. They set their wages in monopolistic competition as in \cite{ErcegHendersonLevin2000}. Those households, which cannot update their wages in a given period index it to the steady state inflation rate.
		
		\item Aggregate Supply: The production sector is comprised of final good producers, and intermediate good producers. Intermediate good producers produce their goods via a Cobb-Douglas production function employing labor and capital. The production function features variable capital utilization and fixed costs. They are monopolistically competitive and face price stickiness as in the Calvo framework. Final good producers act in perfect competition. They buy intermediate goods and bundle them to a final good. Capital accumulation (by the households) is subject to investment adjustment costs.
		
		\item Monetary and Fiscal authorities: Monetary policy is conducted using an interest rate rule that exhibits interest rate smoothing and a response to inflation and the output gap. The government raises taxes on labor and capital income which are modelled as feedback rules reacting to the levels of government debt and output. Government consumption and transfers evolve according to exogenous AR(1) processes. 
		\item Shocks: A preference shock, a technology shock, an investment-specific efficiency shock, a price markup shock, a wage markup shock, a monetary policy shock, a transfer shock, a government spending shock, a shock to the resource constraint and shocks to the labor and capital income tax rates.
		\item Estimation: The model is estimated on US data for the period of 1983:1 -2008:3 using Bayesian methods. As 12 observables are used in the estimation, additional to the 11 structural shocks, the authors add measurement error to the observation equation for tax revenues. 
		
	\end{itemize}
	
	\subsection{US\_KS15: \texorpdfstring{\cite{kriwoluzky2015stoltenberg}}{Kriwoluzky and Stoltenberg (2014)}}
	\label{USKS15}
	\cite{kriwoluzky2015stoltenberg} incorporate an explicit transaction role for money in a standard cashless new Keynesian model (\cite{Woodford2003}) and compare the role of money in the pre-Volcker period (before 1979) with the period from 1982 on. They estimate that before 1979, money played an important role in facilitating transactions while after 1982, the importance of money declined sharply. They argue that this shift can possibly explain the switch in US interest rate policy from a passive to an active setting.
	\begin{itemize}
		
		\item Aggregate Demand: In the model there is a continuum of infinitely lived households that maximize expected lifetime utility subject to a budget constraint that incorporates transaction costs for purchasing consumption goods. The instantaneous utility function is increasing in consumption and decreasing in labour that the households supply to firms.
		
		\item Aggregate Supply: Using the labour supply from households as the sole input, monopolistically competitive firms produce differentiated goods that are aggregated to the final consumption good. Price setting by the firms follows \cite{Calvo1983}, leading to nominal rigidities in the model.
		
		\item Shocks: There is a technology, a government spending, a wage mark-up, a taste, a monetary policy shock and a shock to transaction costs.
		
		\item Estimation: The model is estimated using Bayesian techniques on US data using real output, real consumption, annual inflation, the federal funds rate, real money balances, and real wages as observable variables. The quarterly data ranges from 1964Q1 to 2008Q2 and is split into two parts: from 1964Q1 to 1978Q4 and from 1983Q1 to 2008Q2, excluding the disinflation years.
		
	\end{itemize}
	
	\subsection{US\_LTW17, US\_LTW17gz, US\_LTW17nu, US\_LTW17rot: \texorpdfstring{\cite{leeper2017traum}}{Leeper et al. (2017)} }
	\label{USLTW17}
	
	\cite{leeper2017traum} implement a new Keynesian model, based on \cite{SmetsWouters2007,SmetsWouters2003}, yet add distorting tax rates on capital and labor income and consumption. The additional features of the model are that utility also depends on government consumption and saver households have also access to a portfolio of long-term government zero-bonds with maturity decaying at a constant rate (not only to short-term bonds). The model is used to assess the fiscal multiplier in the US.
	
	\begin{itemize}
		
		\item Aggregate Demand: The model economy is populated by a continuum of infinitely lived households of which a fraction is non-saver. Non-saver households do not have access to any savings technology, thus they consume their entire disposable income every period. The firms and the capital stock are owned entirely by saver households. The utility function is separable in consumption and leisure and assumes external habits that depend on aggregate consumption in the last period. In addition, household consumption also depends on government consumption in an additive manner.
		
		\item Aggregate Supply: Production is carried out in two stages, by a perfectly competitive final goods producer and a continuum of monopolistically competitive intermediate goods producers using capital and labour as input factors. Households provide uniquely differentiated labor in monopolistic competition. Saver households set wages optimally while non-savers follow a rule-of thumb to set their wage rates to be the average wage rates chosen by savers. Wages and prices are allowed to adjust only gradually by assuming Calvo pricing with partial adjustment of the contracts to past inflation.
		
		\item Monetary and Fiscal authorities: Monetary authorities follow a Taylor-type rule with lagged policy rates. Fiscal authorities levy distortionary taxes on income from capital, labor and consumption taxes and sell the nominal bond portfolio to finance its interest payments, government consumption and lump-sum transfers to households. Fiscal rules include a response of fiscal instruments to the market value of the debt-to-GDP ratio and an autoregressive term to allow for serial correlation. The model was restricted such that only public consumption and transfers potentially respond to debt. Tax distortions enter only the steady state.
		
		\item Shocks: Government consumption shock, transfer shock, total factor productivity shock, preference shock, investment adjustment costs shock, monetary policy shock, wage markup shock, price markup shock.
		
		\item Estimation: The baseline model, implemented into the MMB, is estimated for the U.S. by means of Bayesian techniques for the period 1955:1-2007:4 using eight key macroeconomic variables: log differences of aggregate consumption, investment, real wages, real government consumption, the real market-value of government debt, and the GDP deflator; log hours worked; the federal funds rate. Data are neither detrended nor demeaned. Drawing on the information from the prior predictive analysis, the authors eliminated rule-of-thumb agents. They also did not include tax revenues or tax rates in the observables because quarterly measures of marginal tax rates are problematic.
		
		\item Replication: The original NK-model was separated from the Matlab based code provided by the authors and translated into Dynare. The baseline scenario (solid line) among the impulse response functions in Figure 5 was replicated and compared also with the implemented version of the model. The IRFs in the model base appear to match those from the original Matlab code if one eliminates the autoregressive coefficient in the error term of the monetary policy rule in the original code. In addition to the baseline setting two counterfactual models were added, three from Figure 5 ( 1. lower habit formation coefficient with no government spending in utility (US$\_$LTW17nu); 2. Inclusion of rule of thumb consumers (US$\_$LTW17rot); 3. Instead of government consumption response to debt, there are changes in transfers (US$\_$LTW17gz)).
		
	\end{itemize}
	
	
	
	\subsection{US\_LWY13: \cite{leeper2013fiscal}}
	\label{USLWY13}
	\cite{leeper2013fiscal} implement a new Keynesian model, similar to those in \cite{SmetsWouters2007,SmetsWouters2003}, yet add distorting tax rates on capital and labor income. The model is used to assess the effect fiscal foresight entails to a naive econometrician who estimates impulse-response functions conditioning solely on the variables observed and disregards fiscal foresight. 
	\begin{itemize}
		\item Aggregate demand: The model economy is populated by a continuum of infinitely lived households of which a fraction is non-Ricardian. Non-Ricardian households do not have access to any savings technology, thus they consume their entire disposable income every period. The firms and the capital stock are owned entirely by Ricardian households. The utility function is separable in consumption and leisure and assumes external habits that depend on aggregate consumption in the last period. Households provide uniquely differentiated labor in monopolistic competition. Ricardian households have also access to state-contingent claims to eliminate the income differentials due to differentiated labor. 
		
		\item Aggregate supply: Production is carried out in two stages, by a perfectly competitive final good producer and a continuum of monopolistically competitive intermediate goods producers using capital and labour as input asfactors. Wages and prices are allowed to adjust only gradually by assuming  Calvo pricing with partial adjustment of the contracts to past inflation. 
		
		\item Government: Monetary authorities follow a Taylor-type rule. Fiscal authorities levy distortionary taxes on income from capital and labor and pay lump-sum transfers to households.
		
		\item Shocks: Total factor productivity shock, preference shock, investment adjustment costs shock, monetary policy shock, wage markup shock, price markup shock, government spending shock, capital tax rate shock, labour tax rate shock and government transfers shock .
		
		\item Calibration/Estimation: The model is estimated for the U.S. by means of Bayesian techniques for the period 1984:1$-$2007:4 using ten key macroeconomic variables: consumption, investment, labor, wage rate, the nominal interest rate, inflation, capital tax revenues, labor tax revenues, the sum of real government consumption and investment, and government transfers. Government data include all federal, state, and local levels.
		
		\item Replication: Unfortunately there were no impulse response functions to be replicated. Therefore, the original NK-model was separated from the code and translated to Dynare. The impulse response functions of output and the interest rate were replicated and compared also with the implemented version of the model. The IRFs in the model base seemed to match those from the original Matlab code which were calculated manually.
		
	\end{itemize}
	
	
	\subsection{US\_MI07: \texorpdfstring{\cite{Milani2007}}{Milani (2007)}}
	\label{USMI07}
	
	\cite{Milani2007} presents an estimated model with learning and provides evidence that learning can improve the fit of popular monetary DSGE models and endogenously generate realistic levels of persistence. The rational expectations version of the model is based on the those applied by \cite{boivin2006has}, \cite{giannoni2003optimal}, and also described in \cite{Woodford2003}. The model incorporates some of the structural sources of persistence, such as habit formation in consumption and inflation indexation. The main finding of the paper is that the empirical results show that when learning replaces rational expectations, the estimated degrees of habits and indexation drop near zero. This finding suggests that persistence arises in the model economy mainly from expectations and learning.
	
	
	\begin{itemize}
		
		\item Aggregate Demand: The representative household maximizes lifetime utility subject to an intertemporal budget constraint. Utility from consumption and disutility from labor is separable. Preferences for consumption are subject to habit persistence. The representative household offers a continuum of different types of labor to the firms.
		
		\item Aggregate Supply: There exists a continuum of monopolistically competitive firms. Price stickiness is embedded into the model via the \cite{Calvo1983} framework. Each good is produced using a decreasing return to scale technology and capital is assumed to be fixed, leaving labor as the only variable factor of production. The natural real rate of interest is modeled as an exogenous AR(1)-process.
		
		\item Shocks: Natural real interest rate shock, cost-push shock, monetary policy shock
		
		\item Estimation: The model is estimated using likelihood-based Bayesian methods to fit the series for output gap, inflation, and the nominal interest rate as used in a number of papers, surveyed in \cite{an2007bayesian}. Yet, the paper provides an example of estimation of a simple DSGE model with non-fully rational expectations and learning. The data are quarterly for the period 1960:I to 2004:II.
		
	\end{itemize}
	
	
	%
	%\subsection{EA\_CW05: \cite{CoenenWieland2005}}
	%\label{EACW05ta}
	%\label{EACW05fm}
	%\cite{CoenenWieland2005} develop a small-scale macroeconomic model for various staggered pricing schemes. We use a version with the nominal contract specification of \cite{Taylor1980}, labeled EA\_CW05ta, and a version with the relative real wage contract specification of \cite{FuhrerMoore1995}, labeled EA\_CW05fm.
	%
	%\begin{itemize}
	%%\item Purpose of the Model: Develop and estimate a small structural model of the euro area to be used for evaluating alternative monetary policy strategies.
	%\item Aggregate Demand: The aggregate demand equation is backward looking: two lags of aggregate demand (should account for habit persistence in consumption, adjustment costs and accelerator effects in investment) and one lag of the long-term interest rate (allows for a transmission lag of monetary policy). The long-term nominal interest rate is an average of expected future nominal short-term rates. The long-term real rate is determined by the Fisher equation.
	%\item Aggregate Supply: As in US\_FM95 and US\_OW98.
	%%\item The Foreign Sector: no foreign sector
	%%\item Microeconomic foundation: no
	%\item Shocks: A demand shock, a contract wage shock and the common monetary policy shock.
	%%\item Variable dimension: Original variables are expressed in percent/100. The common Modelbase variables are expressed in percent.
	%\item Calibration/Estimation: The model has been estimated on data from the ECB Area Wide Model data set from 1974:1--1998:4. The contract wage specifications have been estimated by a limited information indirect inference technique while the IS equation has been estimated by means of the GMM.
	%%\item Replication: We replicated the impulse response functions of annual inflation and the output gap to a 100bps temporary unanticipated rise in the nominal short term rate in the upper panel of Figure 7 of \cite{KuesterWieland2005} for both versions of the model.
	%%\item Impulse responses: Figure \ref{img:EA_CW05ta} (EA\_CW05ta), figure \ref{img:EA_CW05fm} (EA\_CW05fm).
	%
	%%\item Impulse responses: figure \ref{img:CW05ta} shows impulse responses for the model with Taylor contracts. Figure \ref{img:CW05fm} shows impulse responses for the model with Fuhrer-Moore contracts. The first row shows impulse responses to a one unit monetary policy shock. The second row shows impulse responses to a one unit demand shock that is added to the IS equation. One can see the higher inflation persistence implied by Fuhrer-Moore contracts.
	%\end{itemize}
	%
	%\subsection{EA\_AWM05: Area Wide model linearized by Dieppe, Kuester and McAdam (2005)}
	%\label{EAAWM05}
	%The model is described in \cite{FaganHenryMestre2005}. It was one of the first models to treat the Euro area as a single economy. In the Modelbase we use the linearized version from Dieppe, Kuester and McAdam (2005) that is also used in \cite{KuesterWieland2005}. The EA\_AWM05 is an open economy model of the Euro area. Expectation formation is largely backward-looking. Activity is demand-determined in the short-run but supply-determined in the long-run with employment having converged to a level consistent with the exogenously given level of equilibrium unemployment. Stock-flow adjustments are accounted for, e.g., the inclusion of a wealth term in consumption.
	%
	%\begin{itemize}
	%%\item Purpose of the Model: Building a model for the Euro area for simulation and forecasting purposes.
	%\item Aggregate Demand: Demand is disaggregated into private consumption, government consumption, investment, variation of inventories, exports, and imports. The term structure (12-year bond) is forward-looking. Private consumption is specified as a function of households' real disposable income and wealth, where the latter consists of net foreign assets, public debt and the capital stock. The change in the log of the investment/output ratio depends on the real interest rate, the real GDP/capital stock ratio and the lagged investment/output ratio. The authors stress that this investment equation represents the key channel through which interest rates affect aggregate demand. Government consumption is treated as exogenous.
	%\item Aggregate Supply: Output follows a whole economy production function.  Short-run employment dynamics are driven by output growth and real wages. The deflator for real GDP at factor costs, which according to \cite{FaganHenryMestre2005} is the key price index of the model, is a function of unit labor costs, import prices, the output gap and inflation expectations. The growth rate of wages depends on consumer price inflation, productivity and the unemployment gap, defined as the deviation of the current unemployment rate from the NAIRU.
	%\item Foreign sector: Besides extra-area flows, exports and imports also include intra-area flows. %Exports depend on world demand.
	%World GDP and world GDP deflator are treated as exogenous variables. The exchange rate is a forward-looking variable determined by uncovered interest rate parity.
	%%\item Microeconomic foundation: no
	%\item Shocks: Employment shock, factor cost-push shock, private consumption cost-push shock, gross investment cost-push shock, gross investment shock, exports cost-push shock, imports cost-push shock, private consumption shock, term structure shock, common fiscal policy shock and common monetary policy shock.
	%%\item Variable dimension: Original variables are expressed in percent/100. The common Modelbase variables are expressed in percent.
	%\item Calibration/Estimation: Estimation on Euro area data equation by equation from 1970:1--1997:4, whereas the estimation period of some equations starts later, but not later than 1980:1.
	%%\item Replication: We replicated the impulse response functions of annual inflation and the output gap to a 100bps temporary unanticipated rise in the nominal short term rate in the upper panel of Figure 7 of \cite{KuesterWieland2005}.
	%%\item Replication: Using the policy rule by \cite{GerdesmeierRoffia2004} we can replicate the impulse responses presented in \cite{KuesterWieland2005}. Note, in the version in this modelbase the original fiscal policy rule is substituted by a fiscal policy rule that does not lead to permanent shocks. For the replication exercise one must use the original rule. Otherwise one gets slight differences compared to \cite{KuesterWieland2005}. Furthermore, one has to use output instead of the output gap in the policy rule.
	%
	%%\item Impulse responses: Figure \ref{img:EA_AWM05}.
	%
	%%\item Impulse responses: The first row of figure \ref{img:AWM} shows impulse responses to a one unit monetary policy shock. The high persistence implied by the backward-looking expectations of the model can be seen in the graph. The second row shows impulse responses to a government spending shock that equals one percent of GDP. The AWM is estimated assuming that government spending is non-stationary. Hence the original rule adds a shock to the first difference of government spending. To make our experiment of a temporary shock comparable across models we assume an AR(1) process for government spending and scale the shock with the inverse of the share of government spending in GDP, which is equal to 0.14. Thus, a one unit shock equals one percent of GDP. The AR(1) coefficient is 0.97.
	%\end{itemize}
	%
	%
	%
	%\subsection{EA\_SW03: \cite{SmetsWouters2003}}
	%\label{EASW03}
	%The EA\_SW03 model of \cite{SmetsWouters2003} is a medium-scale closed economy DSGE model with various frictions and estimated for the Euro area with Bayesian techniques.
	%
	%\begin{itemize}
	%\item Aggregate Demand: Households maximize their lifetime utility, where the utility function is separable in consumption, leisure and real money balances, subject to an intertemporal budget constraint. \cite{SmetsWouters2003} include external habit formation to make the consumption response in the model more persistent. Households own firms, rent capital services to firms and decide how much capital to accumulate given certain capital adjustment costs. They additionally hold their financial wealth in the form of cash balances and one-period, state-contingent bonds. Exogenous spending is introduced by a first-order autoregressive process with an iid-normal error term.
	%\item Aggregate Supply: The final goods, which are produced under perfect competition, are used for consumption and investment by the households and by the government. The final goods producer maximizes profits subject to a Dixit-Stiglitz aggregator of intermediate goods, which introduces monopolistic competition in the market for intermediate goods and features a constant elasticity of substitution between individual, intermediate goods. A continuum of intermediate firms produce differentiated goods using a production function with Cobb-Douglas technology and fixed costs and sell these goods to the final-goods sector. They decide on labor and capital inputs, and set prices according to the Calvo model. Labor is differentiated over households using the Dixit-Stiglitz aggregator, too, so that there is some monopoly power over wages, which results in an explicit wage equation. Sticky wages a la Calvo are additionally assumed. The Calvo model in both wage and price setting is augmented by the assumption that prices that can not be freely set, are partially indexed to past inflation rates.
	%\item Shocks: Ten orthogonal structural shocks are introduced in the model. Three preference shocks in the utility function: a general shock to preferences, a shock to labor supply and a money demand shock. Two technology shocks: an AR(1) process with an iid shock to the investment cost function and a productivity shock to the production function. Three cost push-shocks: shocks to the wage and price mark-up, which are iid around a constant and a shock to the required rate of return on equity investment. And finally two monetary policy shocks: a persistent shock to the inflation objective and a temporary common monetary policy shock. In addition, the common fiscal policy shock is added in the form of a government spending shock. Since government spending is expressed in output units, we set the coefficient which scales the shock to unity to achieve a shock size of one percent of GDP.
	%%\item Variable dimension: The model is log-linearized around the steady state. Variables are expressed in terms of percentage deviations from steady state.
	%\item Calibration/Estimation: The model is estimated using Bayesian techniques on quarterly Euro area data.
	%The data set used is comprised of seven key macroeconomic variables consisting of real GDP, real consumption, real investment, the GDP deflator, real wages, employment and the nominal interest rate over the period 1970:1--1999:4.
	%%\item Replication: We replicated the impulse response functions of annual inflation and the output gap to a 100bps temporary unanticipated rise in the nominal short term rate in the upper panel of Figure 7 of \cite{KuesterWieland2005}.
	%
	%%\item Impulse responses: Figure \ref{img:EA_SW03}.
	%
	%%\item Impulse Responses: The first row of figure \ref{img:SW03} shows impulse responses to a one unit monetary policy shock. The second row shows impulse responses to an increase in government consumption of one percent of GDP. The AR(1) coefficient of the shock process is 0.95. In SW03 government spending is expressed in output units. We therefore set the coefficient which scales the shock to unity to achieve a shock size of one percent of GDP.
	%\end{itemize}
	%
	%\subsection{EA\_SR07: Euro Area Model of Sveriges Riksbank, \cite{AdolfsonLaseenLindeVillani2007}}
	%\label{EASR07}
	%\cite{AdolfsonLaseenLindeVillani2007} develop an open economy DSGE model and estimate it for the Euro area using Bayesian estimation techniques. They analyse the importance of several rigidities and shocks to match the dynamics of an open economy.
	%\begin{itemize}
	%\item Aggregate Demand: Households maximize lifetime utility subject to a standard budget constraint. Preferences are separable in consumption, labor and real cash holdings. Persistent preference shocks to consumption and labor supply are added to the representative utility function. Internal habit formation is imposed with respect to consumption. Aggregate consumption is specified as a CES function, being composed of domestically produced as well as imported consumption goods. Households rent capital to firms. Capital services can be increased via investment and via an increase in the capital utilization rate, where both options are involved with costs. Total investment in the domestic economy is represented by a CES aggregate consisting of domestic and imported investment goods. Households are assumed to be able to save through acquiring domestic bonds and foreign bonds in addition to holding cash and accumulating physical capital. A premium on foreign bond holdings assures the existence of a well-defined steady state. Households monopolistically supply a differentiated labor service. Wage stickiness is introduced in the form of the Calvo model augmented by partial indexation. \\ Government consumption of the final domestic good is financed via taxes on capital income, labor income, consumption and payroll. Any surplus or deficit is assumed to be carried over as a lump-sum transfer to households.
	%\item Aggregate Supply: The final good is produced via a CES aggregator using a continuum of differentiated intermediate goods as inputs. The production of intermediate goods requires homogeneous labor and capital services as inputs and is affected by a unit-root technology shock representing world productivity as well as a domestic technology shock. Fixed costs are imposed such that profits are zero in steady state. Due to working capital, (a fraction of) the wage bill has to be financed in advance of the production process. Price stickiness of intermediate goods is modeled as in the \cite{Calvo1983} model. In addition, partial indexation to the contemporaneous inflation target of the central bank and the previous periods inflation rate is included for those firms that do not receive a Calvo signal in a given period. This results in a hybrid new Keynesian Phillips curve.
	%\item Foreign sector: Importing firms are assumed to buy a homogeneous good in the world market and differentiate it to sell it in the domestic market. Similarly, exporting firms buy the homogeneous final consumption good produced in the domestic economy and differentiate it to sell it abroad. Specifically, the differentiated investment and consumption import goods are aggregated in a second step via a CES function, respectively. The same applies to the export goods. Calvo pricing is also assumed for the import and export sector, allowing for incomplete exchange rate pass-through in the short run. The foreign economy is described by an identified VAR model for foreign prices, foreign output and the foreign interest rate.
	%\item Shocks: Unit root technology shock, stationary technology shock, investment specific technology shock, asymmetric technology shock, consumption preference shock, labor supply shock, risk premium shock, domestic mark-up shock, imported consumption mark-up shock, imported investment mark-up shock, export mark-up shock, inflation target shock, the common monetary policy shock, shocks to the four different tax rates and a government spending shock which represents the common fiscal policy shock and which we have adjusted so that we achieve a shock size of one percent of GDP.
	%%\item Variable dimension: The model is log-linearized around the steady state. Variables are expressed as percentage deviations from steady state.
	%\item Calibration/Estimation: The model is estimated using Bayesian estimation techniques for the Euro area using quarterly data from 1970:1--2002:4 in order to match the dynamics of 15 selected variables. According to the authors, they calibrated those parameters that should be weakly identified by the 15 variables used for estimation.
	%%\item Replication: We replicated the impulse response functions for annualized quarterly inflation, output, employment and the annualized interest rate to a one standard deviation monetary policy shock in Figure 3 of \cite{AdolfsonLaseenLindeVillani2007}.
	%%\item Impulse responses: Figure \ref{img:EA_SR07}.
	%
	%\end{itemize}
	%\subsection{EA\_QUEST3: \cite{RattoRoegerVeld2009}}
	%\label{EAQUEST3}
	%\cite{RattoRoegerVeld2009} develop and estimate an open economy DSGE model for the euro area with emphasis on monetary and fiscal rules, in order to explore their stabilization properties. The role of fiscal policy is explored in an environment with rules for government consumption, investment and transfers and with financial frictions in the form of liquidity-constrained households.
	%\begin{itemize}
	%
	%\item Aggregate Demand: There are two types of households: liquidity- and non-liquidity-constrained households. They posses the same utility function, non-separable in consumption and leisure with habit persistence in both consumption and leisure. Liquidity-constrained households do not optimize, they just consume their labor income. On the other side, non-liquidity-constrained households have access to domestic and foreign currency denominated assets, accumulate capital subject to investment adjustment costs and rent it to firms, earn profits from owning the firms and pay taxes. Income from foreign financial assets is subject to an external financial intermediation risk premium while real asset holdings are subject to an equity risk premium. Both types of households supply differentiated labor to a trade union which sets the wages by maximizing their joint utility (weighted by the share of each type). The wage setting process is subject to a wage mark-up and to slow adjustments in the real consumption wage. The wage mark-up arises because of wage adjustment costs and the fact that a part of workers index the growth rate of wages to past inflation.
	%
	%\item Aggregate Supply: The final goods, which are produced from monopolistically competitive firms, are used for household
	%consumption, investment, government consumption and export. These goods are produced with a Cobb-Douglas production function with capital and production workers (labor adjusted for overhead labor) as inputs. These firms face technological and regulatory constraints, restricting their price setting, employment and capacity utilization decisions. The final goods producer maximizes profits subject to these specific adjustment costs (all having convex functional forms) and demand conditions. Investment good producers combine domestic and foreign final goods using a CES aggregator to produce investment goods which are sold to non-liquidity-constrained households in a perfectly competitive market.
	%
	%\item The Foreign Sector: Demand behavior is considered the same for the home country and the rest of the world, therefore export demand and import demand are symmetric. Both equations are characterized by a lag structure in relative prices which captures delivery lags. Export firms buy domestic goods, transform them using a linear technology and sell them in the foreign market, charging a mark-up over the domestic prices. The same situation is faced by importer firms. Mark-up fluctuations arise because of price adjustment costs in both sectors. Mark-up equations are given as a function of past and future inflation and are also subject to random shocks.
	%
	%\item Shocks: A wage mark up shock, a price mark-up shock, a monetary policy shock, a fiscal policy shock, world demand shock, a risk premium shock, a technology shock, an investment shock, a consumption shock, a trade shock, a labor demand shock, a foreign monetary policy shock.
	%
	%\item Calibration/Estimation: Estimated with Bayesian methods, using quarterly data for the euro area for the period 1981:1--2006:1.
	%
	%%\item Replication: Check the model in the Modelbase!.
	%
	%
	%\end{itemize}
	%
	%
	%\subsection{EA\_GE10: \cite{Gelain2010}}
	%\label{EAGE10}
	%The model of \cite{Gelain2010} incorporates financial frictions \`{a} la \cite{BernankeGertlerGilchrist1999} into a New Keynesian DSGE model which closely follows the structure of the model developed in \cite{SmetsWouters2003}. The structural model allows for a dynamic analysis of the external finance premium. The paper shows that the estimated premium is not necessarily countercyclical as suggested by former studies on the Euro Area external finance premium. In the presence of certain shocks the premium responds procyclically.
	%
	%\begin{itemize}
	%\item Aggregate Demand: A representative household maximizes its intertemporal utility function choosing the level of consumption, hours worked and the amount of bank deposits, subject to a budget constraint. The household's consumption preferences exhibit habit formation.
	%
	%\item Aggregate Supply: Each household is a monopolistic supplier of differentiated labor services requested by the domestic firms. After setting their wages in a Calvo staggered way, households inelastically supply the firms' demand for labor at the ongoing wage rate. An indexation rule is assumed for those households who are not allowed to re-optimize. \\
	%The production sector consists of three types of firms: entrepreneurs, capital producers and retailers. Entrepreneurs hire labor from households and buy capital from capital producers to produce intermediate goods using a Cobb-Douglas production technology. Entrepreneurs have a finite expected lifetime horizon. The capital purchases are financed partly by the entrepreneur's net worth and partly by borrowing from a financial intermediary. The presence of asymmetric information between entrepreneurs and lenders creates a financial friction as in \cite{BernankeGertlerGilchrist1999}. Entrepreneurs can reoptimize their prices only from time to time, as in \cite{Calvo1983}. \\
	%Capital producers buy final goods to produce capital subject to investment adjustment costs. Retailers operate in a perfectly competitive market, they use a Dixit-Stiglitz technology using the entrepreneurs' intermediate goods as inputs.
	%
	%\item Shocks: The model exhibits eight shocks. Two preference shocks, a shock to investment adjustment costs, a technology shock in entrepreneurs' production function, a wage and a price mark up shock, a government spending shock and a monetary policy shock.
	%
	%\item The model is estimated using Bayesian techniques on quarterly Euro Area data for 1980:Q1 to 2008:Q3.
	%The data set used is comprised of seven key macroeconomic variables aggregated for the Euro Area consisting of real GDP, real consumption, real gross investment, hours worked, the nominal short term interest rate, real wages per head and inflation rate.
	%
	%%\item Replication: We replicated the impulse responses to a one standard deviation orthogonalized monetary policy, technology, an investment specific and a labor supply shock as in Figures 5-8 on pages 64-67 in \cite{Gelain2010}.
	%\end{itemize}
	%
	%%\subsection{EA\_QR14: \cite{QR2014}}
	%%\label{EARQ14}
	%%
	%%\cite{QR2014} builds a two-country, two-sector, two-agent general equilibrium model of a single currency area. The two countries, referred to as ”home” and ”foreign” (or alternatively, ”core” and ”periphery”), use the same currency and are subject
	%%to the same monetary policy, which targets union-wide inflation. In each country there are two type
	%%of goods: durable goods, taken to represent housing, which cannot be traded across countries, and
	%%non-durable goods which can be traded. Production is undertaken in both countries by two types of
	%%producers: final good producers operating in a perfectly competitive market and intermediate goods
	%%producers operating under monopolistic competition with price setting `a la Calvo.
	%%
	%%Each country presents borrowers and savers, which are distinguished by their different discount factor and guarantee
	%%that there is a credit market both within and between countries. Finally, there are two types of financial
	%%intermediaries in the model, domestic and international. The former take deposits from savers and
	%%issue bonds that can be traded across countries by the latter. Domestic intermediaries pay the deposit
	%%interest rate on the liabilities they issue. On the asset side, they lend to borrowers at the lending rate.
	%%The model features a financial accelerator mechanism on the household side, such that changes in the
	%%balance sheet of borrowers due to fluctuations in durables' prices affect the spread between lending
	%%and deposit rates.
	%%
	%%The model features 13 exogenous shocks, preference shocks, sectoral technology shocks, financial shocks, non-stationary union-wide technology shock and a monetary policy shock.
	%
	%\subsection{EA\_GNSS10: \cite{Geralietal2010}}
	%\label{EAGNSS10}
	%
	%The model of \cite{Geralietal2010} incorporates an imperfect competitive banking sector in a DSGE model with financial frictions \`{a} la \cite{Iacoviello2005}. The model allows to asses the role of both financial frictions and banking intermediation in shaping business-cycle dynamics.
	%
	%\begin{itemize}
	%\item Aggregate Demand: There are two type of households, patient (savers) and impatient (borrowers) and entrepreneurs. A representative household maximizes its intertemporal utility function choosing the level of consumption, hours worked and housing services, subject to a budget constraint. Impatient households face in addition a borrowing constraint, linked to the expected value of their collateralizable housing stock. The household's consumption preferences exhibit external habit formation.
	%Households supply differentiated labor services through unions. Wage-setting is subject to adjustment costs, \`{a} la Rotemberg with indexation to both past and steady state inflation. Entrepreneurs maximize their utility by choosing consumption, physical capital, loans from banks, the degree of capacity utilization and labor. Entrepreneurs face also a borrowing constraint, linked to the value of their holdings of physical capital.
	%
	%\item Aggregate Supply: The production sector consists of three types of firms: entrepreneurs, capital good producers and retailers. Entrepreneurs hire labor from households and buy capital from capital good producers to produce intermediate goods. Capital producers buy final goods to produce capital subject to investment adjustment costs. Retailers buy intermediate goods from entrepreneurs and differentiate them. Pricing is subject to nominal rigidities.
	%
	%\item Banking Sector: Banks enjoy market power in conducting their intermediation activity. Bank loans should be met by deposits and/or bank capital. Each bank has three parts, two ``retail'' branches (giving out differentiated loans to impatient households and to entrepreneurs and raising differentiated deposits from patient households) and one ``wholesale'' unit (managing the capital position of the group).
	%
	%\item Shocks: The model exhibits a technology shock, price and wage markups shocks, a consumption preferences shock, a housing demand shock, an investment-specific technology shock, a monetary policy shock, shocks to the loan-to-value ratios on loans to firms and households, shocks to the markup on bank interest rates and balance sheet shocks.
	%
	%\item The model is estimated with Bayesian techniques, for the euro area for 1998:Q1-2009:Q1. For estimation twelve observables are used: real consumption, real investment, real house prices, real deposits, real loans to households and firms, overnight rate, interest rates on deposits, loans to firms and households, wage inflation and consumer price inflation.
	%\end{itemize}
	%\subsection{EA\_QR14: \cite{QR2014}}
	%\label{EAQR14}
	%\cite{QR2014} use a two-country, two-sector, two-agent DSGE model of the euro area with nominal and financial frictions to study the interactions of monetary and macroprudential policy. The two countries represent the core and the periphery in the EMU; the two sectors capture non-durable and durable goods, the latter being housing goods. It builds on the model presented in \cite{Rabanal2009} by extending it with a financial accelerator mechanism in households' balance sheets. Macroprudential policy aims to stabilize credit markets by affecting the fraction of liabilities banks can lend.
	%\begin{itemize}
	%\item Aggregate Demand: Households in the core and in the periphery maximize expected lifetime utility by choosing consumption of durable and non-durable goods, and leisure. The composite non-durable consumption good consists of domestic non-durable (i.e., tradable) and foreign non-durable goods. Purchases of durable goods take the form of residential investment. Households in each country can save in deposits and bonds, and can take out one-period loans from domestic competitive financial intermediaries.
	%\item Aggregate Supply: Each economy is characterized by two sectors. Monopolistic firms (subject to Calvo-style rigidities) use household labor to produce durable and non-durable intermediate goods that are combined by competitive final-good producers into durable and non-durable goods. Imperfect substitutability of labor is assumed between the two sectors and wages are flexible. Final durable goods are sold only to domestic households, which they use to increase the value of their housing – subject to adjustment costs.
	%\item Financial Sector and Macroprudential Policy: Impatient households finance part of their residential investment through loans subject to a contract, analogous to that of \cite{BernankeGertlerGilchrist1999}, where default occurs when the value of their outstanding debt is higher than the value of the house they own (which is common knowledge) – depending on the realization of their idiosyncratic “housing quality shock.” This induces a spread between the lending (i.e., mortgage) and deposit rates, which depends on housing market conditions. Savers' funds are channeled across countries through international financial intermediaries which trade domestic financial intermediaries' bonds and charge a risk premium, which depends on the net foreign asset position of the country.
	%\item	Shocks: Thirteen shocks are present in the model: four sector-specific technology shocks (two for each country), four preference shocks (one for each type of good in each country), two housing quality variance shocks (one for each country), a risk premium shock, and two union-wide shocks (technology and monetary policy).
	%\item Calibration/Estimation: The model is estimated by means of Bayesian techniques using quarterly euro area data for the period 1995:Q4 to 2011:Q4. The core country is an aggregate of France and Germany and the periphery is represented by the GDP-weighted average of Greece, Ireland, Italy, Portugal and Spain.
	%\end{itemize}
	\subsection{US\_MR07: \cite{MankiwReis2007}}
	\label{USMR07}
	\cite{MankiwReis2007} develop a general equilibrium model where rigidities come from the fact that agents are inattentive and do not update information regularly when setting prices, wages and deciding on consumption. {US\_MR07} is a model with information stickiness. Estimation of the model using U.S. data confirms the presence of such rigidities, especially for consumers and workers.
	
	\begin{itemize}
		
		\item Aggregate Demand: Infinitely lived households are of two types: consumers and workers. Their utility function is additively separable in consumption and leisure. They are able to save and borrow by trading bonds between themselves. Workers choose how much to work and what wage to charge for the particular variety of labor over which they hold a monopoly. Both consumers and workers take decisions but only a fraction of them, randomly drawn from their respective population, obtain new information and can re-optimize their actions. If they obtain new information, they revise their plans for future consumption and labor supply, respectively. Both, the aggregate demand (IS equation) and the equation of wages, depend on the sum of past expectations of current economic conditions, reflecting the fact that households have different sets of information. The stickier the information is (low share of informed households), the smaller the impact of shocks on spending and wages, since fewer consumers and workers are aware of them. The natural (long-run) equilibrium corresponds to a situation where all agents are perfectly informed.
		
		\item Aggregate Supply: Firms produce output using labor and sell their differentiated goods in a monopolistic competitive market. Firms are constrained in information gathering in the same fashion as households. Each period, a fraction of firms, randomly drawn from the population, obtains new information and recalculates the optimal price. The optimizing process of the firms leads to a Phillips curve equation where the price level is determined as a sum of past expectations of current economic conditions (prices, output, marginal costs, technology shocks). The summation captures the fact that firms have different sets of information. Shocks to the variables in the Phillips curve equation will have gradual effects as some firms remain unaware of these shocks and only react to them once they update their information set.
		
		\item Shocks: A mark-up good shock, a mark-up labor shock, a government shock, a technology shock and a monetary policy shock.
		
		\item Calibration/Estimation: Estimated with maximum likelihood and Bayesian methods, using quarterly U.S. data for the period 1954:Q3--2006:Q1.
		
		%\item Replication: The impulse responses of the output gap, inflation and hours worked are well replicated using 30 lags of information.
		
	\end{itemize}
	
	
	\subsection{US\_OR03:\cite{Orphanides2003}}
	\label{USOR03}
	
	\cite{Orphanides2003} conducts a counterfactual analysis based on the historical experience of the United States economy to give an example of the difficulties in identifying robust policy strategies. The counterfactual analysis gives an insight how inflation and the output gap would have evolved from the 1960s to the 1990s if the Federal Reserve had actually followed two distinct activist monetary policy rules taking into account the difference between realistic and non-realistic assumptions on the availability of information on the output gap.
	\begin{itemize}
		\item Aggregate demand: The demand side of the structural model of the economy is represented by an IS equation which relates the output gap to its own lags, lags of inflation and the federal funds rate.
		\item Aggregate supply: The supply side is represented by an accelerationist form of the Philips curve with an adaptive representation of inflation expectations.
		%\item Variable dimension: Variables are expressed in percent.
		\item Shocks: A cost-push shock, a demand shock and the common monetary policy shock.
		\item Calibration/estimation: The Aggregate Demand and Aggregate Supply equation are estimated in a setup that can be interpreted as a mildly restricted structural vector autoregression (VAR) of up to four lags estimated using quarterly data from 1960 to 1993.
		%\item Replication:
		%\item Impulse responses: Figure \ref{img:US_ORP03}
	\end{itemize}
	
	
	\subsection{US\_OW98: FRB Monetary Studies, \cite{OrphanidesWieland1998}}
	\label{USOW98}
	This is a small open economy model described in \cite{OrphanidesWieland1998} and used to
	investigate the consequences of the zero bound on nominal interest rates.
	%Taking the zero bound into account worsens the economic outcome only for inflation targets between 0 and 1 percent. Output volatility increases significantly. The long-run Phillips curve becomes non-vertical. Output falls short of potential on average.
	\begin{itemize}
		\item Aggregate Demand: The US\_OW98 model disaggregates real spending into five components: private consumption, fixed
		investment, inventory investment, net exports, and government purchases. The aggregate demand
		components exhibit partial adjustment to their respective equilibrium levels, measured as shares of
		potential GDP. Partial adjustments reflect habit persistence. Equilibrium consumption and fixed investment are functions of permanent income (discounted at 10 percent) and depend on the long-term real rate. The long-term nominal interest rate is an average of expected future nominal short-term rates. The long-term real rate is determined by the Fisher equation. Inventory investment depends on three lags of output. Government spending is an AR(1) process.
		\item Aggregate Supply: The structure is similar to the US\_FM95 model.
		In US\_FM95 and US\_OW98, the aggregate price level is a constant mark-up over the aggregate wage rate.
		\item Foreign Sector: Net exports depend on domestic output, world output, the real exchange rate and lagged net exports. The exchange rate is determined by an UIP condition.
		%\item Microeconomic foundation: no
		\item Shocks: Five demand shocks including the common fiscal policy shock in the government spending equation, an ad hoc cost push shock to the nominal wage contracts and the common monetary policy shock.
		%\item Variable dimension: Original variables are expressed in percent/100. The common Modelbase variables are expressed in percent.
		\item Calibration/Estimation: The model is estimated for the period 1980--1996 using U.S. data. The demand block is estimated via IV-estimation equation-by-equation. For the supply side simulation-based indirect inference methods are used.
		\item Replication: We replicated the impulse response functions for annualized quarterly inflation and the output gap to a 100 basis point innovation to the federal funds rate in Figure 2 of \cite{LevinWielandWilliams2003}.
		%\item Notes: the authors discuss the pros and cons of optimizing behavour of agents: they prefer a good empirical fit with white-noise structural shocks instead of ad-hoc serially correlated shocks as has been neccessary in comparable models with optimizing agents, see \citep{RotembergWoodford1997}.
		
		%\item Impulse responses: Figure \ref{img:US_MSR98}.
		
		%\item Impulse responses: The first row of figure \ref{img:MSR04} shows impulse responses to a one unit monetary policy shock. Despite the same wage/price contracts as in FM95, the responses are less persistent, but have a higher maximum effect. As the contract distribution is very similar to FM95, the reason is located in the demand block: due to lower coefficients in the MSR04 model, the lag structure in the demand curve is less persistent, even though the number of lags is higher than in FM95 which results in a higher maximal response of the outputgap. The lower demand persistence in turn lowers the persistence of inflation, as the demand terms enter with a higher weight ($0.055$) the pricing equations than in the FM95 model ($0.02$). This lower inflation persistence yields also a lower peak of the absolute inflation response. The second row shows impulse responses to an increase in government consumption of one percent of GDP. The strong effects might partly be attributed to the high AR(1) coefficient (0.98) in the government spending process.
	\end{itemize}
	
	
	
	
	\subsection{US\_PM08 and US\_PM08fl: \cite{Carabenciovetal2008}}
	\label{USPM08}
	\label{USPM08fl}
	
	\cite{Carabenciovetal2008} design and estimate two versions of a small projection model for the U.S. economy: one with financial real linkages, US\_PM08fl and one without, US\_PM08. These models are part of the IMF research agenda in developing a Small Quarterly Global Projection Model (GMP) which consists of many small country models integrated into a single global market. Both versions of the model consist of few behavioral equations, focusing on the joint determination of output, unemployment, inflation and the federal funds rate.
	
	\begin{itemize}
		
		\item Aggregate Demand: The behavioral IS curve relates the output gap to its past and expected future value, to the past value of the short interest rate gap and to a disturbance term. This specification allows for inertia and persistent effects of the shocks. In the model with financial linkages, {US\_PM08fl}, the output gap is a function of a financial variable as well, constructed using information from FED's quarterly Senior Loan Officer Opinion Survey on Bank Lending Practices. This variable enters in the form of a shock and it is supposed to reflect the bank lending conditions (tightening or loosening). Thus, if lending conditions are tighter than anticipated, the effect will be a lower output gap and a weaker economy.
		
		\item Aggregate Supply: In the Phillips curve equation, inflation is linked to its past and expected future values, to the lagged output gap and a disturbance term. This representation reflects the way agents set their prices: a share of them uses indexation to past inflation and others are forward looking. These expectations are based on model-consistent estimates of future inflation.
		
		\item Shocks: A shock to the level and the growth rate of potential output, a shock to the level and the growth rate of the equilibrium rate of unemployment, a shock to the equilibrium real interest rate. In the model with financial linkages, {US\_PM08fl}, a financial shock is introduced in addition and cross correlations of the error terms between certain shocks are allowed.
		
		\item Estimation: Both models are estimated with Bayesian techniques, using U.S. quarterly data over the period 1994:Q1--2008:Q1.
	\end{itemize}
	
	
	\subsection{US\_PV15: \cite{poutineau2015financial}}
	\label{USPV15}
	
	\cite{poutineau2015financial} evaluate the role of financial intermediaries, such as banks, on the extensive margin of activity. They build a DSGE model that combines the endogenous determination of the number of firms operating on the goods market with financial frictions through a financial accelerator mechanism, given the fact that the creation of a new activity partly requires loans to finance spending during the setting period. Three main results have been obtained. First, financial frictions play a key role in determining the number of new firms. Second, in contrast with real macroeconomic shocks (where investment in existing production lines and the creation of new firms move in the opposite direction), financial shocks have a cumulative effect on the two margins of activity, amplifying macroeconomic fluctuations. Third, the critical role of financial factors is mainly observed in the period corresponding to the creation of new firms.
	
	\begin{itemize}
		
		\item Aggregate Demand: There is a continuum of identical households who consume, save and work in intermediate firms. To single out the determination and the dynamics of nominal wages, it is assumed that households delegate the task of negotiating their salary to labor unions. Formally, households provide differentiated types of labor, sold by labor unions to perfectly competitive labor packers who assemble them in a CES aggregator and sell the homogenous labor to intermediate firms.
		
		\item Aggregate Supply: The firm sector is populated by two groups of agents: intermediate firms and final goods firms. Intermediate firms produce differentiated goods, choose labor and capital inputs, and set prices according to the \cite{Rotemberg1982} model. Final goods producers act as a consumption bundler by combining national intermediate goods to produce the homogenous final good. The total number of final firms/goods is normalized to 1, while the total number of intermediate firms/goods is endogenously determined in the model to define the extensive margin of activity. Each period, hence, a continuum of new firms decides to enter the market.
		
		\item Financial Sector: The economy is additionally populated by entrepreneurs, where the representative entrepreneur is a key agent for introducing financial frictions. This agent finances both the intensive margin (by renting capital to existin\`{a}g firms) and the extensive margin (by financing the wage bill for the creation of new firms). Entrepreneurs face a trade-off between intensive and extensive margins financing. Financial intermediaries provide funds to entrepreneurs. The representative financial intermediary collects deposits from households and lends them. From the balance sheet of the financial intermediary, the loan supply is equal to the deposits. Additionally, there is the imperfect pass-through of the policy rate on financial intermediary lending rate. It is assumed that financial intermediaries set their interest rates on a staggered basis with some degree of nominal rigidity \`{a} la \cite{Rotemberg1982}.
		
		\item Authorities: The government finances public spending by charging a tax on households. The total amount of public spending is assumed to evolve according to an AR(1) exogenous shock process. The central bank sets the interest rate in accordance with the fluctuations of price and activity imbalances.
		
		\item Shocks: There are 10 structural shocks in the model: productivity, spending, premium, investment, price cost-push, wage cost-push, rate cost-push, entry shock, collateral, and monetary policy shock.
		
		\item Estimation: The model is estimated with Bayesian methods on US quarterly data over the sample time period 1993Q1 to 2012Q3.
	\end{itemize}
	
	
	\subsection{US\_RA07: \cite{Rabanal2007}}
	\label{USRA07}
	\cite{Rabanal2007} incorporates a cost channel of monetary transmission into an otherwise standard medium-scale New Keynesian DSGE model by assuming that a fraction of firms need to borrow money to pay their wage bill prior to their sales receipts. The model is estimated on US data in order to analyze whether the cost channel empirically accounts for the so-called price puzzle.
	
	\begin{itemize}
		
		\item Aggregate Demand: Households obtain utility from consuming the final good and disutility from supplying labor, they own intermediate firms, lend capital services to firms and make investment and capital utilization decisions. Moreover, their utility function displays external habit formation. Capital is predetermined at the beginning of a period, but households can adjust its utilization rate subject to adjustment costs. Financial markets are assumed to be complete.
		
		\item Aggregate Supply: Intermediate good producers combine labor and capital services to produce their goods while taking the capital utilization rate decision of households as given. A fraction of intermediate good producers have to pay their wage bill every period before they sell their product. These firms borrow at the riskless nominal interest rate. Goods and labor markets are characterized by monopolistic competition. Prices and wages are set in a staggered way, following the formalism of \cite{Calvo1983}. Indexation to last period's average inflation rate is assumed for firms and households whenever they are not allowed to reoptimize. A continuum of final good producers operating under perfect competition uses intermediate goods for the production of final goods.
		
		\item Shocks: Four orthogonal structural shocks are introduced in the model. The government spending and technology shocks follow an AR(1) process. The monetary and the price markup shock are assumed to be iid processes.
		
		\item Calibration/Estimation: The model is estimated using Bayesian techniques on quarterly US data. The data set used comprises four key macroeconomic variables: real output, real wage, inflation rate and the nominal interest rate over the period 1959:Q1--2004:Q4.
		
		%\item Replication: We replicated the impulse response functions in Figure 1 of \cite{Rabanal2007}.
		
		%\item Note: The impulse responses as depicted in the paper were a result of typos in author's code confirmed by the author. Hence, the corrected version was implemented.
		
	\end{itemize}
	
	\subsection{US\_RE09: \cite{reis2009sticky}}
	\label{USRE09}
	\cite{reis2009sticky} presents a dynamic stochastic general-equilibrium model with a single friction in all markets: sticky information. In this economy, agents are inattentive because of costs of acquiring, absorbing and processing information, so that the actions of consumers, workers and firms are slow to incorporate news. The paper includes the details of how an economy with pervasive inattentiveness functions, then solves the model and in the end estimates it. Uncertainty in the model arises because every period there is a different realization of the random variables characterizing productivity, aggregate demand, price and wage markups, and monetary policy. While the expectations of each agent are formed rationally, they do not necessarily use all available information. More concretely, it assumes that there are fixed costs of acquiring, absorbing and processing information, so that agents optimally choose to only update their information sporadically.
	\begin{itemize}
		\item Aggregate Demand:  While the discussion presents consumers (shoppers and saver-planners) and workers separately, they are all members of one household. Representative household gains utility from consumption and leisure subject to its budget constraint. Shoppers consume a continuum of varieties of goods and determine demand for them, whereas saver-planners meet each other in the bond market in order to trade one-period bonds. In the labor market workers sell their labor. While inattentiveness occurs in all markets, not all agents in this economy are inattentive. In the goods market, the model assumes that the consumer is separated into two units: the saver-planner who updates information infrequently and the shopper who knows about the expenditure plan of the saver and observes the relative prices of the different goods. Additionally, separating consumers from workers allows them to potentially update their information at different frequencies. They do not necessarily need to share information, although belong to the same household. When workers update their information, they also learn about what the consumer has been doing, and vice-versa for consumers.	
		\item Aggregate Supply:  On the selling side of the market, there are monopolistic firms for each variety of the good. They operate a technology that uses labor in order to produce goods under diminishing returns to scale and common technology shock.
		
		\item Shocks: There are shocks in technology, monetary policy, aggregate demand, goods substitutability and labor substitutability.
		
		\item Calibration/Estimation: The model is estimated using full-information techniques that exploit the restrictions imposed by general equilibrium. Quarterly observations are used for two large economies: the United States from 1986:3 to 2006:1 and the Euro area from 1993:4 to 2005:4.
		
		
	\end{itemize}
	
	
	
	\subsection{US\_RS99: \cite{RudebuschSvensson1999}}
	\label{USRS99}
	\cite{RudebuschSvensson1999} set up a simple linear model of the U.S. economy which is used to examine the performance of different policy rules taking into account an inflation targeting monetary policy regime. The model equations are backward looking.
	\begin{itemize}
		\item Aggregate Demand: An IS curve relates the output gap to its own lags and the difference between the average federal funds rate and the average inflation rate over the current and three preceding quarters.
		\item Aggregate Supply: Phillips curve of the accelerationist form.
		\item Shocks: A cost-push shock, a demand shock and the common monetary policy shock.
		%\item Variable dimension: Variables are demeaned and expressed in percent.
		\item Calibration/Estimation: The model equations are estimated individually by ordinary least squares for U.S. data. The sample period comprises 1961:1-1996:2.
		%\item Replication: HAS TO BE FILLED IN
		%\item Impulse Responses: Figure \ref{img:US_RS99}
	\end{itemize}
	
	
	
	
	
	\subsection{US\_SW07: \cite{SmetsWouters2007}}
	\label{USSW07}
	
	
	
	\cite{SmetsWouters2007} develop a medium-scale closed economy DSGE-Model and estimate it for the U.S. with Bayesian techniques. The model features a deterministic growth rate driven by labor-augmenting technological progress, so that the data do not need to be detrended before estimation.
	
	\begin{itemize}
		\item Aggregate Demand: Households maximize their lifetime utility, where the utility function is nonseparable in consumption and leisure, subject to an intertemporal budget constraint. \cite{SmetsWouters2007} include external habit formation to make the consumption response in the model more persistent. Households own firms, rent capital services to firms and decide how much capital to accumulate given certain capital adjustment costs. They additionally hold their financial wealth in the form of one-period, state-contingent bonds. Exogenous spending follows a first-order autoregressive process with an iid-normal error term and is also affected by the productivity shock.
		\item Aggregate Supply: The final goods, which are produced under perfect competition, are used for consumption and investment by the households and by the government. The final goods producer maximizes profits subject to a \cite{Kimball1995} aggregator of intermediate goods, which introduces monopolistic competition in the market for intermediate goods and features a non constant elasticity of substitution between different intermediate goods, which depends on their relative price. A continuum of intermediate firms produce differentiated goods using a production function with Cobb-Douglas technology and fixed costs and sell these goods to the final-good sector. They decide on labor and capital inputs, and set prices according to the Calvo model. Labor is differentiated by a union using the Kimball aggregator, too, so that there is some monopoly power over wages, which results in an explicit wage equation. Labor packers buy the labor from the unions and resell it to the intermediate goods producer in a perfectly competitive environment. Sticky wages \`{a} la Calvo are additionally assumed. The Calvo model in both wage and price setting is augmented by the assumption that prices that can not be freely set, are partially indexed to past inflation rates.
		%\item Microeconomic foundation: yes
		\item Shocks: A total factor productivity shock, a risk premium shock, an investment-specific technology shock, a wage and a price mark-up shock and two policy shocks: the common fiscal policy shock entering the government spending equation and the common monetary policy shock.
		%\item Variable dimension: The model is log-linearized around the steady state. Variables are expressed in terms of percentage deviations from steady state.
		\item Calibration/Estimation: The model is estimated for the U.S. with Bayesian techniques for the period 1966:1--2004:4 using seven key macroeconomic variables: real GDP, consumption, investment, the GDP deflator, real wages, employment and the nominal short-term interest rate.
		\item Replication: We replicated the impulse response functions to a positive one standard deviation monetary policy shock in Figure 6 of \cite{SmetsWouters2007}. The variables include output, hours, quarterly inflation and the interest rate.
		
		%\item Replication: we estimated the model with Bayesian techniques using the same data set as SW and replicated the estimation results in \cite{SmetsWouters2007} (Table 1A and 1B). We used the means of the posterior distributions as fixed parameter values for the simulation of the model.
		
		%\item Impulse responses: Figure \ref{img:US_SW07}.
		
		%\item Impulse responses: The first row of figure \ref{img:SW07} shows impulse responses to a one unit monetary policy shock. The second row shows impulse responses to an increase in government consumption of one percent of GDP. The AR(1) coefficient of the shock process is 0.98.
	\end{itemize}
	
	\subsection{US\_VI16bgg and US\_VI16gk: \texorpdfstring{\cite{villa2016}}{Villa (2016)}}
	\label{USVI16}
	\cite{villa2016} assesses the empirical relevance of financial frictions in the US and in the Euro Area, where the above versions of the model refer to the US. It develops a medium-scale closed economy DSGE-model based on \cite{SmetsWouters2007} and two different financial sector extensions of this framework, in particular \cite{bernanke1996financial} and the \cite{GertlerKaradi2011} types.
	
	
	\begin{itemize}
		
		\item US\_V16bgg model:
		
		\begin{itemize}	
			
			\item Aggregate Demand: Households maximize their lifetime utility, where the utility function is separable in consumption and leisure, subject to an intertemporal budget constraint. In addition, the external habit formation makes the consumption response more persistent. Households own firms, rent capital services to firms and decide how much capital to accumulate given certain capital adjustment costs. They additionally hold their financial wealth in the form of one-period, state-contingent government bonds.
			
			\item Aggregate Supply: Intermediate good firms maximize their profits by choosing factors of production and by signing a financial contract to obtain additional funds from lenders. Since lenders have to pay some auditing costs to observe the idiosyncratic return to capital, an agency problem arises. The financial contract implies external finance premium that depends on the inverse of the firm's leverage ratio. Retailers buy goods from intermediate good firms, differentiate them, and sell them in a monopolistically competitive market according to the Calvo model. The aggregate final good is assembled by perfectly competitive final good firms, and is used for consumption and investment by the households and by the government. The final goods producer maximizes profits subject to a Dixit-Stiglitz aggregator of intermediate goods, which introduces monopolistic competition in the market for intermediate goods and features a constant elasticity of substitution between individual, intermediate goods. Labor is differentiated by a union using the Dixit-Stiglitz aggregator, too, so that there is some monopoly power over wages, which results in an explicit wage equation. Labor packers buy the labor from the unions and resell it to the intermediate goods producer in a perfectly competitive environment. Sticky wages à la Calvo are additionally assumed. 
			
		\end{itemize}
		
		\item US\_V16gk model:
		
		
		\begin{itemize}	
			
			\item Aggregate Demand: The representative household's utility is separable in consumption and leisure and allows for habit formation in consumption. Households postpone their consumption by holding deposits with the financial intermediaries. The amount of deposits is determined in such a way as to guarantee that the bankers' incentive constraint is satisfied. Expected-lifetime utility is maximized by choosing consumption and labor supplied to intermediate firms.
			
			\item Aggregate Supply: Competitive firms produce intermediate goods using labor services and capital. They face adjustment costs for varying their utilization rate of capital. They finance the capital stock with loans from the financial intermediaries and buy it from capital producing firms to which they re-sell it at the end of the period after having used it. Capital producers face investment adjustment costs. The intermediate goods are bought by retail firms, which act under monopolistic competition and face nominal rigidities as in \cite{Calvo1983}. Non-reoptimizing retailers index their prices to the previous period's inflation rate.
			
			\item Banking sector: Banks receive their funds in the form of deposits from households and lend to non-financial firms. A moral hazard/costly enforcement problem constrains the ability of banks to obtain funds from households, while they are able to perfectly monitor firms and enforce contracts.
			
		\end{itemize}
		
		
		\item Shocks: Seven structural shocks: the technology shock, the investment-specific technology shock, the capital quality shock, the price mark-up shock, the wage mark-up shock, and two policy shocks - the common fiscal policy shock entering the government spending equation and the common monetary policy shock.
		
		\item Estimation: The model is estimated for EA and US with Bayesian techniques for the period 1983:Q1-2008:Q3 using seven key macroeconomic variables: real GDP, real investment, real private consumption, hours worked, GDP deflator inflation, real wage, and the nominal short-term interest rate.
		
		\item Replication: The impulse response functions to negative one-standard-deviation shocks were replicated, similar to those in Figures 3-6 in \cite{villa2016}. The variables include output, investment, inflation, net worth and spread.
		
		
	\end{itemize}
	
	
	
	
	\subsection{ US\_VMDno, US\_VMDop: \cite{Veronaetal2013}}
	\label{USVMDno}
	\label{USVMDop}
	The US\_VMD model is an extended model of \cite{Christianoetal2010} with a shadow banking system to account for the pattern of financial and economic boom-bust. Two versions of the model are calibrated based on the bond spread over the long run and during the 2000s' boom: the model for normal times (US\_VMDno model) and the model for times of over-optimism (US\_VMDop model).
	\begin{itemize}
		\item Aggregate Demand: As in US\_CMR10
		\item Aggregate Supply: As in US\_CMR10
		\item Financial System: It consists of loan market (traditional banking system) and bond market (shadow banking system). The loan market is modeled as a risky debt contract between the entrepreneurs and perfectly-competitive retail banks under monitoring cost in the form of costly state verification as \cite{BernankeGertlerGilchrist1999}. Retail banks can diversify the idiosyncratic risk of many entrepreneurs and thus can generate a safe return on households' deposit. In the bond market, there are a continuum of safer entrepreneurs who are assumed not to default and investment banks who has some monopolistic power. The coupon rate of bonds issued is determined as a time-varying markup over the risk-free interest rate, which is a function of the elasticity of the demand for funds in the bond market. The elasticity of the demand in times of over-optimism is calibrated to be more elastic than one in normal times.
		\item Shocks: A monetary policy shock
		\item Calibration/Estimation: The model is calibrated for the U.S. economy. The parameters related to the bond market and entrepreneurs who rely on bond finance are calibrated to match empirical statistics to their counterpart of the model. The values of the remaining parameters are taken from other studies, especially from \cite{Christianoetal2010}.
	\end{itemize}
	
	
	
	\subsection{US\_YR13: \texorpdfstring{\cite{rychalovska2016}}{Rychalovska (2016)}}
	\label{USYR13}
	\cite{rychalovska2016} incorporates financial frictions combined with an imperfectly rational expectation formation mechanism into a medium-scale DSGE model based on \cite{SmetsWouters2007}. The financial frictions are integrated in the form of the financial accelerator as originally applied in \cite{bernanke1989agency} and \cite{bernanke1996financial}. The model contains a number of nominal and real rigidities such as monopolistic competition on goods and labor markets, Calvo price and wage stickiness, habit formation in consumption and capital adjustment costs. The paper explores the properties of the model assuming, on the one hand, complete rationality of expectations and, alternatively, several learning algorithms that differ in terms of the information set used by agents to produce the forecasts. The results suggest that the learning scheme based on small forecasting functions is able to amplify the effects of financial frictions relative to the model with Rational Expectations.
	
	\begin{itemize}
		
		\item Aggregate Demand: As in \cite{SmetsWouters2007}, households maximize their lifetime utility function, non-separable in consumption and leisure, subject to an intertemporal budget constraint. Preferences for consumption are subject to habit persistence. Households own firms and supply labor monopolistically. Wage stickiness is introduced via the Calvo framework.
		
		\item Aggregate Supply: Apart from the intermediate and final goods firms as in \cite{SmetsWouters2007}, following \cite{BernankeGertlerGilchrist1999} and \cite{ChristianoMottoRostagno2003}, capital goods producers and entrepreneurs are introduced into the model. Competitive capital-goods producers, owned by households, produce new capital goods which are sold to entrepreneurs. Capital-goods producers combine investment goods, purchased from the final good producers, with the existing capital stock, rented from the entrepreneurs, to produce new capital goods. Capital-goods producers are subject to quadratic adjustment costs. In the formal representation of the entrepreneurs' problem the paper follows \cite{christiano2010financial} and deviates from the original \cite{BernankeGertlerGilchrist1999} specification, assuming that entrepreneurs are not directly involved in the production of intermediate goods. In addition, banks are also introduced into the model, which interact with entrepreneurs and bring financial frictions into play. Entrepreneurs, who are risk neutral and survive until the next period with a certain probability, use their own funds (the net worth) and loans from the bank to finance capital that is rented to the production sector. Competitive banks finance the loans by accepting deposits from the households at the risk-free rate while entrepreneurs have to pay an external finance premium over the riskless rate in order to borrow funds.
		
		\item Shocks: total factor productivity shock, investment-specific shock, external financing premium shock, fiscal shock, a monetary policy shock, a wage and price mark-up shock.
		
		\item Estimation: The model is estimated using seven macroeconomic quarterly U.S. time series: real GDP, real consumption, real investment, real wage, hours worked, GDP deflator and the federal funds rate. The data are quarterly for the sample period 1954:1-2008:3.
		
	\end{itemize}
	
	
	
	%\subsection{US\_NFED08: \cite{EdgeKileyLaforte2007}}
	%\label{USNFED08}
	%The US\_NFED08 is a version of the medium-scale closed economy model as in \cite{EdgeKileyLaforte2007} used for estimation in \cite{WielandWolters2011}. In this model, specifications regarding production and expenditures are motivated by the long-run and cyclical properties observed in the U.S. data. Production sectors in \cite{EdgeKileyLaforte2007} differ in the rate of the technological growth while expenditures are categorized as business spending and household spending. The model as in \cite{EdgeKileyLaforte2007} is used at the Federal Reserve Board as a complimentary model for policy analysis along FRB/US and other small models.
	%
	%\begin{itemize}
	%	
	%	\item Aggregate Demand: Households derive utility from four sources: purchases of the consumer non-durable goods and non-housing services, the flow of services from their rental of consumer-durable capital, the flow of services from their rental of residential capital, and leisure. Internal habit persistence is present in all three components of consumption. Households supply differentiated labor to two production sectors. They face quadratic wage adjustment costs when setting wages. Furthermore, they face additional costs when changing the mix of labor supplied to each of the production sectors. The consumption components and the disutility from labor are subject to specific AR(1) aggregate shocks.
	%	
	%	\item Aggregate Supply: There are two production sectors in this model, differing on what type of final goods and services they are producing. One of the sectors (comprised of businesses and institutions) produces slow-growing ``consumption'' goods and services while the other sector (only businesses) produces fast-growing ``capital'' goods. Final goods are an aggregate (using Dixit-Stiglitz technology) of sector-specific differentiated intermediate goods. The latter are produced by intermediate goods producers by combining aggregated labor with utilized non-residential capital in a Cobb-Douglas production function. Labor input for each sector is aggregated using Dixit-Stiglitz technology. The level of productivity in the Cobb-Douglas production function has a common and a sector specific factor. Based on historical data for the U.S., faster technological progress for capital-specific goods is assumed. Price setting decisions (under price adjustment costs) of intermediate goods firms deliver a New Keynesian Phillips curve with backward and forward-looking terms. Capital owners choose how much residential capital, non-residential capital and consumer durables will be invested in each production sector. These decisions are subject to investment and capital adjustment costs. In addition, the decision for the utilization of non-residential capital is subject to utilization costs.
	%	
	%	\item Shocks: A shock to preferences over durables, a shock to preferences over non-durables and non-housing services, a shock to preferences over residential capital, a shock to preferences over leisure, a shock to efficiency of investment in non-residential capital, a shock to efficiency of investment in residential capital, a shock to efficiency of investment in consumer durable goods, a mark-up shock, a shock to the elasticity of substitution between the differentiated intermediate goods inputs, an economy-wide productivity shock, a sector specific productivity shock, an intermediate labor substitution shock, a labor substitution shock, a monetary policy shock.
	%	
	%	\item Calibration/Estimation: Estimated with Bayesian methods, using quarterly U.S. data for the period 1984:Q1--2004:Q4.
	%	
	%	%\item Replication: Check the model in the Modelbase!.
	%	
	%\end{itemize}
	
	%\subsection{US\_NFED08: \cite{EdgeKileyLaforte2007}}
	%\label{USNFED08}
	%The US\_NFED08 is a version of the medium-scale closed economy model as in \cite{EdgeKileyLaforte2007} used for estimation in \cite{WielandWolters2011}. In this model, specifications regarding production and expenditures are motivated by the long-run and cyclical properties observed in the U.S. data. Production sectors in \cite{EdgeKileyLaforte2007} differ in the rate of the technological growth while expenditures are categorized as business spending and household spending. The model as in \cite{EdgeKileyLaforte2007} is used at the Federal Reserve Board as a complimentary model for policy analysis along FRB/US and other small models.
	
	%\begin{itemize}
	
	%	\item Aggregate Demand: Households derive utility from four sources: purchases of the consumer non-durable goods and non-housing services, the flow of services from their rental of consumer-durable capital, the flow of services from their rental of residential capital, and leisure. Internal habit persistence is present in all three components of consumption. Households supply differentiated labor to two production sectors. They face quadratic wage adjustment costs when setting wages. Furthermore, they face additional costs when changing the mix of labor supplied to each of the production sectors. The consumption components and the disutility from labor are subject to specific AR(1) aggregate shocks.
	
	%	\item Aggregate Supply: There are two production sectors in this model, differing on what type of final goods and services they are producing. One of the sectors (comprised of businesses and institutions) produces slow-growing ``consumption'' goods and services while the other sector (only businesses) produces fast-growing ``capital'' goods. Final goods are an aggregate (using Dixit-Stiglitz technology) of sector-specific differentiated intermediate goods. The latter are produced by intermediate goods producers by combining aggregated labor with utilized non-residential capital in a Cobb-Douglas production function. Labor input for each sector is aggregated using Dixit-Stiglitz technology. The level of productivity in the Cobb-Douglas production function has a common and a sector specific factor. Based on historical data for the U.S., faster technological progress for capital-specific goods is assumed. Price setting decisions (under price adjustment costs) of intermediate goods firms deliver a New Keynesian Phillips curve with backward and forward-looking terms. Capital owners choose how much residential capital, non-residential capital and consumer durables will be invested in each production sector. These decisions are subject to investment and capital adjustment costs. In addition, the decision for the utilization of non-residential capital is subject to utilization costs.
	
	%	\item Shocks: A shock to preferences over durables, a shock to preferences over non-durables and non-housing services, a shock to preferences over residential capital, a shock to preferences over leisure, a shock to efficiency of investment in non-residential capital, a shock to efficiency of investment in residential capital, a shock to efficiency of investment in consumer durable goods, a mark-up shock, a shock to the elasticity of substitution between the differentiated intermediate goods inputs, an economy-wide productivity shock, a sector specific productivity shock, an intermediate labor substitution shock, a labor substitution shock, a monetary policy shock.
	
	%	\item Calibration/Estimation: Estimated with Bayesian methods, using quarterly U.S. data for the period 1984:Q1--2004:Q4.
	
	%\item Replication: Check the model in the Modelbase!.
	
	%\end{itemize}
	
	\section{Estimated Euro Area Models}
	
	\subsection{EA\_ALSV06: \texorpdfstring{ \cite{andres2006lopezsalido}}{Andrés et al. (2006)}}
	\label{EAALSV06}
	\cite{andres2006lopezsalido} develop a small-scale New Keynesian model with real money balances entering both the forward-looking IS curve and the Phillips curve. The model is used to assess the role played by money in the joint evolution of output, interest rates and inflation in the euro area economy. The authors find no effect of real balances on marginal utility of consumption and that prices, output and interest rates are mainly explained by real shocks and not money demand shocks.
	
	\begin{itemize}
		
		\item Aggregate Demand: A representative household maximizes expected utility, non-separable between consumption and real money balances while separable in hours worked, subject to a budget constraint. The utility function features habit formation in consumption. The optimizing behavior leads to a forward looking IS curve in which real balances enter. This is due to the intra-temporal non-separability of real balances and consumption in the utility function. Further, due to the habit formation in consumption, the IS curve includes a lag
		in output and two period leads in output, real balances, as well as the money demand and preference shocks.
		
		\item Aggregate Supply: A continuum of firms produces goods according to Cobb-Douglas production functions using labor as input. Firms sell their output in a monopolistically competitive market. Nominal prices are set on a staggered basis as in Calvo. Further, some of the adjusting firms use a backward-looking rule of thumb while the rest of the firms adjust their prices based on optimization of expected future revenues. The optimizing behavior of firms leads to a forward-looking Phillips curve. Due to the rule-of-thumb adjusting firms, the Phillips curve also has a backward-looking component. Via real marginal costs, besides the technology shock, also output and real balances enter the Phillips curve; under habits also lags and/or leads in output, real balances as well as money demand and preference shocks enter the specification.
		
		\item Shocks: An overall preference shock, a real money balances preferences (velocity) shock, a productivity (technology) shock, and a monetary policy shock.
		
		\item Estimation: Estimated via maximum likelihood using EA quarterly data over the period 1980:1-1999:4 for logs of detrended output, detrended real balances, inflation and gross nominal interest rates. Output, inflation and interest rate data come from the Area Wide Model dataset, while real balances are measured dividing M3 by the GDP deflator.
		
	\end{itemize}
	
	
	
	\subsection{EA\_AWM05: Area Wide model linearized by \cite{DieppeKuesterMcAdam2005}}
	\label{EAAWM05}
	The model is described in \cite{FaganHenryMestre2005}. It was one of the first models to treat the Euro area as a single economy. In the Modelbase we use the linearized version from \cite{DieppeKuesterMcAdam2005} that is also used in \cite{KuesterWieland2005}. The EA\_AWM05 is an open economy model of the Euro area. Expectation formation is largely backward-looking. Activity is demand-determined in the short-run but supply-determined in the long-run with employment having converged to a level consistent with the exogenously given level of equilibrium unemployment. Stock-flow adjustments are accounted for, e.g., the inclusion of a wealth term in consumption.
	
	\begin{itemize}
		%\item Purpose of the Model: Building a model for the Euro area for simulation and forecasting purposes.
		\item Aggregate Demand: Demand is disaggregated into private consumption, government consumption, investment, variation of inventories, exports, and imports. The term structure (12-year bond) is forward-looking. Private consumption is specified as a function of households' real disposable income and wealth, where the latter consists of net foreign assets, public debt and the capital stock. The change in the log of the investment/output ratio depends on the real interest rate, the real GDP/capital stock ratio and the lagged investment/output ratio. The authors stress that this investment equation represents the key channel through which interest rates affect aggregate demand. Government consumption is treated as exogenous.
		\item Aggregate Supply: Output follows a whole economy production function.  Short-run employment dynamics are driven by output growth and real wages. The deflator for real GDP at factor costs, which according to \cite{FaganHenryMestre2005} is the key price index of the model, is a function of unit labor costs, import prices, the output gap and inflation expectations. The growth rate of wages depends on consumer price inflation, productivity and the unemployment gap, defined as the deviation of the current unemployment rate from the NAIRU.
		\item Foreign sector: Besides extra-area flows, exports and imports also include intra-area flows. %Exports depend on world demand.
		World GDP and world GDP deflator are treated as exogenous variables. The exchange rate is a forward-looking variable determined by uncovered interest rate parity.
		%\item Microeconomic foundation: no
		\item Shocks: Employment shock, factor cost-push shock, private consumption cost-push shock, gross investment cost-push shock, gross investment shock, exports cost-push shock, imports cost-push shock, private consumption shock, term structure shock, common fiscal policy shock and common monetary policy shock.
		%\item Variable dimension: Original variables are expressed in percent/100. The common Modelbase variables are expressed in percent.
		\item Calibration/Estimation: Estimation on Euro area data equation by equation from 1970:1--1997:4, whereas the estimation period of some equations starts later, but not later than 1980:1.
		%\item Replication: We replicated the impulse response functions of annual inflation and the output gap to a 100bps temporary unanticipated rise in the nominal short term rate in the upper panel of Figure 7 of \cite{KuesterWieland2005}.
		%\item Replication: Using the policy rule by \cite{GerdesmeierRoffia2004} we can replicate the impulse responses presented in \cite{KuesterWieland2005}. Note, in the version in this modelbase the original fiscal policy rule is substituted by a fiscal policy rule that does not lead to permanent shocks. For the replication exercise one must use the original rule. Otherwise one gets slight differences compared to \cite{KuesterWieland2005}. Furthermore, one has to use output instead of the output gap in the policy rule.
		
		%\item Impulse responses: Figure \ref{img:EA_AWM05}.
		
		%\item Impulse responses: The first row of figure \ref{img:AWM} shows impulse responses to a one unit monetary policy shock. The high persistence implied by the backward-looking expectations of the model can be seen in the graph. The second row shows impulse responses to a government spending shock that equals one percent of GDP. The AWM is estimated assuming that government spending is non-stationary. Hence the original rule adds a shock to the first difference of government spending. To make our experiment of a temporary shock comparable across models we assume an AR(1) process for government spending and scale the shock with the inverse of the share of government spending in GDP, which is equal to 0.14. Thus, a one unit shock equals one percent of GDP. The AR(1) coefficient is 0.97.
	\end{itemize}
	
	
	\subsection{EA\_BE15: \cite{benchimol2015money}}
	\label{EABE15}
	\cite{benchimol2015money} checks whether money is an omitted variable in the production process. For this purpose he estimates a NK-model with real money balances in the production func-tion on European data. While he finds no strong evidence for the importance for money in production, the demand for money by firms plays a role for the economy.
	
	\begin{itemize}
		\item Aggregate Demand: A representative household maximizes expected lifetime utility by optimizes consumption, labor supply, money holdings and bond holdings. Money enters the utility function.
		\item Aggregate Supply: Firms are monopolistic competitors and prices are sticky (Calvo pricing). The production factors are labor, (exogenous) technology and real balances held by the firm. Labor markets are frictionless. 
		\item Shocks: In the model there is a technology shock, a preference shock, a monetary policy shock, and two money demand shocks (one for households' and one for firms' demand for money).
		\item Estimation: The model is estimated on quarterly Eurozone data using the observables GDP per capita, GDP deflator, short term nominal interest rate and M3.
	\end{itemize}
	
	
	\subsection{EA\_BF17: \cite{benchimol2017money}}
	\label{EABF17}
	\cite{benchimol2017money} analyse the role of money and monetary policy as well as the forecasting performance of NK models with and without separability between consump-tion and money over three crisis periods in the Eurozone (ERM crisis, dot-com crisis and global financial crisis). For this purpose they estimate an NK-model on European data, and find that the nonseparable model generally provides better forecasting performance. Addi-tionally, they find that the effects of monetary policy differ across the three crises.
	
	\begin{itemize}
		\item Aggregate Demand: A representative household optimizes consumption, labor supply, money holdings and bond holdings. Money features the utility function. In the MMB the model with nonseparabilities between consumption and real balanced is implemented.
		\item Aggregate Supply: Firms are monopolistically competitive and face price rigidities (Calvo pricing). Labor is the only factor for production. 
		\item Shocks: The model features a technology shock, a price markup shock as well as an interest rate shock and a shock to the money supply.
		\item Estimation: The model is estimated on quarterly Eurozone data using the observables GDP per capita, GDP deflator, short term nominal interest rate and M3.
	\end{itemize}
	
	
	
	\subsection{EA\_CKL09: \texorpdfstring{\cite{ChristoffelKuesterLinzert2009}}{Christoffel et al. (2009)}}
	%\label{NKCKL09}
	\label{EACKL09}
	
	\cite{ChristoffelKuesterLinzert2009} explore the role of labor markets for monetary policy in the Euro Area in a closed-economy, single-country New Keynesian model with \cite{MortensenPissarides1994} type of matching frictions. To allow for a direct channel from wages to inflation, the model builds on the right-to-manage framework of \cite{Trigari2006}. Moreover, \cite{ChristoffelKuesterLinzert2009} incorporate staggered wage-setting \`{a} la Calvo and account for job-related fixed costs as in \cite{ChristoffelKuester2008}. The aim of the paper is to investigate to which extent a more flexible labor market would alter the business cycle behavior and the transmission of monetary policy, employing a genuine Euro Area calibration (NK\_CKL09,  see section  \ref{NKCKL09}). Second, by estimating the model with Bayesian techniques (EA\_CKL09) they analyze to which extent labor market shocks are important determinants of business cycle fluctuations. The results support current central bank practice to put considerable effort into monitoring Euro Area wage dynamics and treat some of the other market information as less important for monetary policy.
	
	\begin{itemize}
		
		\item Aggregate Demand: The demand as well as the supply structure follow closely the one described in \cite{ChristoffelKuester2008}. The economy consists of a large number of identical families that comprise unemployed and employed members with time-additive expected utility preferences that exhibit an external habit. The representative family pools the labor income of its working members, unemployment benefits of the unemployed members and financial income from assets that family members hold via a mutual fund. Each household also owns representative shares of all firms in the economy. It maximizes the sum of unweighted expected utilities of its individual members, by taking consumption, saving, vacancy posting, and labor supply decisions on their behalf.
		
		\item Aggregate Supply: The economy consists of three production sectors. The labor packers use exactly one worker as input to produce a homogeneous intermediate good labeled labor good. The process of labor bargaining is governed by wage rigidities. The wholesale sector buys the labor good from the labor packers in a perfectly competitive market and produces differentiated goods using a constant-return-to-scale production technology. These goods are sold under monopolistic competition to a final retail sector at a price that is subject to impediments \`{a} la Calvo and to a partial indexation rule. Retailers bundle the differentiated goods into a homogeneous consumption/investment basket and sell it to the consumers and to the government.
		
		\item Shocks: Three labor market shocks: a shock to the costs of posting a vacancy, a shock to the rate of separation, and a shock to the bargaining power of workers; a government spending shock; a wholesale sector cost-push shock.
		
		\item Calibration/Estimation: For the calibration exercise (NK\_CKL09) a quarterly Euro Area data set from 1984:Q1 to 2006:Q3 is used. The model is also estimated with Bayesian techniques (EA\_CKL09) employing output, year-on-year inflation, the nominal interest rate, wages per employee, unemployment and proxies for total hours worked and vacancies as observable variables.
		
		%\item Implementation:
		
	\end{itemize}
	
	
	
	\subsection{EA\_CW05: \cite{CoenenWieland2005}}
	\label{EACW05ta}
	\label{EACW05fm}
	\cite{CoenenWieland2005} develop a small-scale macroeconomic model for various staggered pricing schemes. We use a version with the nominal contract specification of \cite{Taylor1980}, labeled EA\_CW05ta, and a version with the relative real wage contract specification of \cite{FuhrerMoore1995}, labeled EA\_CW05fm.
	
	\begin{itemize}
		%\item Purpose of the Model: Develop and estimate a small structural model of the euro area to be used for evaluating alternative monetary policy strategies.
		\item Aggregate Demand: The aggregate demand equation is backward looking: two lags of aggregate demand (should account for habit persistence in consumption, adjustment costs and accelerator effects in investment) and one lag of the long-term interest rate (allows for a transmission lag of monetary policy). The long-term nominal interest rate is an average of expected future nominal short-term rates. The long-term real rate is determined by the Fisher equation.
		\item Aggregate Supply: As in US\_FM95 and US\_OW98.
		%\item The Foreign Sector: no foreign sector
		%\item Microeconomic foundation: no
		\item Shocks: A demand shock, a contract wage shock and the common monetary policy shock.
		%\item Variable dimension: Original variables are expressed in percent/100. The common Modelbase variables are expressed in percent.
		\item Calibration/Estimation: The model has been estimated on data from the ECB Area Wide Model data set from 1974:1--1998:4. The contract wage specifications have been estimated by a limited information indirect inference technique while the IS equation has been estimated by means of the GMM.
		%\item Replication: We replicated the impulse response functions of annual inflation and the output gap to a 100bps temporary unanticipated rise in the nominal short term rate in the upper panel of Figure 7 of \cite{KuesterWieland2005} for both versions of the model.
		%\item Impulse responses: Figure \ref{img:EA_CW05ta} (EA\_CW05ta), figure \ref{img:EA_CW05fm} (EA\_CW05fm).
		
		%\item Impulse responses: figure \ref{img:CW05ta} shows impulse responses for the model with Taylor contracts. Figure \ref{img:CW05fm} shows impulse responses for the model with Fuhrer-Moore contracts. The first row shows impulse responses to a one unit monetary policy shock. The second row shows impulse responses to a one unit demand shock that is added to the IS equation. One can see the higher inflation persistence implied by Fuhrer-Moore contracts.
	\end{itemize}
	
	
	\subsection{EA\_DKR11: \cite{DarracqPariesetal2011}}
	\label{EADKR11}
	The model of \cite{DarracqPariesetal2011} incorporates the imperfect competitive banking sector of \cite{Geralietal2010} in a DSGE model augmented by a financial accelerator in the spirit of \cite{BernankeGertlerGilchrist1999}. The model allows to examine the macroeconomic implications of various financial frictions on the supply and demand of credit.
	\begin{itemize}
		\item Aggregate Demand: The model features  patient (savers) and impatient (borrowers) households, as well as entrepreneurs. A representative household maximizes its intertemporal utility function choosing the level of consumption, the housing stock and labor supply, subject to a budget constraint. Wage-setting is subject to rigidities \`{a} la Calvo with indexation to both past and steady state inflation.
		
		Since impatient households can default on their loans, banks seize collateral. Thus, impatient households face a borrowing constraint, linked to the aggregate repayments and defaults.  Entrepreneurs maximize their utility by choosing consumption, physical capital, loans from banks, the degree of capacity utilization and labor. Entrepreneurs also face a borrowing constraint, linked to the aggregate repayments and defaults on outstanding debt.
		\item Aggregate Supply: The production sector consists of four types of firms: entrepreneurs, capital good and housing stock producers, retailers and the distribution sector. Entrepreneurs hire labor from households and buy capital from capital good producers to produce intermediate residential and non-residential goods. Capital and housing stock producers buy final goods to produce capital and the housing stock subject to investment adjustment costs. Retailers buy intermediate goods from entrepreneurs and differentiate them. Pricing is subject to nominal rigidities \`{a} la Calvo. The distribution sector aggregates differentiated goods and sells them to the agents.
		\item Banking Sector: The perfectly competitive wholesale branch receives funding from the money market and manages the capital position of the banking group. Loan book financing branches and the retail deposit branches work under monopolistic competition. The latter collects deposits from savers and places them in the money market. The loan book financing branches provide funding to the commercial lending branches who supply loans to entrepreneurs and impatient households under perfect competition.
		\item Shocks: Efficient shocks are technology shocks in both residential and non-residential sectors, a non-residential investment specific productivity shock, a labor supply shock, a public expenditure shock, a consumption preference shock and a housing preference shock. Inefficient shocks are a price markup shock in the non-residential sector and shocks to the markup on bank interest rates. Riskiness shocks are shocks to the standard deviation of the idiosyncratic risk for impatient households and entrepreneurs. In addition, there is a monetary policy shock and a shock to bank capital.
		\item The model is estimated with Bayesian techniques for the euro area for 1986:Q1 to 2008:Q2. For the estimation, fifteen observables are used: output, consumption, nonresidential fixed investment, hours worked, real wages, CPI inflation rate, three-month short-term interest rate, residential investment, real house prices, household vloans, non-financial corporation loans, household deposits, and bank lending rates on household loans, on non-financial corporation loans, and on household deposits.
	\end{itemize}
	
	
	%\newpage
	%\section{Estimated/Calibrated Multi-Country Models}
	%\subsection{G7\_TAY93: \cite{Taylor1993a} G7 countries}
	%\label{G7TAY93}
	%\cite{Taylor1993a} describes an estimated international macroeconomic framework for policy analysis in the G7 countries: USA, Canada, France, Germany, Italy, Japan and the UK. The model consists of 98 equations and a number of identities.
	%This model was the first to demonstrate that it is possible to construct, estimate, and simulate large-scale models for real-world policy analysis \citep{Yellen2007}. \cite{Taylor1993a} argues that a multicountry model is appropriate for the evaluation of policy questions like the appropriate mix of fiscal and monetary policy or the choice of an exchange rate policy.
	%\begin{itemize}
	%\item Aggregate Demand: The IS components are more disaggregated than in the US\_OW98 model. For example, spending on fixed investment is separated into three components: equipment, nonresidential structures, and residential construction. % see the number of shocks to get an overview about the level of disaggregation.
	%The specification of these equations is very similar to that of the more aggregated equations in the US\_OW98 model. The aggregate demand components exhibit partial adjustment to their respective equilibrium levels. In G7\_TAY93, imports follow partial adjustment to an equilibrium level that depends on U.S. income and the relative price of imports, while exports display partial adjustment to an equilibrium level that depends on foreign output and the relative price of exports. Uncovered interest rate parity determines each bilateral exchange rate (up to a time-varying risk premium); e.g., the expected one-period-ahead percent change in the DM/U.S.\$ exchange rate equals the current difference between U.S. and German short-term interest rates.
	%\item Aggregate Supply: The aggregate wage rate is determined by overlapping wage contracts. In particular, the aggregate wage is defined to be the weighted average of current and three lagged values of the contract wage rate. In contrast to the US\_FM95 model and the US\_OW98 model, G7\_TAY93 follows the specification in \cite{Taylor1980}, where the current nominal contract wage is determined as a weighted average of expected nominal contract wages, adjusted for the expected state of the economy over the life of the contract. This implies less persistence of inflation than in the US\_FM95 and the US\_OW98 model. The aggregate price level is not set as a constant mark-up over the aggregate wage rate as in US\_FM95 and US\_OW98. Prices are set as a mark-up over wage costs and imported input costs. This mark-up varies and prices adjust slowly to changes in costs. Prices follow a backward-looking error-correction specification. Current output price inflation depends positively on its own lagged value, on current wage inflation, and on lagged import price inflation, and responds negatively (with a coefficient of -0.2) to the
	%lagged percent deviation of the actual price level from equilibrium. Import prices adjust slowly (error-correction form) to an equilibrium level equal to a constant mark-up over a weighted average of foreign prices converted to dollars. This partial adjustment of import and output prices imposes somewhat more persistence to output price inflation than would result from staggered nominal wages alone.
	%\item Foreign sector: G7\_TAY93 features estimated equations for demand components and wages and prices for the other G7 countries at about the level of aggregation of the U.S. sector. Financial capital is mobile across countries.
	%%\item Microeconomic foundation: no
	%\item Shocks: Interest rate parity shock, term structure shock, durable consumption shock, nondurable consumption shock, services consumption shock, total consumption shock, aggregate consumption shocks for Germany and Italy, for the other countries disaggregated, nonresidential equipment investment shock, nonresidential structures investment shock, residential investment shock, inventory investment shock, fixed investment shock, inventory investment shock, real export shock, real import shock, contract wage shock, cost-push shock, import price shock, export price shock, fiscal policy shock, where we have adjusted the size of the fiscal policy shock for the U.S. - the common fiscal shock - so that a unit shock represents a 1 percent of GNP shock and a monetary policy shock where again the common Modelbase monetary policy shock enters the monetary policy rule for the U.S..
	%%\item Variable dimension: Original variables are expressed in percent/100. The common Modelbase variables are expressed in percent.
	%\item Calibration/Estimation: The model is estimated with single equation methods on G7 data from 1971--1986.
	%\item Replication: We replicated the impulse response functions for annualized quarterly inflation and the output gap to a 100 basis point innovation to the federal funds rate in Figure 2 of \cite{LevinWielandWilliams2003}.
	
	%\item Impulse responses: Figure \ref{img:G7_TAY93}.
	
	%\item Impulse Responses: the first row of figure \ref{img:TAY93} shows impulse responses to a one unit monetary policy shock. The second row shows impulse responses to a one government spending shock of one percent of GDP. The nominal wage contracts imply a purely forward looking Phillips curve. However, in the TAY93 model inflation exhibits persistence comparable to the MSR04 model with real wage contracts. The reasons is that prices are not set as a constant mark-up over marginal cost. Prices follow a backward-looking error-correction specification and can thus be interpreted as a time-varying mark-up over marginal cost. Hence, even though the contract specification of MSR04 and FM95 was not known at the time of the construction of the TAY93 model, the implied Phillips curve is of a hybrid type due to the additional lag structure. The absolute peak response of inflation and output are even higher than in the MSR04 model. Note that this is not caused by explicitly modeling the foreign sector. Even though one would expect imports to increase as the US\$ appreciates this is not the case. Imports are highly elastic to domestic income and decrease. Exports show a very small reaction. Therefore the higher responses compared to FM95 and MSR04 are caused by the structure of the domestic economy.
	%\end{itemize}
	%
	%
	%
	%\subsection{G3\_CW03: Coenen, Wieland (2002, 2003) G3 countries}
	%\label{G3CW03}
	%In this model different kinds of nominal rigidities are considered in order to match inflation and output dynamics in the U.S., the Euro area and Japan. Staggered contracts by \cite{Taylor1980} explain best inflation dynamics in the Euro area and Japan and staggered contracts by \cite{FuhrerMoore1995} explain best U.S. inflation dynamics. The authors evaluate the role of the exchange rate for monetary policy and find little gain from direct policy response to exchange rates.
	%\begin{itemize}
	%\item Aggregate Demand: The open-economy aggregate demand equation relates output to the lagged ex-ante long-term real interest rate and the trade-weighted real exchange rate and additional lags of the output gap. The demand equation is very similar to the G7\_TAY93 model without any sectoral disaggregation. Lagged output terms are supposed to account for habit persistence in consumption as well as adjustment costs and accelerator effects in investment. The lagged interest rate allows for lags in the transmission of monetary policy. The exchange rate influences net exports and thus enters the aggregate demand equation. The long term nominal interest rate is an average of expected future nominal short-term rates. The long-term  real interest rate is determined by the Fisher equation.
	%\item Aggregate Supply: For the U.S., relative real wage staggered contracts by \cite{FuhrerMoore1995} are used (see the US\_FM95 model for a detailed exposition). For the Euro area and Japan the nominal wage contracts by \cite{Taylor1980} are used. Note that Taylor contracts, with a maximum contract length exceeding two quarters, result in Phillips curves that explicitly include lagged inflation and lagged output gaps. Thus, the critique that with Taylor contracts inflation persistence is solely driven by output persistence  \citep{FuhrerMoore1995} is mitigated.
	%\item Foreign sector: All three countries are modeled explicitly. The Modelbase rule replaces monetary policy for the U.S.. For the Euro area and Japan the original interest rules remain. Foreign output does not affect domestic output directly, but indirectly via the exchange rate in the demand equation. The bilateral exchange rates are determined by UIP conditions.
	%%\item Monetary policy rules: the modelbase rule replaces monetary policy for the euro area. For the US and Japan the original interest rules remain: interest rate smoothing parameter $\rho=1$, coefficients on inflation and output are each $0.5$.
	%%\item Microeconomic foundation: No explicit microfoundations. However, the authors discuss the advantage of Taylor contracts over Fuhrer-Moore contracts due to possible microfoundations as derived in \cite{ChariKehoeMcGrattan2000}. The original paper (but not the specification in this modelbase) considers also the Calvo model with profit maximizing firms. The authors neglect microfoundations of the demand side due to the following reasoning:
	%%\begin{quote}
	%%   Although there is an active and rapidly growing literature on closed and open-economy models, which are consistent with optimizing behavior of representative households and firms, these models do not yet seem able to match hump-shaped output dynamics without introducing persistence in unobservables.
	%%\end{quote}
	%\item Shocks: Contract wage shocks, demand shocks and the common monetary policy shock which is added for the U.S..
	%%\item Variable dimension: Original variables are expressed in percent/100. The common Modelbase variables are expressed in percent.
	%\item Calibration/Estimation: Euro area data, (fixed GDP weights at PPP rates from the ECB area-wide model database), U.S. data and Japanese data. For the U.S. and Japan OECD's output gap estimates are used. For the Euro area log-linear trends are used to derive potential output. The estimation is robust to different output gap estimations.
	%Demand block: GMM estimation where lagged values of output, inflation, interest rates, and real exchange rates are used as instruments. Supply side: simulation-based indirect inference methods. Estimation period: U.S. 1980:1--1998:4, Euro area 1980:1--1998:4 and Japan 1980:1--1997:1.
	%%\item Replication: We replicated the impulse response functions to 0.5 percentage points demand shocks in the United States, the Euro Area und Japan plotted in Figure 3 of Coenen, Wieland (2003). Variables include the output gap, annual inflation and the short-term nominal interest rate of the United States, the Euro Area and Japan.
	%
	%%\item Replication: using the original monetary policy rule for all countries, we generated the impulse responses of aggregate demand, the short-term interest rate and annual inflation to a contract wage shock in the euro area and compared it to the ones generated with the original code using the Anderson-Moore algorithm (in the original model there is no monetary policy shock).
	%
	%%\item Impulse responses: Figure \ref{img:G3_CW03}.
	%
	%%\item Impulse responses: figure \ref{img:CW03} shows impulse responses. The first row shows Euro area impulse responses to a one unit monetary policy shock in the Euro area. The second row shows impulse responses to a one unit Euro area demand shock that is added to the IS equation.
	%\end{itemize}
	%
	%\subsection{EACZ\_GEM03: IMF model of Euro Area and Czech Republic, \cite{LaxtonPesenti2003}}
	%\label{EACZGEM03}
	%The model is a variant of the IMF's Global Economy Model (GEM) and consists of a small and a large open economy.
	%The authors study the effectiveness of Taylor rules and inflation-forecast-based rules in stabilizing variability in output and inflation. They check if policy rules designed for large and relatively closed economies can be adopted by small, trade-dependent countries with less developed financial markets and strong movements in productivity and relative prices and destabilizing exposure to volatile capital flows. In contrast to \cite{LaxtonPesenti2003} we focus on the results for the large open economy (Euro area) rather than the small open economy (Czech Republic).
	%\begin{itemize}
	%\item Aggregate Demand: Infinitely lived optimizing households; government spending falls exclusively on nontradable goods, both final and intermediate. Households face a transaction cost if they take a position in the foreign bond market.
	%\item Aggregate Supply: Monopolistic intermediate goods firms produce nontradeable goods and tradable goods. It exists a distribution sector consisting of perfectly competitive firms. They purchase tradable intermediate goods worldwide (at the producer price) and distribute them to firms producing the final good (at the consumer price). Perfectly competitive final good firms (Dixit-Stiglitz aggregator) use nontradable and tradeable goods and imports as inputs. Households are monopolistic suppliers of labor and wage contracts are subject to adjustment costs. Households own domestic firms, nonreproducable resources and the domestic capital stock. Markets for land and capital are competitive. Capital accumulation is subject to adjustment costs. Labor, capital and land are immobile internationally. Households trade a short-term nominal bond, denominated in foreign currency. All firms exhibit local currency pricing, thus exchange rate pass-through is low.
	%%\item Microeconomic foundation: yes
	%\item Shocks: Risk premium shock, productivity shock, shock to the investment depreciation rate, shock to the marginal utility of consumption, government absorption shock where the one affecting the large foreign economy represents the common fiscal policy shock, shock to the marginal disutility of labor, preference shifter. We add the common monetary policy shock to the policy rule of the large economy.
	%%\item Variable dimension: Original variables are expressed in percent/100. The common Modelbase variables are expressed in percent.
	%\item Calibration/Estimation: Calibrated to fit measures of macro-variability of the Euro area (1970:1--2000:4) and Czech Republic (1993:1--2001:4).
	%\item Notes: Due to the symmetric setup of the model, we use the same policy rule in both countries.
	%%\item Replication: We replicated the standard deviations of annual inflation, the output gap and the first difference of the interest rate under the optimal Taylor rule implied by the loss function specification 2 of \cite{LaxtonPesenti2003} as listed in the second row of Table 4 in their paper.
	%%\item Impulse responses: Figure \ref{img:EACZ_GEM03}.
	%
	%%\item Impulse responses: The first row of figure \ref{img:GEM03} shows impulse responses to a one unit monetary policy shock. Using the Taylor rule, there is no persistence in the model, and hence all variables return back to steady state after one period. The second row shows impulse responses to an increase in government spending by one percent of GDP.
	%\end{itemize}
	%
	%
	%\subsection{G2\_SIGMA08: FRB-SIGMA by \cite{ErcegGuerrieriGust2008}}
	%\label{G2SIGMA08}
	%The SIGMA model is a medium-scale, open-economy, DSGE model calibrated for the U.S. economy. \cite{ErcegGuerrieriGust2008} in particular take account of the expenditure composition of U.S. trade and analyse the implications for the reactions of trade to shocks compared to standard model specifications.
	%
	%\begin{itemize}
	%\item Aggregate Demand: There are two types of households: households that maximize a utility function separable in consumption, with external habit formation and a preference shock, leisure and real money balances, subject to an intertemporal budget constraint (forward-looking households) and the remainder that simply consume after-tax disposable income (hand-to-mouth households). Households consume, own the firms and accumulate capital, which they rent to the intermediate goods producers. \cite{ErcegGuerrieriGust2008} introduce investment adjustment costs a la \cite{ChristianoEichenbaumEvans2005}, where it is costly for the households to change the level of gross investment. Households also choose optimal portfolios of financial assets, which include domestic money balances, government bonds, state-contingent domestic bonds and a non-state contingent foreign bond. It is assumed that households in the home country pay an intermediation cost when purchasing foreign bonds, which ensures the stationarity of net foreign assets. Households rent their labor in a monopolistic market to firms, where forward-looking households set their nominal wage in Calvo-style staggered contracts analogous to the price contracts and hand-to-mouth households simply set their wage each period equal to the average wage of the forward-looking households.
	%\item Aggregate Supply: Intermediate-goods producers have an identical CES production function and rent capital and labor from competitive factor markets. They sell their goods to final goods producers under monopolistic competition and set prices in Calvo-style staggered contracts. Firms, who don't get a signal to optimize their price in the current period, mechanically adjust their price based on lagged aggregate inflation. Final good producers in the domestic and foreign market assemble the domestic and foreign intermediate goods into a single composite good by a CES production function of the Dixit-Stiglitz form and sell the final good to households in their country. \cite{ErcegGuerrieriGust2008} introduce quadratic import adjustment costs into the final goods aggregator, which are zero in steady state. It is costly for a firm to change its share of imports in a final good relative to their lagged aggregate shares. Thus the import share of consumption or investment goods is relatively unresponsive in the short-run to changes in the relative price of imported goods even while allowing the level of imports to jump costlessly in response to changes in overall consumption or investment demand. Government purchases are assumed to be a constant fraction of output. Government revenue consists of income from capital taxes (net of the depreciation write off), seignorage income and revenue from lump-sum taxes (net of transfers). The government issues bonds to finance the difference between government revenue and expenditure. Lump-sum taxes are adjusted both in response to deviations of the government debt/GDP ratio from a target level and to the change in that ratio.
	%
	%\item Foreign sector: Local currency pricing is assumed. Intermediate goods producers price their product separately in the home and foreign market leading to an incomplete exchange rate pass-through. \cite{ErcegGuerrieriGust2008} point out, that empirically imports and exports in the U.S. are heavily concentrated, with about 75 percent in capital goods and consumer durables, but the production share of capital goods and consumer durables is very low. To account for this fact in the two-country model they allow the import share in the final good aggregator for investment goods to be higher than the import share in the final good aggregator for consumption goods.
	%\item Shocks: Since we have no information about the variances of the shock terms, we set all shock variances equal to zero. The government spending shock of the home country represents the common fiscal policy shock. The common monetary policy shock is added for the home country.
	%%\item Variable dimension: percent
	%\item Calibration/Estimation: The model is calibrated at a quarterly frequency. Parameters of the original monetary policy rule are estimated using U.S. data from 1983:1--2003:4.
	%\item Replication: We replicated the impulse response functions for real exports, real imports and the exchange rate to a foreign investment demand sock represented by a decline in the foreign capital income tax rate as plotted in Figure 3 (disaggregated trade case) of \cite{ErcegGuerrieriGust2008}.
	%\item Replication: We replicated the impulse responses of real GDP, real investment, real consumption, real exchange rate, real imports and the trade balance after a technology shock, that boosts real GDP by 1 percent in the long-run, using the setup with disaggregated trade.
	
	%\item Impulse responses: Figure \ref{img:G2_SIGMA08}.
	
	%\item Impulse responses: The first row of figure \ref{img:Sigma07} shows impulse responses to a one unit monetary policy shock. Using the Taylor rule, there is no persistence in the model, and hence all variables return back to steady state after one period. The second row shows impulse responses to an increase in government spending by one percent of GDP.
	%\end{itemize}
	%
	%\subsection{EAUS\_NAWM08: \cite{CoenenMcAdamStraub2008}}
	%\label{EAUSNAWM08}
	%\cite{CoenenMcAdamStraub2008} use a calibrated, two-country version of the New Area-Wide Model developed at the European Central Bank to examine the Euro Area tax structure and the potential benefits and spillovers of a tax reform (reducing labor market distortions). The real effects of fiscal policies are analyzed in an environment with heterogeneous households. Countries in \cite{CoenenMcAdamStraub2008} are symmetric but of different size where the U.S. represents the rest of the world.
	%
	%\begin{itemize}
	%
	%\item Aggregate Demand: Only a share of households have access to domestic and international financial markets, accumulates capital and holds money. The other part of households do not have access to financial markets and neither holds capital. They smooth consumption solely by adjusting their money holdings. Both types of households maximize a lifetime utility function with external habit in consumption and supply differentiated labor services with monopoly power in wage setting. Wages are determined in a la \cite{Calvo1983} fashion. Households that receive permission to re-optimize their wages choose the same wage while the other part follows an indexation scheme, with wages being a geometric average of past changes in the price of the consumption good. Households gross income is subject to a rich taxation structure. They pay taxes on consumption purchases, on wage income, on rental capital income and on dividend income. Furthermore, they pay social security contributions, a lump-sum tax and receive transfers. Purchases of consumption, financial investment in international markets and capital utilization are subject to specific proportional costs.
	%
	%\item Aggregate Supply: Producers are distinguished between producing tradable and non-tradable goods. The intermediate goods firm produces a single, tradable differentiated good using an increasing-returns-to-scale Cobb-Douglas technology with capital services and labor as inputs. These goods are sold both in domestic and foreign market under monopolistic competition. Price setting is subject to staggered price contracts a la \cite{Calvo1983}. Firms that receive permission to re-optimize their prices choose the same price (be it for the domestic or for the foreign market) while the other firms follow an indexation scheme, with prices being a geometric average of past changes in the aggregate price indexes. The final goods firms produce three non-tradable final goods: private consumption goods, investment goods and public consumption goods. Final non-tradable private consumption and private investment goods are modeled in an analogous manner. These final goods are assembled with CES technology, combining intermediate domestic and imported foreign goods. Varying the use of imported intermediate goods in the production process is subject to adjustment costs, therefore changes in the relative price of imported goods go unreflected in the short-run. These final goods are sold taking the price as given. On the other side, the public consumption good is a composite of only domestically produced intermediate goods.
	%
	%\item The Foreign Sector: The demand for imported goods is equal to the sum of the respective demands for intermediate goods for private consumption and investment. These intermediate goods are sold in the home market by the foreign intermediate-good producer. The price of the intermediate good imported from abroad is equal to the price charged by the foreign producer (local currency pricing).
	%
	%\item Shocks: A government spending shock, a transfer shock, a productivity shock, a monetary policy shock. (Distortionary tax rates on consumption, on dividends, on rental capital income, on labor income and payments on social security contributions are given as exogenous processes but constant).
	%
	%\item Calibration/Estimation: The model is calibrated to the \cite{SmetsWouters2003} model, with steady-state ratios based on observed data for the euro area and U.S., respectively.
	%
	%%\item Replication: All impulse responses to different fiscal policy shocks, as appearing in \cite{CoenenMcAdamStraub2008}, have been replicated.
	%
	%\end{itemize}
	%
	%
	%\subsection{EAES\_RA09: \cite{Rabanal2009}}
	%\label{EAESRA09}
	%\cite{Rabanal2009} uses a two-country, two-sector DSGE model of a currency union with nominal rigidities to study the sources of persistent inflation differentials between the EMU and one of its member countries, Spain. Moreover, the paper aims at explaining the first moments of the data by introducing time trends for the country- and sector-specific technology shock processes that can give rise to permanent inflation differentials in the model.
	%
	%\begin{itemize}
	%\item Aggregate Demand: Households in Spain and in the rest of EMU have utility functions separable in consumption and leisure and displaying external habit formation in consumption.  The composite consumption good is defined as a CES aggregate consisting of domestic tradable and nontradable, and foreign tradable goods. Preferences are assumed to be the same across countries, but countries differ with respect to the composition of their consumption basket.
	%
	%\item Aggregate Supply: Each economy is characterized by two sectors. Monopolistic intermediate firms use labor, supplied by the households, as the only input to produce tradable and nontradable goods. They set prices to maximize profits subject to a set of demand equations. Price setting follows a modified version of the Calvo framework with two indexation mechanisms in place that account for the fact that steady state inflation might be non-zero. Across countries the same production technologies are deployed but countries differ in the degree of wage and price stickiness and in the degree of indexation.
	%
	%\item Foreign sector: \cite{Rabanal2009} models two countries in the European monetary union of unequal size. They produce differentiated tradable goods that are imperfect substitutes of each other, but there is no price discrimination for the same type of good across countries.
	%
	%\item Shocks: Ten shocks are introduced in the model: sector- and country-specific AR(1) shock processes for the government spending and the technology shock with an Euro Area tradable shock component, and an iid monetary policy shock.
	%
	%\item Calibration/Estimation: The model is estimated using Bayesian estimation techniques using quarterly euro area data for the period 1996:Q1--2007:Q4.
	%
	%%\item Note: Due to sector- and country-specific government spending shock, we define the fiscal shock to be the government spending shock in the nontradable sector in the foreign country, here equivalent to the rest of the EMU.
	%
	%%\item Replication: All impulse response functions for headline and nontradable inflation and the real GDP growth, as appearing in Figure 5 of \cite{Rabanal2009}, have been replicated.
	%\end{itemize}
	%
	%\subsection{EAUS\_NAWM08CTWW13: \cite{CoganTaylorWielandWolters2013}}
	%\label{EAUSNAWM08CTWW13}
	%
	%\cite{CoganTaylorWielandWolters2013} use a version of the EAUS\_NAWM08 model of \cite{CoenenMcAdamStraub2008} to study the fiscal consolidation plan on the U.S.economy. In EAUS\_NAWM08CTWW13, the US economy is calibrated following \cite{CoganCwikTaylorWieland2010}.
	%
	%\begin{itemize}
	%
	%\item Aggregate Demand: As in EAUS\_NAWM08.
	%
	%\item Aggregate Supply: As in EAUS\_NAWM08.
	%
	%\item The Foreign Sector: As in EAUS\_NAWM08.
	%
	%\item Shocks: As in EAUS\_NAWM08.
	%
	%\item Calibration/Estimation: Differently from the EAUS\_NAWM08 model, parameters for the US are calibrated with reference to other estimated models, including the \cite{CoganCwikTaylorWieland2010}.
	%
	%%\item Replication: All impulse responses to different fiscal policy shocks, as appearing in \cite{CoenenMcAdamStraub2008}, have been replicated.
	%
	%\end{itemize}
	%\subsection{GPM6\_IMF13: \cite{Carabenciovetal2013}}
	%\label{GPM6IMF13}
	%\cite{Carabenciovetal2013} construct and estimate a six region model with both financial and real linkages. The study is the sixth of a series the IMF research agenda in developing a Small Quarterly Global Projection Model (GMP) which consists of small country models integrated into a single global market. The six regions represent the US, the euro area (EA), Japan, Emerging Asia, a five-country block of inflation-targeting Latin American countries and a “remaining-countries” group. The three first regions are regrouped under the label G3 and differ from the rest in five ways: (i) they have an unemployment sector; (ii) there is a trend of appreciation of the real exchange rate for the emerging economies; (iii) there is no bank lending tightening variable for non-G3 economies; (iv) G3 economies are assumed to have achieved their inflation-targets; and (v) priors for estimations differ between the two groups. In addition, the model includes financial spillovers not only from the US but from the EA and Japan as well, a global demand shock, a medium-term interest rate, and real exchange rate linkages. Each of the six economies is characterized by a few behavioral equations. 
	%\begin{itemize}
	%\item Aggregate Demand: The behavioral IS curve relates the output gap to domestic, external and financial-real linkages. The domestic effects consist of past and expected future values of the output gap and of the past value of the medium-term real interest rate. The specification allows for inertia in the system, with complex forward looking elements. The medium-term real interest rate provides the transmission channel between monetary policy action and the real economy. The external effects are driven by the effective real exchange rate gaps and the foreign demand channel. An overvaluation of the currency, i.e. a negative exchange rate gap, has a negative impact on the output gap. Foreign demand captures the spillovers from trade and allows for a direct and a global impact of the foreign output gaps separately. The financial-real linkages capture the bank lending conditions originating from the G3 economies and for each of these countries, tighter lending conditions translate into a negative output gap.  
	%\item Aggregate Supply: The Phillips curve expresses inflation as a function of its past and its future value, the lagged output gap, the change in the effective real exchange rate gap of the country and a disturbance term. Backward-looking elements represent the direct and indirect indexation to past inflation, as well as the decision made by price setters who base their expectations on past rates of inflation. The forward-looking element captures the proportion of price setters who have model-consistent inflation expectation. The real effective exchange rate gap is the import weighted real exchange rate gaps of the trading partners, as import prices capture the pass-through from exchange rate movements to the CPI the best.
	%
	%\item The Policy Rule: is an Inflation-Forecast-Based rule that determines the short-term nominal rate for the G3 countries. It reacts to three quarters ahead inflation following \cite{Orphanides20032}, the real interest rate and the domestic output gap.
	%\item The Medium-term Real Interest Rate: is a model-specific variable that enters the Aggregate Demand equation.  It is a function of the current real policy rate, the expected average real policy rate over the coming year, the expected average real policy rate over the next three years, and the expected average real policy rate over the next five years.
	%
	%\item The Uncovered Interest Parity: serves to link the country models beyond the aggregate demand block. It is augmented for the emerging country models to include a trend component to capture real appreciation of their currencies.
	%\item  Unemployment rate: The model specifies a dynamic version of Okun's law for the G3 regions. It links the unemployment rate to its lagged value and the contemporaneous output gap.
	%
	%\item Shocks: Shocks to aggregate demand, to the bank lending conditions, to inflation, to the short run rate, to the uncovered interest parity and to the unemployment rate enter the G3 models. The model specifies more stochastic shocks than observables to prevent the model from generating systematic forecast errors over extended periods. Thus, the model features a shock to the level and the growth rate of potential output, a shock to the level and the growth rate of the equilibrium rate of unemployment, a shock to the equilibrium real interest rate and finally a shock to the equilibrium real exchange rate in each economy.
	%
	%\item Calibration/Estimation: The model is estimated with Bayesian techniques. The estimated coefficients of \cite{Carabenciovetal20082} are taken as a starting point, i.e. as given, then each emerging country region is added individually to estimate the region's seven coefficient: the forward and backward looking components on inflation and the coefficient of the output gap in the Phillips Curve; the lagged and forward looking coefficients of output in the Aggregate Demand equation; and all three coefficients in the Policy Rule. The rest of the coefficients are calibrated. The resulting six country regions are put together to estimate three parameters: the coefficient on the spillover activity variable, the coefficient on the financial spillover variable for emerging economies in the output gap equations and finally, the coefficient on the real exchange rate gap terms in the Phillips Curve equation.
	%The model uses quarterly data over the period 1999:Q1 to 2010:Q2.
	%
	%\end{itemize}
	%\section{Estimated Models of Other Countries}
	%\subsection{CL\_MS07: \cite{MedinaSoto2007}}
	%\label{CLMS07}
	%\cite{MedinaSoto2007} develop a small-open economy DSGE model for the Chilean economy. The CL\_MS07 is structurally similar to models developed by \cite{ChristianoEichenbaumEvans2005}, \cite{AltigChristianoEichenbaumLinde2005}, and \cite{SmetsWouters2007}. Still, a richer specification for the production sector and for fiscal policy is designed to account for special characteristics of the Chilean economy.
	%
	%\begin{itemize}
	%\item Aggregate Demand: There are two types of households, Ricardian and non-Ricardian households. The Ricardian type households maximize a utility function separable in consumption, leisure and real money balances subject to their intertemporal budget constraint. They have access to three types of assets, namely money and one-period non-contingent foreign and domestic bonds. Each of these households is a monopolistic supplier of differentiated labour and only a fraction of them can re-optimize their nominal wage. Rigidity a la Calvo in wage setting follows \cite{ErcegHendersonLevin2000}. Households that cannot re-optimize their wages follow an updating rule considering a geometric weighted average of past CPI inflation and the inflation target. On the other side, the non-Ricardian households do not have access to any of the assets and own no shares in domestic firms. They simply consume the after-tax disposable income and set their wage equal to the average wage of the Ricardian households. The aggregate consumption for both types of households is a composite of a core consumption bundle (domestic and foreign goods, given by a CES aggregator) and oil consumption.
	%
	%\item Aggregate Supply: The economy is characterized by three types of firms: intermediate tradable-goods producers, import goods retailers and commodity good producers. Intermediate-goods producers have monopoly power and maximize profits by choosing the prices of their differentiated goods subject to the corresponding demands, and the available technology with labor, capital and oil as inputs. Capital is rented to them from a representative firm which accumulates capital and assembles new capital goods subject to investment adjustment costs. Optimal price setting of intermediate-goods producers is subject to a Calvo probability. Firms that cannot re-optimize their price follow a rule with partial indexation to past inflation and the inflation target. The pricing structure leads to a hybrid New Keynesian Phillips curve. A commodity good producer is introduced in the model to match a particular relevant sector for the Chilean economy, namely the cooper sector. This firm produces a homogeneous commodity good only for export. The production technology follows an exogenous stochastic process that does not require any input. The price of the homogeneous commodity good is determined in the foreign market.
	%
	%\item Foreign sector: Local currency pricing is introduced through a la Calvo price stickiness faced by import goods retailers, which resale foreign goods in the domestic market. This allows for incomplete exchange rate pass-through in the short-run, important for expenditure-switching effects of the exchange rate. A CES technology is used to combine a continuum of differentiated imported varieties to produce a final foreign good, which is consumed by households and used for assembling new capital goods.
	%
	%\item Shocks: a transitory productivity shock, a permanent productivity shock, a commodity production shock, a labor supply shock, an investment adjustment cost shock, a preference shock, a government expenditure shock, a monetary policy shock, a foreign commodity price shock, a foreign oil price shock, a foreign output shock, a foreign interest shock, a foreign inflation shock and a price of imports shock.
	%
	%\item Calibration/Estimation: The model is estimated using Chilean quarterly data for the period 1987:1--2005:4.
	
	%\item Replication: We replicated most of the impulse response functions shown in \cite{MedinaSoto2007}, however only when using the original code from the authors, which presents an extended version from the model given in the paper. Among some of these differences we can mention the presence of capital utilization, not discussed in the paper.
	%
	%\end{itemize}
	%
	%\subsection{CA\_ToTEM10: \cite{MurchisonRennison2006}}
	%\label{CAToTEM10}
	%CA\_ToTEM10 represents the 2010 vintage of ToTEM (Terms-of-Trade Economic Model) which is an open-economy, DSGE model developed by \cite{MurchisonRennison2006}. The Bank of Canada uses this model as a tool for policy analysis and projections for the Canadian economy. %CA\_ToTEM10 takes this role from the previously existing Quarterly Projection Model.
	%
	%\begin{itemize}
	%
	%\item Aggregate Demand: Households are classified as ``lifetime income'' consumers and ``current income'' consumers, reflecting the fact that not all consumers can access credit markets. Lifetime income consumers smooth their consumption across time through borrowing and saving while ``current income'' consumers consume their current income each period. Lifetime income consumers choose consumption, domestic and foreign bond holdings, labor supply and wages to maximize a utility function non-separable in consumption and leisure subject to a dynamic budget constraint. Both types of households supply differentiated labor services giving them power when negotiating the wages with the domestic producers. However, renegotiation of the wages is allowed only once in six months, on average, and only a constant proportion of wage contracts are renewed every period. The dynamic wage equation is a function of past and expected future wage inflation and an error-correction component.
	%\item Aggregate Supply: The production sector is comprised of final good producers, an import sector and a commodity sector. Final goods firms produce consumption goods and services, investment goods, and export goods. The production process of these goods is analogous, differing only on the share of imported goods used in production. In this process, first a capital-labor composite is produced using CES technology, which is then combined with a commodity input to produce the domestic good. Final goods then are a combination of the domestic good and the imported good. Through these steps, the firm faces capital adjustment costs, investment adjustment costs and labor adjustment costs. Final goods firms sell their differentiated goods in a monopolistic competitive fashion having power over prices. However, not all firms can re-optimize their prices every period. A share of firms updates prices according to a geometric average of lagged core inflation and expectations of the inflation target. In ToTEM, pricing decisions are considered as strategic complements, where firms have a strong incentive to follow what other firms do. The commodity sector is represented by a domestic firm operating in a competitive market, producing commodities using capital services, labor and land under a CES technology. These raw goods are either sold to a continuum of imperfectly competitive commodity distributors or exported (for the world price of the commodity denominated in Canadian currency). The commodity distributors repackage the commodity goods and sell them to households and to the final goods producers. These distributors face nominal rigidities a la Calvo in price setting, which limits the degree of exchange rate pass-through to consumer prices in the short-run.
	%
	%\item The Foreign Sector: The import sector is represented by firms who buy imported goods in the world market for a given world price (law of one price holds). These goods are sold to domestic firms, which use them as inputs in their respective production functions. Imperfect exchange rate pass-through in the short-run is present as the price of imports is temporarily fixed in the currency of the importing country and because import firms face nominal rigidities a la Calvo when setting prices. As in other sectors, imported goods inflation is a function of past and expected future imported goods inflation and an error-correction component. Export goods firms are part of the final good producers sector as discussed above. They have some degree of market power and therefore face a downward-sloped demand curve (rest of the world demand).
	%
	%\item Shocks: A demand shock, a risk-premium shock, an inflation target shock, a commodity price shock, a technology shock, world demand shock and a price mark-up shock.
	%
	%\item Calibration/Estimation: Calibrated with parametrization chosen to match univariate autocorrelations, bivariate correlations and variances estimated using Canadian data for the period 1980--2004.
	%
	%%\item Replication: To replicate the model, the 2010 vintage of ToTEM is used. Impulse responses of a consumption demand shock, risk premium shock, inflation target shock and price mark up shock are replicated successfully as in \cite{MurchisonRennison2006}.
	%
	%\end{itemize}
	%
	%\subsection{BRA\_SAMBA08: \cite{Gouveaetal2008}}
	%\label{BRASAMBA08}
	%\cite{Gouveaetal2008} build and estimate a small open economy model for the Brazilian economy. The BRA\_SAMBA08 model is developed at the Central Bank of Brazil to provide support for its policy decisions. This version of the model is used as a tool to analyze the response of the Brazilian economy when subject to different shocks.
	%
	%\begin{itemize}
	%
	%\item Aggregate Demand: There are two types of households: optimizers and rule-of-thumbers. Both maximize a similar utility function separable in consumption and leisure but subject to different budget constraints. Unlike the optimizers, the rule-of-thumb households do not have access to credit, asset and capital markets. They just consume their wage income. The optimizers have access to domestic and foreign currency denominated bonds, accumulate capital subject to capital adjustment costs, earn from renting the capital and pay taxes. On the other hand, both types of households supply labor in a competitive market.
	%
	%\item Aggregate Supply: The production sector is comprised of producers and assemblers. Monopolistic competitive firms are the ones producing differentiated goods under a Cobb-Douglas technology with labor, capital services and imported goods as inputs. Following \cite{GaliGertler1999}, only a fraction of firms are allowed to adjust prices optimally ("forward-looking firms"). The remaining firms follow a rule of thumb. The homogeneous final good is assembled by a representative firm using a CES aggregator and is sold in a competitive market. The final good can be used for private consumption, government consumption, investment and exports.
	%
	%\item The Foreign Sector: The world is assumed to be populated by a continuum of small open economies as in \cite{GaliMonacelli2005}, each of them producing a differentiated good in the global market. The demand for home country's exports is obtained from the aggregation of the demands from foreign countries, expressed in a world currency. The domestic importing firm takes the demand for its goods from the producers' input choices.
	%
	%\item Shocks: An inflation target shock, a fiscal target shock, a preference shock, a labor supply shock, an investment shock, a foreign investor's risk aversion shock, a country risk premium shock, a technology shock, a monetary policy shock, a fiscal policy shock, a world imports shock, a world inflation shock and a world interest rate shock.
	%
	%\item Calibration/Estimation: Estimated with Bayesian methods, using quarterly Brazilian data for the period 1999:Q2--2007:Q4.
	%
	%%\item Replication: (As far as I remember replication was successful).
	%
	%
	%\end{itemize}
	%
	%\subsection{CA\_LS07: \cite{LubikSchorfheide2007}}
	%\label{CALS07}
	%\cite{LubikSchorfheide2007} estimate four small-scale open economy DSGE models with Bayesian techniques for Canada, Australia, New Zealand and the UK. The paper studies to what extent central banks respond to exchange rate movements when setting nominal interest rates, finding that the Bank of Canada and the Bank of England do include the nominal exchange rate in their policy rule. The database contains the model for Canada.
	%
	%\begin{itemize}
	%
	%\item Aggregate Demand: The model treats the world economy as a continuum of small open economies. The representative household maximizes its utility separable between consumption and leisure subject to its budget constraint. Consumption is a composite of tradable home and foreign goods.
	%
	%\item Aggregate Supply: Differentiated goods are produced by monopolistic-competitive firms using a linear technology with labor being the only production input. The firms set their prices in a Calvo staggered way. The marginal costs depend positively on the terms of trade and world output.
	%
	%\item The Foreign Sector: Purchasing power parity and the law of one price hold. There is perfect exchange rate pass-through.  The securities markets are assumed to be complete, and hence international risk sharing in the form of the uncovered interest rate parity is obtained.
	%
	%\item Shocks: A nominal interest rate shock, a terms of trade shock, a shock to world demand and a shock to the world inflation rate are introduced in the model.
	%
	%\item Calibration/Estimation: The model is estimated with Bayesian methods using quarterly Canadian data for the period 1983:Q1--2002:Q4.
	%
	%%\item Replication: The impulse responses in Figure 1 in \cite{LubikSchorfheide2007} are Bayesian. The IRFs for a monetary, a terms of trade, a row output and a row inflation shock have been qualitatively replicated.
	%
	%\end{itemize}
	%
	%\subsection{HK\_FPP11: \cite{FunkePaetzPytlarczyk2011}}
	%\label{HKFPP11}
	%\cite{FunkePaetzPytlarczyk2011} develop a small open economy DSGE model and estimate it for Hong Kong with Bayesian techniques. The model adopts the perpetual youth approach and allows for wealth effects from the stock market on consumption
	%behavior.
	%
	%\begin{itemize}
	%
	%\item Aggregate Demand:  The economy consists of an indefinite number of cohorts facing a constant probability of dying each period, which implies a constant expected effective decision horizon of consumers. Given the lifetime uncertainty, agents' consumption pattern is affected by their expected lifetime wealth (in terms of the wealth in stock market), where the stock price is modeled as the discounted sum of future dividends. In this open economy the consumers are free to allocate their consumption between domestic goods and foreign goods, and the intertemporal allocation is characterized by an otherwise conventional Euler equation that captures the impact of stock-price dynamics.
	%
	%\item Aggregate Supply: Domestic firms act under monopolistic competition and produce consumption goods. Nominal frictions are introduced in the form of Calvo sticky prices. Non-reoptimizing firms index their prices to previous period's domestic producer price inflation.
	%
	%\item The Foreign Sector: The rest of the world is modeled exogenously. Foreign output affects domestic output through international risk sharing directly, and also indirectly via the terms of trade channel.
	%
	%\item Shocks: A productivity shock, a foreign demand shock, a cost push shock and a stock-price gap shock.
	%
	%\item Calibration/Estimation: The model is estimated using Bayesian methods. \cite{FunkePaetzPytlarczyk2011} employ quarterly data on four observables for the sample 1981:Q1--2007:Q3: the real GDP of Hong Kong, the Hang Seng index, the consumer price index of Hong Kong and US GDP. The last series is used as a proxy for foreign demand.
	%
	%%\item Replication: We replicated the impulse responses to a positive technology shock, a foreign demand shock and a cost-push shock in Figure 5-7 of \cite{FunkePaetzPytlarczyk2011}.
	%
	%\end{itemize}
	%
	%\subsection{HK\_FP13: \cite{FunkePaetz2013}}
	%\label{HKFP13}
	%\cite{FunkePaetz2013} develop a two-agent, two-sector, open-economy DSGE model and estimate it for Hong Kong with Bayesian methods. The model introduces credit market frictions as a form of a binding collateral constraint on borrowers and adopts a fixed exchange-rate regime as monetary policy.
	%
	%
	%\begin{itemize}
	%\item Aggregate Demand: Households consists of borrowers and savers. They both obtain utility from consuming non-housing goods and housing and disutility from providing labor. There is habit formation in consumption, both non-housing goods and housing are CES indices of domestically-produced goods and foreign-produced ones. Borrowers are not able to access to the international financial markets and face the collateral constraint linking to the value of housing and the loan-to-value ratio. Savers can purchase both domestic bonds foreign bonds. A symmetric steady state and perfect international risk/sharing are assumed.
	%
	%\item Aggregate Supply: Each sector has a two-stage structure of production. Perfectly competitive retailers produce final goods by aggregating intermediate goods according to a CES technology, and monopolistically competitive frims produce intermediate goods subject to nominal rigidity a la Calvo.
	%
	%\item The Foreign Sector: The rest of the world is modeled exogenously. Foreign output affects domestic output through international risk sharing directly, and also indirectly via the terms of trade channel.
	%
	%\item Shocks: Sector-specific productivity shocks, housing preference shocks, a loan-to-value shock, a government expenditure shock, sectoral cost push shocks, a foreign consumption shock, a foreign housing shock and shocks on foreign price distortions.
	%
	%\item Calibration/Estimation: The model is estimated with Bayesian methods using quarterly data for seven macroeconomic variables ranging from 1981:Q1 to 2007:Q3.
	%
	%\end{itemize}
	%\subsection{CA\_BMZ12: \cite{Bailliuetal2012}}
	%\label{CABMZ12} 
	%\cite{Bailliuetal2012} investigate interactions between monetary and macroprudential policy and examine whether policy makers should respond to financial imbalances. The model is a closed economy that accounts for standard New Keynesian features and has a financial friction along the lines of \cite{BernankeGertlerGilchrist1999} and \cite{ChristensenDib2008}. The model is estimated using Canadian data. The authors show that it is welfare improving to react to financial imbalances. The size of the benefits, however, depends on the nature of the shock.
	%\begin{itemize}
	%\item Aggregate Demand: The representative household derives utility from consumption and disutility from labor. Accordingly, it maximizes utility subject to its resource constraint. The household purchases consumption goods and a one-period government bond. The household's income consists of labor income, bond payoff and dividends on the equity it owns on retailer firms. 
	%\item Aggregate Supply: Entrepreneurs, capital producers and retailers operate in the production sector of the economy. Entrepreneurs borrow from lenders to purchase capital from capital producers and produce intermediate goods. Capital producers combine investment goods and existing capital to produce new capital, subject to quadratic capital adjustment costs.  Retailers operate in a monopolistically competitive environment and are subject to price rigidities à la Calvo (1983). They buy intermediate goods from entrepreneurs and differentiate them at no cost. A Dixit-Stiglitz aggregator combines intermediate goods to form the final good. 
	%\item Financial Sector and Macroprudential Policy: The financial friction is modeled along the lines of \cite{BernankeGertlerGilchrist1999}, i.e. there is a costly state verification contract between the entrepreneurs (borrowers) and the lenders. In this model, the contract is set in nominal terms, similarly to \cite{ChristensenDib2008}. The contract implies a negative relationship between borrower's net worth and the funding costs (external finance premium). The macroprudential policy tool is modeled as an exogenous component of the external finance premium. 
	%\item Shocks: a financial shock (affecting the external finance premium), a technology shock, a monetary policy shock, a preference shock, and an investment-specific shock.
	%\item Calibration/Estimation: The model is estimated by Bayesian techniques on quarterly Canadian data for the sample 1997:Q1- 2009:Q3. The observable time series are: output (excluding government expenditures), investment, the nominal interest rate, inflation and the external finance cost.
	%\end{itemize} 
	
	\subsection{EA\_GE10: \cite{Gelain2010}}
	\label{EAGE10}
	The model of \cite{Gelain2010} incorporates financial frictions \`{a} la \cite{BernankeGertlerGilchrist1999} into a New Keynesian DSGE model which closely follows the structure of the model developed in \cite{SmetsWouters2003}. The structural model allows for a dynamic analysis of the external finance premium. The paper shows that the estimated premium is not necessarily countercyclical as suggested by former studies on the Euro Area external finance premium. In the presence of certain shocks the premium responds procyclically.
	
	\begin{itemize}
		\item Aggregate Demand: A representative household maximizes its intertemporal utility function choosing the level of consumption, hours worked and the amount of bank deposits, subject to a budget constraint. The household's consumption preferences exhibit habit formation.
		
		\item Aggregate Supply: Each household is a monopolistic supplier of differentiated labor services requested by the domestic firms. After setting their wages in a Calvo staggered way, households inelastically supply the firms' demand for labor at the ongoing wage rate. An indexation rule is assumed for those households who are not allowed to re-optimize. \\
		The production sector consists of three types of firms: entrepreneurs, capital producers and retailers. Entrepreneurs hire labor from households and buy capital from capital producers to produce intermediate goods using a Cobb-Douglas production technology. Entrepreneurs have a finite expected lifetime horizon. The capital purchases are financed partly by the entrepreneur's net worth and partly by borrowing from a financial intermediary. The presence of asymmetric information between entrepreneurs and lenders creates a financial friction as in \cite{BernankeGertlerGilchrist1999}. Entrepreneurs can reoptimize their prices only from time to time, as in \cite{Calvo1983}. \\
		Capital producers buy final goods to produce capital subject to investment adjustment costs. Retailers operate in a perfectly competitive market, they use a Dixit-Stiglitz technology using the entrepreneurs' intermediate goods as inputs.
		
		\item Shocks: The model exhibits eight shocks. Two preference shocks, a shock to investment adjustment costs, a technology shock in entrepreneurs' production function, a wage and a price mark up shock, a government spending shock and a monetary policy shock.
		
		\item The model is estimated using Bayesian techniques on quarterly Euro Area data for 1980:Q1 to 2008:Q3.
		The data set used is comprised of seven key macroeconomic variables aggregated for the Euro Area consisting of real GDP, real consumption, real gross investment, hours worked, the nominal short term interest rate, real wages per head and inflation rate.
		
		%\item Replication: We replicated the impulse responses to a one standard deviation orthogonalized monetary policy, technology, an investment specific and a labor supply shock as in Figures 5-8 on pages 64-67 in \cite{Gelain2010}.
	\end{itemize}
	
	%\subsection{EA\_QR14: \cite{QR2014}}
	%\label{EARQ14}
	%
	%\cite{QR2014} builds a two-country, two-sector, two-agent general equilibrium model of a single currency area. The two countries, referred to as ”home” and ”foreign” (or alternatively, ”core” and ”periphery”), use the same currency and are subject
	%to the same monetary policy, which targets union-wide inflation. In each country there are two type
	%of goods: durable goods, taken to represent housing, which cannot be traded across countries, and
	%non-durable goods which can be traded. Production is undertaken in both countries by two types of
	%producers: final good producers operating in a perfectly competitive market and intermediate goods
	%producers operating under monopolistic competition with price setting `a la Calvo.
	%
	%Each country presents borrowers and savers, which are distinguished by their different discount factor and guarantee
	%that there is a credit market both within and between countries. Finally, there are two types of financial
	%intermediaries in the model, domestic and international. The former take deposits from savers and
	%issue bonds that can be traded across countries by the latter. Domestic intermediaries pay the deposit
	%interest rate on the liabilities they issue. On the asset side, they lend to borrowers at the lending rate.
	%The model features a financial accelerator mechanism on the household side, such that changes in the
	%balance sheet of borrowers due to fluctuations in durables' prices affect the spread between lending
	%and deposit rates.
	%
	%The model features 13 exogenous shocks, preference shocks, sectoral technology shocks, financial shocks, non-stationary union-wide technology shock and a monetary policy shock.
	
	\subsection{EA\_GNSS10: \cite{Geralietal2010}}
	\label{EAGNSS10}
	
	The model of \cite{Geralietal2010} incorporates an imperfect competitive banking sector in a DSGE model with financial frictions \`{a} la \cite{Iacoviello2005}. The model allows to asses the role of both financial frictions and banking intermediation in shaping business-cycle dynamics.
	
	\begin{itemize}
		\item Aggregate Demand: There are two type of households, patient (savers) and impatient (borrowers) and entrepreneurs. A representative household maximizes its intertemporal utility function choosing the level of consumption, hours worked and housing services, subject to a budget constraint. Impatient households face in addition a borrowing constraint, linked to the expected value of their collateralizable housing stock. The household's consumption preferences exhibit external habit formation.
		Households supply differentiated labor services through unions. Wage-setting is subject to adjustment costs, \`{a} la Rotemberg with indexation to both past and steady state inflation. Entrepreneurs maximize their utility by choosing consumption, physical capital, loans from banks, the degree of capacity utilization and labor. Entrepreneurs face also a borrowing constraint, linked to the value of their holdings of physical capital.
		
		\item Aggregate Supply: The production sector consists of three types of firms: entrepreneurs, capital good producers and retailers. Entrepreneurs hire labor from households and buy capital from capital good producers to produce intermediate goods. Capital producers buy final goods to produce capital subject to investment adjustment costs. Retailers buy intermediate goods from entrepreneurs and differentiate them. Pricing is subject to nominal rigidities.
		
		\item Banking Sector: Banks enjoy market power in conducting their intermediation activity. Bank loans should be met by deposits and/or bank capital. Each bank has three parts, two ``retail'' branches (giving out differentiated loans to impatient households and to entrepreneurs and raising differentiated deposits from patient households) and one ``wholesale'' unit (managing the capital position of the group).
		
		\item Shocks: The model exhibits a technology shock, price and wage markups shocks, a consumption preferences shock, a housing demand shock, an investment-specific technology shock, a monetary policy shock, shocks to the loan-to-value ratios on loans to firms and households, shocks to the markup on bank interest rates and balance sheet shocks.
		
		\item The model is estimated with Bayesian techniques, for the euro area for 1998:Q1-2009:Q1. For estimation twelve observables are used: real consumption, real investment, real house prices, real deposits, real loans to households and firms, overnight rate, interest rates on deposits, loans to firms and households, wage inflation and consumer price inflation.
	\end{itemize}
	
	\subsection{EA\_PV15: \cite{poutineau2015cross}}
	\label{EAPV15}
	\cite{poutineau2015cross} evaluate quantitatively how interbank and corporate cross-border flows shape business cycles in a monetary union. They estimate a two-country DSGE model (equal-size Eurozone core and peripheral countries) that accounts for national heterogeneities and a set of real, nominal and financial frictions. Each country is populated by consumers, labor unions, intermediate and final producers, entrepreneurs, capital suppliers and a banking system. The set of real rigidities encompasses consumption habits, investment adjustment costs and loan demand habits. Regarding nominal rigidities, they account for stickiness in final goods prices, wages and loan interest rates. Obtained results support the key role of the cross-border channel as an amplifying mechanism in the diffusion of asymmetric shocks.
	\begin{itemize}
		\item Aggregate  Demand:  Households  in  both countries consume, save and work in intermediate firms, and  maximize  expected  lifetime  utility  with respect to the consumption and labor effort.  They spend their incomes on consumption, bond subscription and tax payments. In addition to that, there exist quadratic adjustment costs necessary to buy new bonds (\cite{schmitt2003closing}). Households provide differentiated labor types, sold by labor unions to perfectly competitive labor packers who assemble them in a CES aggregator and sell the homogenous labor to intermediate firms.
		\item Aggregate  Supply:  Each  economy  is  characterized  by  two groups of firms: intermediate firms and final firms. Intermediate firms produce differentiated goods, choose labor and capital inputs, and set prices according to the Calvo model. Final goods producers act as a consumption bundler by combining national intermediate goods to produce the homogenous final good.
		\item Financial Sector:  In each country the banking sector finances investment projects to home and foreign entrepreneurs by supplying one-period loans. The banking system is heterogeneous with regard to liquidity and banks engage in interbank lending at the national and international levels. Thus, cross-border loans are made of corporate loans (between banks and entrepreneurs) and interbank loans. In order to introduce an interbank market, authors suppose that the banking system combines liquid and illiquid banks, where liquid banks (characterized by direct accessibility to the ECB funding) supply loans to entrepreneurs and to illiquid banks. Additionally, the intermediation process between liquid and illiquid banks is costly (convex monitoring technology). So-called financial accelerator mechanism is borrowed from \cite{de2010scientific} and applied in a different context, by assuming that entrepreneurs' forecasts regarding the aggregate profitability of a given project are optimistic (these values are then compared to the critical threshold which distinguishes profitable and non-profitable projects).
		\item Shocks:  There are in total 8 country specific structural shocks and one shock in the common monetary policy rule. Namely, a productivity shock, demand shock, time-preference shock, net wealth shock, external finance premium shock, bank rate markup shock, wage markup shock, bank liability shock and ECB monetary policy shock.
		\item Calibration/Estimation: The model is estimated with Bayesian methods on Euro Area quarterly data over the sample period 1999Q1 to 2013Q3.
	\end{itemize}
	
	
	\subsection{EA\_PV16: \cite{priftis2016portfolio}}
	\label{EAPV16}
	\cite{priftis2016portfolio} analyse the effects of quantitative easing (QE) in a model of the Eurozone with different assets. QE is captured by asset-purchases by the central bank. They use the model to simulate the path of QE as announced in early 2015 and find an expansionary effect on output, interest rates and inflation that is larger when QE is accommodated by low interest rates. The model is the QUEST III model (which is implemented in the MMB) plus QE.
	\begin{itemize}
		\item Aggregate  Demand:  There are two types of households: liquidity- and non-liquidity-constrained households. They possess the same utility function, non-separable in consumption and leisure with habit persistence in both consumption and leisure. Liquidity-constrained households do not optimize, they just consume their labor income. On the other side, non-liquidity-constrained households have access to domestic and foreign currency denominated assets, accumulate capital subject to investment adjustment costs and rent it to firms, earn profits from owning the firms and pay taxes. Income from foreign financial assets is subject to an external financial intermediation risk premium while real asset holdings are subject to an equity risk premium. Both types of households supply differentiated labor to a trade union which sets the wages by maximizing their joint utility (weighted by the share of each type). The wage setting processs is subject to a wage mark-up and to slow adjustments in the real consumption wage. The wage mark-up arises because of wage adjustment costs and the fact that a part of workers index the growth rate of wages to past inflation.
		\item Aggregate  Supply:  The final goods, which are produced from monopolistically competitive firms, are used for household consumption, investment, government consumption and export. These goods are produced with a Cobb-Douglas production function with capital and production workers (labor adjusted for overhead labor) as inputs. These firms face technological and regulatory constraints, restricting their price setting, employment and capacity utilization decisions. The final goods producer maximizes profits subject to these specific adjustment costs (all having convex functional forms) and demand conditions. Investment good producers combine domestic and foreign final goods using a CES aggregator to produce investment goods which
		are sold to non-liquidity-constrained households in a perfectly competitive market.
		\item The Foreign Sector:  Demand behavior is considered the same for the home country and the rest of the world, therefore export demand and import demand are symmetric. Both equations are characterized by a lag structure in relative prices which captures delivery lags. Export firms buy domestic goods, transform them using a linear technology and sell them in the foreign market, charging a mark-up over the domestic prices. The same situation is faced by importer firms. Mark-up fluctuations arise because of price adjustment costs in both sectors. Mark-up equations are given as a function of past and future inflation and are also subject to random shocks.
		\item Assets and QE: QE is modelled as purchases of domestic long-term bonds in exchange for central bank liquidity. Next to physical capital and money, the model features long-term and short-term bonds.
		\item Shocks: The model contains a large battery of shocks as in the QUEST III model as well as QE shocks.
		\item Calibration: The model is calibrated in line with the QUEST III model, which has been estimated on EA data. \end{itemize}
	
	
	\subsection{EA\_PV17: \cite{priftis2017macroecon}}
	\label{EAPV17}
	\cite{priftis2017macroecon} analyse the effects of quantitative easing (QE) in a  model of the Eurozone. QE is captured by long-term bond purchases by the central bank. The authors find that QE mildly raised output growth and inflation. The model is the QUEST III model (which is implemented in the MMB) plus QE.
	\begin{itemize}
		\item Aggregate  Demand:  There are two types of households: liquidity- and non-liquidity-constrained households. They possess the same utility function, non-separable in consumption and leisure with habit persistence in both consumption and leisure. Liquidity-constrained households do not optimize, they just consume their labor income. On the other side, non-liquidity-constrained households have access to domestic and foreign currency denominated assets, accumulate capital subject to investment adjustment costs and rent it to firms, earn profits from owning the firms and pay taxes. Income from foreign financial assets is subject to an external financial intermediation risk premium while real asset holdings are subject to an equity risk premium. Both types of households supply differentiated labor to a trade union which sets the wages by maximizing their joint utility (weighted by the share of each type). The wage setting process is subject to a wage mark-up and to slow adjustments in the real consumption wage. The wage mark-up arises because of wage adjustment costs and the fact that a part of workers index the growth rate of wages to past inflation.
		\item Aggregate  Supply:  The final goods, which are produced from monopolistically competitive firms, are used for household consumption, investment, government consumption and export. These goods are produced with a Cobb-Douglas production function with capital and production workers (labor adjusted for overhead labor) as inputs. These firms face technological and regulatory constraints, restricting their price setting, employment and capacity utilization decisions. The final goods producer maximizes profits subject to these specific adjustment costs (all having convex functional forms) and demand conditions. Investment good producers combine domestic and foreign final goods using a CES aggregator to produce investment goods which
		are sold to non-liquidity-constrained households in a perfectly competitive market.
		\item The Foreign Sector:  Demand behavior is considered the same for the home country and the rest of the world, therefore export demand and import demand are symmetric. Both equations are characterized by a lag structure in relative prices which captures delivery lags. Export firms buy domestic goods, transform them using a linear technology and sell them in the foreign market, charging a mark-up over the domestic prices. The same situation is faced by importer firms. Mark-up fluctuations arise because of price adjustment costs in both sectors. Mark-up equations are given as a function of past and future inflation and are also subject to random shocks.
		\item Assets and QE: QE is modelled as purchases of domestic long-term bonds in exchange for central bank liquidity. Next to physical capital and money, the model features long-term and short-term bonds. The assets are held by the household and long-term and short-term bond holdings are subject to portfolio adjustment costs.
		\item Shocks: The model uses a large number of shocks for estimation. Discussed in the paper is the long-term bond purchases.
		\item Shocks: The model contains a large battery of shocks as in the QUEST III model as well as exogenous long-term bond purchases.
		\item Calibration: The model is calibrated in line with the QUEST III model, which has been estimated on EA data. \end{itemize}
	
	
	\subsection{EA\_QR14: \cite{QR2014}}
	\label{EAQR14}
	\cite{QR2014} use a two-country, two-sector, two-agent DSGE model of the euro area with nominal and financial frictions to study the interactions of monetary and macroprudential policy. The two countries represent the core and the periphery in the EMU; the two sectors capture non-durable and durable goods, the latter being housing goods. It builds on the model presented in \cite{Rabanal2009} by extending it with a financial accelerator mechanism in households' balance sheets. Macroprudential policy aims to stabilize credit markets by affecting the fraction of liabilities banks can lend.
	\begin{itemize}
		\item Aggregate Demand: Households in the core and in the periphery maximize expected lifetime utility by choosing consumption of durable and non-durable goods, and leisure. The composite non-durable consumption good consists of domestic non-durable (i.e., tradable) and foreign non-durable goods. Purchases of durable goods take the form of residential investment. Households in each country can save in deposits and bonds, and can take out one-period loans from domestic competitive financial intermediaries.
		\item Aggregate Supply: Each economy is characterized by two sectors. Monopolistic firms (subject to Calvo-style rigidities) use household labor to produce durable and non-durable intermediate goods that are combined by competitive final-good producers into durable and non-durable goods. Imperfect substitutability of labor is assumed between the two sectors and wages are flexible. Final durable goods are sold only to domestic households, which they use to increase the value of their housing  subject to adjustment costs.
		\item Financial Sector and Macroprudential Policy: Impatient households finance part of their residential investment through loans subject to a contract, analogous to that of \cite{BernankeGertlerGilchrist1999}, where default occurs when the value of their outstanding debt is higher than the value of the house they own (which is common knowledge)  depending on the realization of their idiosyncratic housing quality shock. This induces a spread between the lending (i.e., mortgage) and deposit rates, which depends on housing market conditions. Savers' funds are channeled across countries through international financial intermediaries which trade domestic financial intermediaries' bonds and charge a risk premium, which depends on the net foreign asset position of the country.
		\item	Shocks: Thirteen shocks are present in the model: four sector-specific technology shocks (two for each country), four preference shocks (one for each type of good in each country), two housing quality variance shocks (one for each country), a risk premium shock, and two union-wide shocks (technology and monetary policy).
		\item Calibration/Estimation: The model is estimated by means of Bayesian techniques using quarterly euro area data for the period 1995:Q4 to 2011:Q4. The core country is an aggregate of France and Germany and the periphery is represented by the GDP-weighted average of Greece, Ireland, Italy, Portugal and Spain.
	\end{itemize}
	
	\subsection{EA\_QUEST3: \cite{RattoRoegerVeld2009}}
	\label{EAQUEST3}
	\cite{RattoRoegerVeld2009} develop and estimate an open economy DSGE model for the euro area with emphasis on monetary and fiscal rules, in order to explore their stabilization properties. The role of fiscal policy is explored in an environment with rules for government consumption, investment and transfers and with financial frictions in the form of liquidity-constrained households.
	\begin{itemize}
		
		\item Aggregate Demand: There are two types of households: liquidity- and non-liquidity-constrained households. They posses the same utility function, non-separable in consumption and leisure with habit persistence in both consumption and leisure. Liquidity-constrained households do not optimize, they just consume their labor income. On the other side, non-liquidity-constrained households have access to domestic and foreign currency denominated assets, accumulate capital subject to investment adjustment costs and rent it to firms, earn profits from owning the firms and pay taxes. Income from foreign financial assets is subject to an external financial intermediation risk premium while real asset holdings are subject to an equity risk premium. Both types of households supply differentiated labor to a trade union which sets the wages by maximizing their joint utility (weighted by the share of each type). The wage setting process is subject to a wage mark-up and to slow adjustments in the real consumption wage. The wage mark-up arises because of wage adjustment costs and the fact that a part of workers index the growth rate of wages to past inflation.
		
		\item Aggregate Supply: The final goods, which are produced from monopolistically competitive firms, are used for household
		consumption, investment, government consumption and export. These goods are produced with a Cobb-Douglas production function with capital and production workers (labor adjusted for overhead labor) as inputs. These firms face technological and regulatory constraints, restricting their price setting, employment and capacity utilization decisions. The final goods producer maximizes profits subject to these specific adjustment costs (all having convex functional forms) and demand conditions. Investment good producers combine domestic and foreign final goods using a CES aggregator to produce investment goods which are sold to non-liquidity-constrained households in a perfectly competitive market.
		
		\item The Foreign Sector: Demand behavior is considered the same for the home country and the rest of the world, therefore export demand and import demand are symmetric. Both equations are characterized by a lag structure in relative prices which captures delivery lags. Export firms buy domestic goods, transform them using a linear technology and sell them in the foreign market, charging a mark-up over the domestic prices. The same situation is faced by importer firms. Mark-up fluctuations arise because of price adjustment costs in both sectors. Mark-up equations are given as a function of past and future inflation and are also subject to random shocks.
		
		\item Shocks: A wage mark up shock, a price mark-up shock, a monetary policy shock, a fiscal policy shock, world demand shock, a risk premium shock, a technology shock, an investment shock, a consumption shock, a trade shock, a labor demand shock, a foreign monetary policy shock.
		
		\item Calibration/Estimation: Estimated with Bayesian methods, using quarterly data for the euro area for the period 1981:1--2006:1.
		
		%\item Replication: Check the model in the Modelbase!.
		
	\end{itemize}
	
	
	\subsection{EA\_SR07: Euro Area Model of Sveriges Riksbank, \cite{AdolfsonLaseenLindeVillani2007}}
	\label{EASR07}
	\cite{AdolfsonLaseenLindeVillani2007} develop an open economy DSGE model and estimate it for the Euro area using Bayesian estimation techniques. They analyse the importance of several rigidities and shocks to match the dynamics of an open economy.
	\begin{itemize}
		\item Aggregate Demand: Households maximize lifetime utility subject to a standard budget constraint. Preferences are separable in consumption, labor and real cash holdings. Persistent preference shocks to consumption and labor supply are added to the representative utility function. Internal habit formation is imposed with respect to consumption. Aggregate consumption is specified as a CES function, being composed of domestically produced as well as imported consumption goods. Households rent capital to firms. Capital services can be increased via investment and via an increase in the capital utilization rate, where both options are involved with costs. Total investment in the domestic economy is represented by a CES aggregate consisting of domestic and imported investment goods. Households are assumed to be able to save through acquiring domestic bonds and foreign bonds in addition to holding cash and accumulating physical capital. A premium on foreign bond holdings assures the existence of a well-defined steady state. Households monopolistically supply a differentiated labor service. Wage stickiness is introduced in the form of the Calvo model augmented by partial indexation. \\ Government consumption of the final domestic good is financed via taxes on capital income, labor income, consumption and payroll. Any surplus or deficit is assumed to be carried over as a lump-sum transfer to households.
		\item Aggregate Supply: The final good is produced via a CES aggregator using a continuum of differentiated intermediate goods as inputs. The production of intermediate goods requires homogeneous labor and capital services as inputs and is affected by a unit-root technology shock representing world productivity as well as a domestic technology shock. Fixed costs are imposed such that profits are zero in steady state. Due to working capital, (a fraction of) the wage bill has to be financed in advance of the production process. Price stickiness of intermediate goods is modeled as in the \cite{Calvo1983} model. In addition, partial indexation to the contemporaneous inflation target of the central bank and the previous periods inflation rate is included for those firms that do not receive a Calvo signal in a given period. This results in a hybrid new Keynesian Phillips curve.
		\item Foreign sector: Importing firms are assumed to buy a homogeneous good in the world market and differentiate it to sell it in the domestic market. Similarly, exporting firms buy the homogeneous final consumption good produced in the domestic economy and differentiate it to sell it abroad. Specifically, the differentiated investment and consumption import goods are aggregated in a second step via a CES function, respectively. The same applies to the export goods. Calvo pricing is also assumed for the import and export sector, allowing for incomplete exchange rate pass-through in the short run. The foreign economy is described by an identified VAR model for foreign prices, foreign output and the foreign interest rate.
		\item Shocks: Unit root technology shock, stationary technology shock, investment specific technology shock, asymmetric technology shock, consumption preference shock, labor supply shock, risk premium shock, domestic mark-up shock, imported consumption mark-up shock, imported investment mark-up shock, export mark-up shock, inflation target shock, the common monetary policy shock, shocks to the four different tax rates and a government spending shock which represents the common fiscal policy shock and which we have adjusted so that we achieve a shock size of one percent of GDP.
		%\item Variable dimension: The model is log-linearized around the steady state. Variables are expressed as percentage deviations from steady state.
		\item Calibration/Estimation: The model is estimated using Bayesian estimation techniques for the Euro area using quarterly data from 1970:1--2002:4 in order to match the dynamics of 15 selected variables. According to the authors, they calibrated those parameters that should be weakly identified by the 15 variables used for estimation.
		%\item Replication: We replicated the impulse response functions for annualized quarterly inflation, output, employment and the annualized interest rate to a one standard deviation monetary policy shock in Figure 3 of \cite{AdolfsonLaseenLindeVillani2007}.
		%\item Impulse responses: Figure \ref{img:EA_SR07}.
		
	\end{itemize}
	
	
	
	
	
	\subsection{EA\_SW03: \cite{SmetsWouters2003}}
	\label{EASW03}
	The EA\_SW03 model of \cite{SmetsWouters2003} is a medium-scale closed economy DSGE model with various frictions and estimated for the Euro area with Bayesian techniques.
	
	\begin{itemize}
		\item Aggregate Demand: Households maximize their lifetime utility, where the utility function is separable in consumption, leisure and real money balances, subject to an intertemporal budget constraint. \cite{SmetsWouters2003} include external habit formation to make the consumption response in the model more persistent. Households own firms, rent capital services to firms and decide how much capital to accumulate given certain capital adjustment costs. They additionally hold their financial wealth in the form of cash balances and one-period, state-contingent bonds. Exogenous spending is introduced by a first-order autoregressive process with an iid-normal error term.
		\item Aggregate Supply: The final goods, which are produced under perfect competition, are used for consumption and investment by the households and by the government. The final goods producer maximizes profits subject to a Dixit-Stiglitz aggregator of intermediate goods, which introduces monopolistic competition in the market for intermediate goods and features a constant elasticity of substitution between individual, intermediate goods. A continuum of intermediate firms produce differentiated goods using a production function with Cobb-Douglas technology and fixed costs and sell these goods to the final-goods sector. They decide on labor and capital inputs, and set prices according to the Calvo model. Labor is differentiated over households using the Dixit-Stiglitz aggregator, too, so that there is some monopoly power over wages, which results in an explicit wage equation. Sticky wages \`{a} la Calvo are additionally assumed. The Calvo model in both wage and price setting is augmented by the assumption that prices that can not be freely set, are partially indexed to past inflation rates.
		\item Shocks: Ten orthogonal structural shocks are introduced in the model. Three preference shocks in the utility function: a general shock to preferences, a shock to labor supply and a money demand shock. Two technology shocks: an AR(1) process with an iid shock to the investment cost function and a productivity shock to the production function. Three cost push-shocks: shocks to the wage and price mark-up, which are iid around a constant and a shock to the required rate of return on equity investment. And finally two monetary policy shocks: a persistent shock to the inflation objective and a temporary common monetary policy shock. In addition, the common fiscal policy shock is added in the form of a government spending shock. Since government spending is expressed in output units, we set the coefficient which scales the shock to unity to achieve a shock size of one percent of GDP.
		%\item Variable dimension: The model is log-linearized around the steady state. Variables are expressed in terms of percentage deviations from steady state.
		\item Calibration/Estimation: The model is estimated using Bayesian techniques on quarterly Euro area data.
		The data set used is comprised of seven key macroeconomic variables consisting of real GDP, real consumption, real investment, the GDP deflator, real wages, employment and the nominal interest rate over the period 1970:1--1999:4.
		%\item Replication: We replicated the impulse response functions of annual inflation and the output gap to a 100bps temporary unanticipated rise in the nominal short term rate in the upper panel of Figure 7 of \cite{KuesterWieland2005}.
		
		%\item Impulse responses: Figure \ref{img:EA_SW03}.
		
		%\item Impulse Responses: The first row of figure \ref{img:SW03} shows impulse responses to a one unit monetary policy shock. The second row shows impulse responses to an increase in government consumption of one percent of GDP. The AR(1) coefficient of the shock process is 0.95. In SW03 government spending is expressed in output units. We therefore set the coefficient which scales the shock to unity to achieve a shock size of one percent of GDP.
	\end{itemize}
	
	\subsection{EA\_SWW14: \texorpdfstring{\cite{smets2014warne}}{Smets et al. (2014)}}
	\label{EASWW14}
	\cite{smets2014warne} uses the \cite{gali2012unemployment} model. It is estimated on euro area data using Bayesian estimation techniques.
	
	\begin{itemize}
		
		\item \cite{gali2012unemployment} is based on \cite{SmetsWouters2007} and differs from the latter in the following ways: 
		\begin{itemize}
			
			\item labor decision on the extensive margin (whether to work or not) rather than the intensive margin (how many hours to work), unemployment is included as an observable variable
			\item logarithmic consumption utility, the utility function is separable in consumption and leisure
			\item the error term in the wage equation captures only the wage markup shock and not the preference shock (as in SW07)
			\item Dixit-Stigliz type aggregator functions for aggregate labor demand and aggregate nominal wage (SW07 uses Kimball).
		\end{itemize}
		
		\item Shocks: A total factor productivity shock, a risk premium shock, an investment-specific technology shock, a labor
		supply shock, a wage and a price mark-up shock and two policy shocks: the common fiscal policy shock entering the government
		spending equation and the common monetary policy shock.
		
		\item Estimation: The model is estimated for the euro area with Bayesian techniques for the period 1985:1-2009:4. In addition to SW07, unemployment is used as an observable variable for the estimation of parameters.
		
	\end{itemize}
	
	
	
	
	
	
	
	
	
	\subsection{EA\_VI16bgg and EA\_VI16gk: \texorpdfstring{\cite{villa2016}}{Villa (2016)}}
	\label{EAVI16}
	\cite{villa2016} assesses the empirical relevance of financial frictions in the US and in the Euro Area, where the above versions of the model refer to the Euro Area. It develops a medium-scale closed economy DSGE-model based on \cite{SmetsWouters2007} and two different financial sector extensions of this framework, in particular \cite{bernanke1996financial} and the \cite{GertlerKaradi2011} types.
	
	
	\begin{itemize}
		
		\item EA\_V16bgg model:
		
		\begin{itemize}	
			
			\item Aggregate Demand: Households maximize their lifetime utility, where the utility function is separable in consumption and leisure, subject to an intertemporal budget constraint. In addition, the external habit formation makes the consumption response more persistent. Households own firms, rent capital services to firms and decide how much capital to accumulate given certain capital adjustment costs. They additionally hold their financial wealth in the form of one-period, state-contingent government bonds.
			
			\item Aggregate Supply: Intermediate good firms maximize their profits by choosing factors of production and by signing a financial contract to obtain additional funds from lenders. Since lenders have to pay some auditing costs to observe the idiosyncratic return to capital, an agency problem arises. The financial contract implies external finance premium that depends on the inverse of the firm's leverage ratio. Retailers buy goods from intermediate good firms, differentiate them, and sell them in a monopolistically competitive market according to the Calvo model. The aggregate final good is assembled by perfectly competitive final good firms, and is used for consumption and investment by the households and by the government. The final goods producer maximizes profits subject to a Dixit-Stiglitz aggregator of intermediate goods, which introduces monopolistic competition in the market for intermediate goods and features a constant elasticity of substitution between individual, intermediate goods. Labor is differentiated by a union using the Dixit-Stiglitz aggregator, too, so that there is some monopoly power over wages, which results in an explicit wage equation. Labor packers buy the labor from the unions and resell it to the intermediate goods producer in a perfectly competitive environment. Sticky wages \`{a}  la Calvo are additionally assumed. 
			
		\end{itemize}
		
		\item EA\_V16gk model:
		
		
		\begin{itemize}	
			
			\item Aggregate Demand: The representative household's utility is separable in consumption and leisure and allows for habit formation in consumption. Households postpone their consumption by holding deposits with the financial intermediaries. The amount of deposits is determined in such a way as to guarantee that the bankers' incentive constraint is satisfied. Expected-lifetime utility is maximized by choosing consumption and labor supplied to intermediate firms.
			
			\item Aggregate Supply: Competitive firms produce intermediate goods using labor services and capital. They face adjustment costs for varying their utilization rate of capital. They finace the capital stock with loans from the financial intermediaries and buy it from capital producing firms to which they re-sell it at the end of the period after having used it. Capital producers face investment adjustment costs. The intermediate goods are bought by retail firms, which act under monopolistic competition and face nominal rigidities as in \cite{Calvo1983}. Non-reoptimizing retailers index their prices to the previous period's inflation rate.
			
			\item Banking sector: Banks receive their funds in the form of deposits from households and lend to non-financial firms. A moral hazard/costly enforcement problem constrains the ability of banks to obtain funds from households, while they are able to perfectly monitor firms and enforce contracts.
			
		\end{itemize}
		
		
		\item Shocks: Seven structural shocks: the technology shock, the investment-specific technology shock, the capital quality shock, the price mark-up shock, the wage mark-up shock, and two policy shocks - the common fiscal policy shock entering the government spending equation and the common monetary policy shock.
		
		\item Estimation: The model is estimated for EA and US with Bayesian techniques for the period 1983:Q1-2008:Q3 using seven key macroeconomic variables: real GDP, real investment, real private consumption, hours worked, GDP deflator inflation, real wage, and the nominal short-term interest rate.
		
		\item Replication: The impulse response functions to negative one-standard-deviation shocks were replicated, similar to those in Figures 3-6 in \cite{villa2016}. The variables include output, investment, inflation, net worth and spread.
		
		
	\end{itemize}
	
	
	
	
	
	\section{Estimated/Calibrated Multi-Country Models}
	
	\subsection{DEREA\_GEAR16: \cite{gadatsch2016fiscal}}
	\label{DEREAGEAR16}
	\cite{gadatsch2016fiscal} is a 3-country New-Keynesian model for the German economy with a comprehensive fiscal block. 2 of those countries form a monetary union and have a common monetary policy. 
	\begin{itemize}
	\item Aggregate Demand: A fraction of households are rule-of-thumb consumers and consume the entire income each period. Optimizing households make optimal choices regarding savings in physical capital, as well as national and international (financial) assets and purchases of consumption goods. Household may find a job in the private or in the public sector or stay unemployed.
	
	\item Aggregate Supply: Monopolistic competitors in each region produce a variety of differentiated products and sell these to the home and foreign market. Price and wage setting are both subject to Rotemberg price adjustment costs. The fiscal authority purchases consumption and investment goods produced in the private sector. Public capital stock and public employment improve private-sector productivity. Fiscal authorities finance themselves with distortionary taxes on private consumption, on labor income and on capital returns, lump-sum taxes as well as social security contributions paid by firms. They can also issue new debt.
	
	\item Foreign sector: Both regions of the currency union are modeled in detail. A VAR is set up to depict the rest of the world. There is international trade in goods and assets.
	
	\item Shocks: The model exhibits 41 structural shocks, where all shocks (besides fiscal and monetary policy shocks) follow an AR(1) process. 
	
	\item Calibration/Estimation: The model is estimated using Bayesian estimation techniques using quarterly data of Germany, Euro Area countries, and eight countries for the rest of the world for the period 1999:Q1–2013:Q4.
	\end{itemize}
	
	\subsection{ESREA\_FIMOD12: \cite{stahler2012fimod}}
	\label{ESREAFIMOD12}
	\cite{stahler2012fimod} build a medium-scale model for the Spanish economy. It contains two countries which form a monetary union and features an extensive fiscal sector.
	\begin{itemize}
	\item Aggregate Demand: A fraction of households are non-Ricardians and consume the entire income each period. The remaining households maximize lifetime utility (habit formation) and can hold capital, government debt, and international bonds. Both types of households enjoy utility from government services. Households can be employed in the private or public sector, or are unemployed. 
	
	\item Aggregate Supply: Intermediate goods firms maximize profits subject to Calvo price stickiness. The public capital stock influences firm productivity. Labor firms hire workers from households and sell labor services to intermediate goods firms. The labor market includes search and matching frictions. The government introduces several distortionary taxes and provides public good and capital. It can finance itself with debt and follows some rules for taxes and other variables.
	
	\item Foreign sector: Both regions are modeled in detail. Consumption and investment goods are traded with a home bias.
	
	\item Shocks: The model exhibits a consumption and technology shock, as well as shocks to monetary and fiscal policy.
	
	\item Calibration/Estimation: The model is calibrated for the Spanish economy inside the monetary union.
	\end{itemize}
	
	
	\subsection{G2\_SIGMA08: FRB-SIGMA by \cite{ErcegGuerrieriGust2008}}
	\label{G2SIGMA08}
	The SIGMA model is a medium-scale, open-economy, DSGE model calibrated for the U.S. economy. \cite{ErcegGuerrieriGust2008} in particular take account of the expenditure composition of U.S. trade and analyse the implications for the reactions of trade to shocks compared to standard model specifications.
	
	\begin{itemize}
		\item Aggregate Demand: There are two types of households: households that maximize a utility function separable in consumption, with external habit formation and a preference shock, leisure and real money balances, subject to an intertemporal budget constraint (forward-looking households) and the remainder that simply consume after-tax disposable income (hand-to-mouth households). Households consume, own the firms and accumulate capital, which they rent to the intermediate goods producers. \cite{ErcegGuerrieriGust2008} introduce investment adjustment costs \`{a} la \cite{ChristianoEichenbaumEvans2005}, where it is costly for the households to change the level of gross investment. Households also choose optimal portfolios of financial assets, which include domestic money balances, government bonds, state-contingent domestic bonds and a non-state contingent foreign bond. It is assumed that households in the home country pay an intermediation cost when purchasing foreign bonds, which ensures the stationarity of net foreign assets. Households rent their labor in a monopolistic market to firms, where forward-looking households set their nominal wage in Calvo-style staggered contracts analogous to the price contracts and hand-to-mouth households simply set their wage each period equal to the average wage of the forward-looking households.
		\item Aggregate Supply: Intermediate-goods producers have an identical CES production function and rent capital and labor from competitive factor markets. They sell their goods to final goods producers under monopolistic competition and set prices in Calvo-style staggered contracts. Firms, who don't get a signal to optimize their price in the current period, mechanically adjust their price based on lagged aggregate inflation. Final good producers in the domestic and foreign market assemble the domestic and foreign intermediate goods into a single composite good by a CES production function of the Dixit-Stiglitz form and sell the final good to households in their country. \cite{ErcegGuerrieriGust2008} introduce quadratic import adjustment costs into the final goods aggregator, which are zero in steady state. It is costly for a firm to change its share of imports in a final good relative to their lagged aggregate shares. Thus the import share of consumption or investment goods is relatively unresponsive in the short-run to changes in the relative price of imported goods even while allowing the level of imports to jump costlessly in response to changes in overall consumption or investment demand. Government purchases are assumed to be a constant fraction of output. Government revenue consists of income from capital taxes (net of the depreciation write off), seignorage income and revenue from lump-sum taxes (net of transfers). The government issues bonds to finance the difference between government revenue and expenditure. Lump-sum taxes are adjusted both in response to deviations of the government debt/GDP ratio from a target level and to the change in that ratio.
		
		\item Foreign sector: Local currency pricing is assumed. Intermediate goods producers price their product separately in the home and foreign market leading to an incomplete exchange rate pass-through. \cite{ErcegGuerrieriGust2008} point out, that empirically imports and exports in the U.S. are heavily concentrated, with about 75 percent in capital goods and consumer durables, but the production share of capital goods and consumer durables is very low. To account for this fact in the two-country model they allow the import share in the final good aggregator for investment goods to be higher than the import share in the final good aggregator for consumption goods.
		\item Shocks: Since we have no information about the variances of the shock terms, we set all shock variances equal to zero. The government spending shock of the home country represents the common fiscal policy shock. The common monetary policy shock is added for the home country.
		%\item Variable dimension: percent
		\item Calibration/Estimation: The model is calibrated at a quarterly frequency. Parameters of the original monetary policy rule are estimated using U.S. data from 1983:1--2003:4.
		%\item Replication: We replicated the impulse response functions for real exports, real imports and the exchange rate to a foreign investment demand sock represented by a decline in the foreign capital income tax rate as plotted in Figure 3 (disaggregated trade case) of \cite{ErcegGuerrieriGust2008}.
		%\item Replication: We replicated the impulse responses of real GDP, real investment, real consumption, real exchange rate, real imports and the trade balance after a technology shock, that boosts real GDP by 1 percent in the long-run, using the setup with disaggregated trade.
		
		%\item Impulse responses: Figure \ref{img:G2_SIGMA08}.
		
		%\item Impulse responses: The first row of figure \ref{img:Sigma07} shows impulse responses to a one unit monetary policy shock. Using the Taylor rule, there is no persistence in the model, and hence all variables return back to steady state after one period. The second row shows impulse responses to an increase in government spending by one percent of GDP.
	\end{itemize}
	
	
	\subsection{G3\_CW03: \cite{CoenenWieland2002} G3 countries}
	\label{G3CW03}
	In this model different kinds of nominal rigidities are considered in order to match inflation and output dynamics in the U.S., the Euro area and Japan. Staggered contracts by \cite{Taylor1980} explain best inflation dynamics in the Euro area and Japan and staggered contracts by \cite{FuhrerMoore1995} explain best U.S. inflation dynamics. The authors evaluate the role of the exchange rate for monetary policy and find little gain from direct policy response to exchange rates.
	\begin{itemize}
		\item Aggregate Demand: The open-economy aggregate demand equation relates output to the lagged ex-ante long-term real interest rate and the trade-weighted real exchange rate and additional lags of the output gap. The demand equation is very similar to the G7\_TAY93 model without any sectoral disaggregation. Lagged output terms are supposed to account for habit persistence in consumption as well as adjustment costs and accelerator effects in investment. The lagged interest rate allows for lags in the transmission of monetary policy. The exchange rate influences net exports and thus enters the aggregate demand equation. The long term nominal interest rate is an average of expected future nominal short-term rates. The long-term  real interest rate is determined by the Fisher equation.
		\item Aggregate Supply: For the U.S., relative real wage staggered contracts by \cite{FuhrerMoore1995} are used (see the US\_FM95 model for a detailed exposition). For the Euro area and Japan the nominal wage contracts by \cite{Taylor1980} are used. Note that Taylor contracts, with a maximum contract length exceeding two quarters, result in Phillips curves that explicitly include lagged inflation and lagged output gaps. Thus, the critique that with Taylor contracts inflation persistence is solely driven by output persistence  \citep{FuhrerMoore1995} is mitigated.
		\item Foreign sector: All three countries are modeled explicitly. The Modelbase rule replaces monetary policy for the U.S.. For the Euro area and Japan the original interest rules remain. Foreign output does not affect domestic output directly, but indirectly via the exchange rate in the demand equation. The bilateral exchange rates are determined by UIP conditions.
		%\item Monetary policy rules: the modelbase rule replaces monetary policy for the euro area. For the US and Japan the original interest rules remain: interest rate smoothing parameter $\rho=1$, coefficients on inflation and output are each $0.5$.
		%\item Microeconomic foundation: No explicit microfoundations. However, the authors discuss the advantage of Taylor contracts over Fuhrer-Moore contracts due to possible microfoundations as derived in \cite{ChariKehoeMcGrattan2000}. The original paper (but not the specification in this modelbase) considers also the Calvo model with profit maximizing firms. The authors neglect microfoundations of the demand side due to the following reasoning:
		%\begin{quote}
		%   Although there is an active and rapidly growing literature on closed and open-economy models, which are consistent with optimizing behavior of representative households and firms, these models do not yet seem able to match hump-shaped output dynamics without introducing persistence in unobservables.
		%\end{quote}
		\item Shocks: Contract wage shocks, demand shocks and the common monetary policy shock which is added for the U.S..
		%\item Variable dimension: Original variables are expressed in percent/100. The common Modelbase variables are expressed in percent.
		\item Calibration/Estimation: Euro area data, (fixed GDP weights at PPP rates from the ECB area-wide model database), U.S. data and Japanese data. For the U.S. and Japan OECD's output gap estimates are used. For the Euro area log-linear trends are used to derive potential output. The estimation is robust to different output gap estimations.
		Demand block: GMM estimation where lagged values of output, inflation, interest rates, and real exchange rates are used as instruments. Supply side: simulation-based indirect inference methods. Estimation period: U.S. 1980:1--1998:4, Euro area 1980:1--1998:4 and Japan 1980:1--1997:1.
		%\item Replication: We replicated the impulse response functions to 0.5 percentage points demand shocks in the United States, the Euro Area und Japan plotted in Figure 3 of Coenen, Wieland (2003). Variables include the output gap, annual inflation and the short-term nominal interest rate of the United States, the Euro Area and Japan.
		
		%\item Replication: using the original monetary policy rule for all countries, we generated the impulse responses of aggregate demand, the short-term interest rate and annual inflation to a contract wage shock in the euro area and compared it to the ones generated with the original code using the Anderson-Moore algorithm (in the original model there is no monetary policy shock).
		
		%\item Impulse responses: Figure \ref{img:G3_CW03}.
		
		%\item Impulse responses: figure \ref{img:CW03} shows impulse responses. The first row shows Euro area impulse responses to a one unit monetary policy shock in the Euro area. The second row shows impulse responses to a one unit Euro area demand shock that is added to the IS equation.
	\end{itemize}
	
	\subsection{G7\_TAY93: \cite{Taylor1993a} G7 countries}
	\label{G7TAY93}
	\cite{Taylor1993a} describes an estimated international macroeconomic framework for policy analysis in the G7 countries: USA, Canada, France, Germany, Italy, Japan and the UK. The model consists of 98 equations and a number of identities.
	This model was the first to demonstrate that it is possible to construct, estimate, and simulate large-scale models for real-world policy analysis \citep{Yellen2007}. \cite{Taylor1993a} argues that a multicountry model is appropriate for the evaluation of policy questions like the appropriate mix of fiscal and monetary policy or the choice of an exchange rate policy.
	\begin{itemize}
		\item Aggregate Demand: The IS components are more disaggregated than in the US\_OW98 model. For example, spending on fixed investment is separated into three components: equipment, nonresidential structures, and residential construction. % see the number of shocks to get an overview about the level of disaggregation.
		The specification of these equations is very similar to that of the more aggregated equations in the US\_OW98 model. The aggregate demand components exhibit partial adjustment to their respective equilibrium levels. In G7\_TAY93, imports follow partial adjustment to an equilibrium level that depends on U.S. income and the relative price of imports, while exports display partial adjustment to an equilibrium level that depends on foreign output and the relative price of exports. Uncovered interest rate parity determines each bilateral exchange rate (up to a time-varying risk premium); e.g., the expected one-period-ahead percent change in the DM/U.S.\$ exchange rate equals the current difference between U.S. and German short-term interest rates.
		\item Aggregate Supply: The aggregate wage rate is determined by overlapping wage contracts. In particular, the aggregate wage is defined to be the weighted average of current and three lagged values of the contract wage rate. In contrast to the US\_FM95 model and the US\_OW98 model, G7\_TAY93 follows the specification in \cite{Taylor1980}, where the current nominal contract wage is determined as a weighted average of expected nominal contract wages, adjusted for the expected state of the economy over the life of the contract. This implies less persistence of inflation than in the US\_FM95 and the US\_OW98 model. The aggregate price level is not set as a constant mark-up over the aggregate wage rate as in US\_FM95 and US\_OW98. Prices are set as a mark-up over wage costs and imported input costs. This mark-up varies and prices adjust slowly to changes in costs. Prices follow a backward-looking error-correction specification. Current output price inflation depends positively on its own lagged value, on current wage inflation, and on lagged import price inflation, and responds negatively (with a coefficient of -0.2) to the
		lagged percent deviation of the actual price level from equilibrium. Import prices adjust slowly (error-correction form) to an equilibrium level equal to a constant mark-up over a weighted average of foreign prices converted to dollars. This partial adjustment of import and output prices imposes somewhat more persistence to output price inflation than would result from staggered nominal wages alone.
		\item Foreign sector: G7\_TAY93 features estimated equations for demand components and wages and prices for the other G7 countries at about the level of aggregation of the U.S. sector. Financial capital is mobile across countries.
		%\item Microeconomic foundation: no
		\item Shocks: Interest rate parity shock, term structure shock, durable consumption shock, nondurable consumption shock, services consumption shock, total consumption shock, aggregate consumption shocks for Germany and Italy, for the other countries disaggregated, nonresidential equipment investment shock, nonresidential structures investment shock, residential investment shock, inventory investment shock, fixed investment shock, inventory investment shock, real export shock, real import shock, contract wage shock, cost-push shock, import price shock, export price shock, fiscal policy shock, where we have adjusted the size of the fiscal policy shock for the U.S. - the common fiscal shock - so that a unit shock represents a 1 percent of GNP shock and a monetary policy shock where again the common Modelbase monetary policy shock enters the monetary policy rule for the U.S..
		%\item Variable dimension: Original variables are expressed in percent/100. The common Modelbase variables are expressed in percent.
		\item Calibration/Estimation: The model is estimated with single equation methods on G7 data from 1971--1986.
		%\item Replication: We replicated the impulse response functions for annualized quarterly inflation and the output gap to a 100 basis point innovation to the federal funds rate in Figure 2 of \cite{LevinWielandWilliams2003}.
		
		%\item Impulse responses: Figure \ref{img:G7_TAY93}.
		
		%\item Impulse Responses: the first row of figure \ref{img:TAY93} shows impulse responses to a one unit monetary policy shock. The second row shows impulse responses to a one government spending shock of one percent of GDP. The nominal wage contracts imply a purely forward looking Phillips curve. However, in the TAY93 model inflation exhibits persistence comparable to the MSR04 model with real wage contracts. The reasons is that prices are not set as a constant mark-up over marginal cost. Prices follow a backward-looking error-correction specification and can thus be interpreted as a time-varying mark-up over marginal cost. Hence, even though the contract specification of MSR04 and FM95 was not known at the time of the construction of the TAY93 model, the implied Phillips curve is of a hybrid type due to the additional lag structure. The absolute peak response of inflation and output are even higher than in the MSR04 model. Note that this is not caused by explicitly modeling the foreign sector. Even though one would expect imports to increase as the US\$ appreciates this is not the case. Imports are highly elastic to domestic income and decrease. Exports show a very small reaction. Therefore the higher responses compared to FM95 and MSR04 are caused by the structure of the domestic economy.
	\end{itemize}
	
	
	
	\subsection{GPM6\_IMF13: \cite{Carabenciovetal2013}}
	\label{GPM6IMF13}
	\cite{Carabenciovetal2013} construct and estimate a six region model with both financial and real linkages. The study is the sixth of a series the IMF research agenda in developing a Small Quarterly Global Projection Model (GMP) which consists of small country models integrated into a single global market. The six regions represent the US, the euro area (EA), Japan, Emerging Asia, a five-country block of inflation-targeting Latin American countries and a ``remaining-countries'' group. The three first regions are regrouped under the label G3 and differ from the rest in five ways: (i) they have an unemployment sector; (ii) there is a trend of appreciation of the real exchange rate for the emerging economies; (iii) there is no bank lending tightening variable for non-G3 economies; (iv) G3 economies are assumed to have achieved their inflation-targets; and (v) priors for estimations differ between the two groups. In addition, the model includes financial spillovers not only from the US but from the EA and Japan as well, a global demand shock, a medium-term interest rate, and real exchange rate linkages. Each of the six economies is characterized by a few behavioral equations. 
	\begin{itemize}
		\item Aggregate Demand: The behavioral IS curve relates the output gap to domestic, external and financial-real linkages. The domestic effects consist of past and expected future values of the output gap and of the past value of the medium-term real interest rate. The specification allows for inertia in the system, with complex forward looking elements. The medium-term real interest rate provides the transmission channel between monetary policy action and the real economy. The external effects are driven by the effective real exchange rate gaps and the foreign demand channel. An overvaluation of the currency, i.e. a negative exchange rate gap, has a negative impact on the output gap. Foreign demand captures the spillovers from trade and allows for a direct and a global impact of the foreign output gaps separately. The financial-real linkages capture the bank lending conditions originating from the G3 economies and for each of these countries, tighter lending conditions translate into a negative output gap.  
		\item Aggregate Supply: The Phillips curve expresses inflation as a function of its past and its future value, the lagged output gap, the change in the effective real exchange rate gap of the country and a disturbance term. Backward-looking elements represent the direct and indirect indexation to past inflation, as well as the decision made by price setters who base their expectations on past rates of inflation. The forward-looking element captures the proportion of price setters who have model-consistent inflation expectation. The real effective exchange rate gap is the import weighted real exchange rate gaps of the trading partners, as import prices capture the pass-through from exchange rate movements to the CPI the best.
		
		\item The Policy Rule: is an Inflation-Forecast-Based rule that determines the short-term nominal rate for the G3 countries. It reacts to three quarters ahead inflation following \cite{Orphanides20032}, the real interest rate and the domestic output gap.
		\item The Medium-term Real Interest Rate: is a model-specific variable that enters the Aggregate Demand equation.  It is a function of the current real policy rate, the expected average real policy rate over the coming year, the expected average real policy rate over the next three years, and the expected average real policy rate over the next five years.
		
		\item The Uncovered Interest Parity: serves to link the country models beyond the aggregate demand block. It is augmented for the emerging country models to include a trend component to capture real appreciation of their currencies.
		\item  Unemployment rate: The model specifies a dynamic version of Okun's law for the G3 regions. It links the unemployment rate to its lagged value and the contemporaneous output gap.
		
		\item Shocks: Shocks to aggregate demand, to the bank lending conditions, to inflation, to the short run rate, to the uncovered interest parity and to the unemployment rate enter the G3 models. The model specifies more stochastic shocks than observables to prevent the model from generating systematic forecast errors over extended periods. Thus, the model features a shock to the level and the growth rate of potential output, a shock to the level and the growth rate of the equilibrium rate of unemployment, a shock to the equilibrium real interest rate and finally a shock to the equilibrium real exchange rate in each economy.
		
		\item Calibration/Estimation: The model is estimated with Bayesian techniques. The estimated coefficients of \cite{Carabenciovetal20082} are taken as a starting point, i.e. as given, then each emerging country region is added individually to estimate the region's seven coefficient: the forward and backward looking components on inflation and the coefficient of the output gap in the Phillips Curve; the lagged and forward looking coefficients of output in the Aggregate Demand equation; and all three coefficients in the Policy Rule. The rest of the coefficients are calibrated. The resulting six country regions are put together to estimate three parameters: the coefficient on the spillover activity variable, the coefficient on the financial spillover variable for emerging economies in the output gap equations and finally, the coefficient on the real exchange rate gap terms in the Phillips Curve equation.
		The model uses quarterly data over the period 1999:Q1 to 2010:Q2.
		
	\end{itemize}
	\subsection{EACZ\_GEM03: IMF model of Euro Area and Czech Republic, \cite{LaxtonPesenti2003}}
	\label{EACZGEM03}
	The model is a variant of the IMF's Global Economy Model (GEM) and consists of a small and a large open economy.
	The authors study the effectiveness of Taylor rules and inflation-forecast-based rules in stabilizing variability in output and inflation. They check if policy rules designed for large and relatively closed economies can be adopted by small, trade-dependent countries with less developed financial markets and strong movements in productivity and relative prices and destabilizing exposure to volatile capital flows. In contrast to \cite{LaxtonPesenti2003} we focus on the results for the large open economy (Euro area) rather than the small open economy (Czech Republic).
	\begin{itemize}
		\item Aggregate Demand: Infinitely lived optimizing households; government spending falls exclusively on nontradable goods, both final and intermediate. Households face a transaction cost if they take a position in the foreign bond market.
		\item Aggregate Supply: Monopolistic intermediate goods firms produce nontradeable goods and tradable goods. It exists a distribution sector consisting of perfectly competitive firms. They purchase tradable intermediate goods worldwide (at the producer price) and distribute them to firms producing the final good (at the consumer price). Perfectly competitive final good firms (Dixit-Stiglitz aggregator) use nontradable and tradeable goods and imports as inputs. Households are monopolistic suppliers of labor and wage contracts are subject to adjustment costs. Households own domestic firms, nonreproducable resources and the domestic capital stock. Markets for land and capital are competitive. Capital accumulation is subject to adjustment costs. Labor, capital and land are immobile internationally. Households trade a short-term nominal bond, denominated in foreign currency. All firms exhibit local currency pricing, thus exchange rate pass-through is low.
		%\item Microeconomic foundation: yes
		\item Shocks: Risk premium shock, productivity shock, shock to the investment depreciation rate, shock to the marginal utility of consumption, government absorption shock where the one affecting the large foreign economy represents the common fiscal policy shock, shock to the marginal disutility of labor, preference shifter. We add the common monetary policy shock to the policy rule of the large economy.
		%\item Variable dimension: Original variables are expressed in percent/100. The common Modelbase variables are expressed in percent.
		\item Calibration/Estimation: Calibrated to fit measures of macro-variability of the Euro area (1970:1--2000:4) and Czech Republic (1993:1--2001:4).
		\item Notes: Due to the symmetric setup of the model, we use the same policy rule in both countries.
		%\item Replication: We replicated the standard deviations of annual inflation, the output gap and the first difference of the interest rate under the optimal Taylor rule implied by the loss function specification 2 of \cite{LaxtonPesenti2003} as listed in the second row of Table 4 in their paper.
		%\item Impulse responses: Figure \ref{img:EACZ_GEM03}.
		
		%\item Impulse responses: The first row of figure \ref{img:GEM03} shows impulse responses to a one unit monetary policy shock. Using the Taylor rule, there is no persistence in the model, and hence all variables return back to steady state after one period. The second row shows impulse responses to an increase in government spending by one percent of GDP.
	\end{itemize}
	
	
	
	\subsection{EAES\_RA09: \cite{Rabanal2009}}
	\label{EAESRA09}
	\cite{Rabanal2009} uses a two-country, two-sector DSGE model of a currency union with nominal rigidities to study the sources of persistent inflation differentials between the EMU and one of its member countries, Spain. Moreover, the paper aims at explaining the first moments of the data by introducing time trends for the country- and sector-specific technology shock processes that can give rise to permanent inflation differentials in the model.
	
	\begin{itemize}
		\item Aggregate Demand: Households in Spain and in the rest of EMU have utility functions separable in consumption and leisure and displaying external habit formation in consumption.  The composite consumption good is defined as a CES aggregate consisting of domestic tradable and nontradable, and foreign tradable goods. Preferences are assumed to be the same across countries, but countries differ with respect to the composition of their consumption basket.
		
		\item Aggregate Supply: Each economy is characterized by two sectors. Monopolistic intermediate firms use labor, supplied by the households, as the only input to produce tradable and nontradable goods. They set prices to maximize profits subject to a set of demand equations. Price setting follows a modified version of the Calvo framework with two indexation mechanisms in place that account for the fact that steady state inflation might be non-zero. Across countries the same production technologies are deployed but countries differ in the degree of wage and price stickiness and in the degree of indexation.
		
		\item Foreign sector: \cite{Rabanal2009} models two countries in the European monetary union of unequal size. They produce differentiated tradable goods that are imperfect substitutes of each other, but there is no price discrimination for the same type of good across countries.
		
		\item Shocks: Ten shocks are introduced in the model: sector- and country-specific AR(1) shock processes for the government spending and the technology shock with an Euro Area tradable shock component, and an iid monetary policy shock.
		
		\item Calibration/Estimation: The model is estimated using Bayesian estimation techniques using quarterly euro area data for the period 1996:Q1--2007:Q4.
		
		%\item Note: Due to sector- and country-specific government spending shock, we define the fiscal shock to be the government spending shock in the nontradable sector in the foreign country, here equivalent to the rest of the EMU.
		
		%\item Replication: All impulse response functions for headline and nontradable inflation and the real GDP growth, as appearing in Figure 5 of \cite{Rabanal2009}, have been replicated.
	\end{itemize}
	
	\subsection{EAUS\_NAWM08: \cite{CoenenMcAdamStraub2008}}
	\label{EAUSNAWM08}
	\cite{CoenenMcAdamStraub2008} use a calibrated, two-country version of the New Area-Wide Model developed at the European Central Bank to examine the Euro Area tax structure and the potential benefits and spillovers of a tax reform (reducing labor market distortions). The real effects of fiscal policies are analyzed in an environment with heterogeneous households. Countries in \cite{CoenenMcAdamStraub2008} are symmetric but of different size where the U.S. represents the rest of the world.
	
	\begin{itemize}
		
		\item Aggregate Demand: Only a share of households have access to domestic and international financial markets, accumulates capital and holds money. The other part of households do not have access to financial markets and neither holds capital. They smooth consumption solely by adjusting their money holdings. Both types of households maximize a lifetime utility function with external habit in consumption and supply differentiated labor services with monopoly power in wage setting. Wages are determined in \`{a} la \cite{Calvo1983} fashion. Households that receive permission to re-optimize their wages choose the same wage while the other part follows an indexation scheme, with wages being a geometric average of past changes in the price of the consumption good. Households gross income is subject to a rich taxation structure. They pay taxes on consumption purchases, on wage income, on rental capital income and on dividend income. Furthermore, they pay social security contributions, a lump-sum tax and receive transfers. Purchases of consumption, financial investment in international markets and capital utilization are subject to specific proportional costs.
		
		\item Aggregate Supply: Producers are distinguished between producing tradable and non-tradable goods. The intermediate goods firm produces a single, tradable differentiated good using an increasing-returns-to-scale Cobb-Douglas technology with capital services and labor as inputs. These goods are sold both in domestic and foreign market under monopolistic competition. Price setting is subject to staggered price contracts \`{a} la \cite{Calvo1983}. Firms that receive permission to re-optimize their prices choose the same price (be it for the domestic or for the foreign market) while the other firms follow an indexation scheme, with prices being a geometric average of past changes in the aggregate price indexes. The final goods firms produce three non-tradable final goods: private consumption goods, investment goods and public consumption goods. Final non-tradable private consumption and private investment goods are modeled in an analogous manner. These final goods are assembled with CES technology, combining intermediate domestic and imported foreign goods. Varying the use of imported intermediate goods in the production process is subject to adjustment costs, therefore changes in the relative price of imported goods go unreflected in the short-run. These final goods are sold taking the price as given. On the other side, the public consumption good is a composite of only domestically produced intermediate goods.
		
		\item The Foreign Sector: The demand for imported goods is equal to the sum of the respective demands for intermediate goods for private consumption and investment. These intermediate goods are sold in the home market by the foreign intermediate-good producer. The price of the intermediate good imported from abroad is equal to the price charged by the foreign producer (local currency pricing).
		
		\item Shocks: A government spending shock, a transfer shock, a productivity shock, a monetary policy shock. (Distortionary tax rates on consumption, on dividends, on rental capital income, on labor income and payments on social security contributions are given as exogenous processes but constant).
		
		\item Calibration/Estimation: The model is calibrated to the \cite{SmetsWouters2003} model, with steady-state ratios based on observed data for the euro area and U.S., respectively.
		
		%\item Replication: All impulse responses to different fiscal policy shocks, as appearing in \cite{CoenenMcAdamStraub2008}, have been replicated.
		
	\end{itemize}
	
	
	
	\subsection{EAUS\_NAWM08CTWW13: \cite{CoganTaylorWielandWolters2013}}
	\label{EAUSNAWM08CTWW13}
	
	\cite{CoganTaylorWielandWolters2013} use a version of the EAUS\_NAWM08 model of \cite{CoenenMcAdamStraub2008} to study the fiscal consolidation plan on the U.S.economy. In EAUS\_NAWM08CTWW13, the US economy is calibrated following \cite{CoganCwikTaylorWieland2010}.
	
	\begin{itemize}
		
		\item Aggregate Demand: As in EAUS\_NAWM08.
		
		\item Aggregate Supply: As in EAUS\_NAWM08.
		
		\item The Foreign Sector: As in EAUS\_NAWM08.
		
		\item Shocks: As in EAUS\_NAWM08.
		
		\item Calibration/Estimation: Differently from the EAUS\_NAWM08 model, parameters for the US are calibrated with reference to other estimated models, including the \cite{CoganCwikTaylorWieland2010}.
		
		%\item Replication: All impulse responses to different fiscal policy shocks, as appearing in \cite{CoenenMcAdamStraub2008}, have been replicated.
		
	\end{itemize}
	
	\section{Estimated Models of Other Countries}
	\subsection{BRA\_SAMBA08: \cite{Gouveaetal2008}}
	\label{BRASAMBA08}
	\cite{Gouveaetal2008} build and estimate a small open economy model for the Brazilian economy. The BRA\_SAMBA08 model is developed at the Central Bank of Brazil to provide support for its policy decisions. This version of the model is used as a tool to analyze the response of the Brazilian economy when subject to different shocks.
	
	\begin{itemize}
		
		\item Aggregate Demand: There are two types of households: optimizers and rule-of-thumbers. Both maximize a similar utility function separable in consumption and leisure but subject to different budget constraints. Unlike the optimizers, the rule-of-thumb households do not have access to credit, asset and capital markets. They just consume their wage income. The optimizers have access to domestic and foreign currency denominated bonds, accumulate capital subject to capital adjustment costs, earn from renting the capital and pay taxes. On the other hand, both types of households supply labor in a competitive market.
		
		\item Aggregate Supply: The production sector is comprised of producers and assemblers. Monopolistic competitive firms are the ones producing differentiated goods under a Cobb-Douglas technology with labor, capital services and imported goods as inputs. Following \cite{GaliGertler1999}, only a fraction of firms are allowed to adjust prices optimally ("forward-looking firms"). The remaining firms follow a rule of thumb. The homogeneous final good is assembled by a representative firm using a CES aggregator and is sold in a competitive market. The final good can be used for private consumption, government consumption, investment and exports.
		
		\item The Foreign Sector: The world is assumed to be populated by a continuum of small open economies as in \cite{GaliMonacelli2005}, each of them producing a differentiated good in the global market. The demand for home country's exports is obtained from the aggregation of the demands from foreign countries, expressed in a world currency. The domestic importing firm takes the demand for its goods from the producers' input choices.
		
		\item Shocks: An inflation target shock, a fiscal target shock, a preference shock, a labor supply shock, an investment shock, a foreign investor's risk aversion shock, a country risk premium shock, a technology shock, a monetary policy shock, a fiscal policy shock, a world imports shock, a world inflation shock and a world interest rate shock.
		
		\item Calibration/Estimation: Estimated with Bayesian methods, using quarterly Brazilian data for the period 1999:Q2--2007:Q4.
		
		%\item Replication: (As far as I remember replication was successful).
		
		
	\end{itemize}
	
	\subsection{CA\_BMZ12: \cite{Bailliuetal2012}}
	\label{CABMZ12} 
	\cite{Bailliuetal2012} investigate interactions between monetary and macroprudential policy and examine whether policy makers should respond to financial imbalances. The model is a closed economy that accounts for standard New Keynesian features and has a financial friction along the lines of \cite{BernankeGertlerGilchrist1999} and \cite{ChristensenDib2008}. The model is estimated using Canadian data. The authors show that it is welfare improving to react to financial imbalances. The size of the benefits, however, depends on the nature of the shock.
	\begin{itemize}
		\item Aggregate Demand: The representative household derives utility from consumption and disutility from labor. Accordingly, it maximizes utility subject to its resource constraint. The household purchases consumption goods and a one-period government bond. The household's income consists of labor income, bond payoff and dividends on the equity it owns on retailer firms. 
		\item Aggregate Supply: Entrepreneurs, capital producers and retailers operate in the production sector of the economy. Entrepreneurs borrow from lenders to purchase capital from capital producers and produce intermediate goods. Capital producers combine investment goods and existing capital to produce new capital, subject to quadratic capital adjustment costs.  Retailers operate in a monopolistically competitive environment and are subject to price rigidities \`{a}  la \cite{Calvo1983}. They buy intermediate goods from entrepreneurs and differentiate them at no cost. A Dixit-Stiglitz aggregator combines intermediate goods to form the final good. 
		\item Financial Sector and Macroprudential Policy: The financial friction is modeled along the lines of \cite{BernankeGertlerGilchrist1999}, i.e. there is a costly state verification contract between the entrepreneurs (borrowers) and the lenders. In this model, the contract is set in nominal terms, similarly to \cite{ChristensenDib2008}. The contract implies a negative relationship between borrower's net worth and the funding costs (external finance premium). The macroprudential policy tool is modeled as an exogenous component of the external finance premium. 
		\item Shocks: a financial shock (affecting the external finance premium), a technology shock, a monetary policy shock, a preference shock, and an investment-specific shock.
		\item Calibration/Estimation: The model is estimated by Bayesian techniques on quarterly Canadian data for the sample 1997:Q1- 2009:Q3. The observable time series are: output (excluding government expenditures), investment, the nominal interest rate, inflation and the external finance cost.
	\end{itemize} 
	
	
	\subsection{CA\_LS07: \cite{LubikSchorfheide2007}}
	\label{CALS07}
	\cite{LubikSchorfheide2007} estimate four small-scale open economy DSGE models with Bayesian techniques for Canada, Australia, New Zealand and the UK. The paper studies to what extent central banks respond to exchange rate movements when setting nominal interest rates, finding that the Bank of Canada and the Bank of England do include the nominal exchange rate in their policy rule. The database contains the model for Canada.
	
	\begin{itemize}
		
		\item Aggregate Demand: The model treats the world economy as a continuum of small open economies. The representative household maximizes its utility separable between consumption and leisure subject to its budget constraint. Consumption is a composite of tradable home and foreign goods.
		
		\item Aggregate Supply: Differentiated goods are produced by monopolistic-competitive firms using a linear technology with labor being the only production input. The firms set their prices in a Calvo staggered way. The marginal costs depend positively on the terms of trade and world output.
		
		\item The Foreign Sector: Purchasing power parity and the law of one price hold. There is perfect exchange rate pass-through.  The securities markets are assumed to be complete, and hence international risk sharing in the form of the uncovered interest rate parity is obtained.
		
		\item Shocks: A nominal interest rate shock, a terms of trade shock, a shock to world demand and a shock to the world inflation rate are introduced in the model.
		
		\item Calibration/Estimation: The model is estimated with Bayesian methods using quarterly Canadian data for the period 1983:Q1--2002:Q4.
		
		%\item Replication: The impulse responses in Figure 1 in \cite{LubikSchorfheide2007} are Bayesian. The IRFs for a monetary, a terms of trade, a row output and a row inflation shock have been qualitatively replicated.
		
	\end{itemize}
	
	%\subsection{CA\_ToTEM10: \cite{MurchisonRennison2006}}
	%\label{CAToTEM10}
	%CA\_ToTEM10 represents the 2010 vintage of ToTEM (Terms-of-Trade Economic Model) which is an open-economy, DSGE model developed by \cite{MurchisonRennison2006}. The Bank of Canada uses this model as a tool for policy analysis and projections for the Canadian economy. %CA\_ToTEM10 takes this role from the previously existing Quarterly Projection Model.
	
	%\begin{itemize}
	
	%	\item Aggregate Demand: Households are classified as ``lifetime income'' consumers and ``current income'' consumers, reflecting the fact that not all consumers can access credit markets. Lifetime income consumers smooth their consumption across time through borrowing and saving while ``current income'' consumers consume their current income each period. Lifetime income consumers choose consumption, domestic and foreign bond holdings, labor supply and wages to maximize a utility function non-separable in consumption and leisure subject to a dynamic budget constraint. Both types of households supply differentiated labor services giving them power when negotiating the wages with the domestic producers. However, renegotiation of the wages is allowed only once in six months, on average, and only a constant proportion of wage contracts are renewed every period. The dynamic wage equation is a function of past and expected future wage inflation and an error-correction component.
	%	\item Aggregate Supply: The production sector is comprised of final good producers, an import sector and a commodity sector. Final goods firms produce consumption goods and services, investment goods, and export goods. The production process of these goods is analogous, differing only on the share of imported goods used in production. In this process, first a capital-labor composite is produced using CES technology, which is then combined with a commodity input to produce the domestic good. Final goods then are a combination of the domestic good and the imported good. Through these steps, the firm faces capital adjustment costs, investment adjustment costs and labor adjustment costs. Final goods firms sell their differentiated goods in a monopolistic competitive fashion having power over prices. However, not all firms can re-optimize their prices every period. A share of firms updates prices according to a geometric average of lagged core inflation and expectations of the inflation target. In ToTEM, pricing decisions are considered as strategic complements, where firms have a strong incentive to follow what other firms do. The commodity sector is represented by a domestic firm operating in a competitive market, producing commodities using capital services, labor and land under a CES technology. These raw goods are either sold to a continuum of imperfectly competitive commodity distributors or exported (for the world price of the commodity denominated in Canadian currency). The commodity distributors repackage the commodity goods and sell them to households and to the final goods producers. These distributors face nominal rigidities a la Calvo in price setting, which limits the degree of exchange rate pass-through to consumer prices in the short-run.
	
	%	\item The Foreign Sector: The import sector is represented by firms who buy imported goods in the world market for a given world price (law of one price holds). These goods are sold to domestic firms, which use them as inputs in their respective production functions. Imperfect exchange rate pass-through in the short-run is present as the price of imports is temporarily fixed in the currency of the importing country and because import firms face nominal rigidities a la Calvo when setting prices. As in other sectors, imported goods inflation is a function of past and expected future imported goods inflation and an error-correction component. Export goods firms are part of the final good producers sector as discussed above. They have some degree of market power and therefore face a downward-sloped demand curve (rest of the world demand).
	
	%	\item Shocks: A demand shock, a risk-premium shock, an inflation target shock, a commodity price shock, a technology shock, world demand shock and a price mark-up shock.
	
	%	\item Calibration/Estimation: Calibrated with parametrization chosen to match univariate autocorrelations, bivariate correlations and variances estimated using Canadian data for the period 1980--2004.
	
	%\item Replication: To replicate the model, the 2010 vintage of ToTEM is used. Impulse responses of a consumption demand shock, risk premium shock, inflation target shock and price mark up shock are replicated successfully as in \cite{MurchisonRennison2006}.
	
	%\end{itemize}
	
	
	\subsection{CA\_TOTEM10: \texorpdfstring{\cite{murchison2006rennison}}{Murchinson and Rennison (2006)}}
	\label{CATOTEM10}
	CA$\_$ToTEM10 represents the 2010 vintage of ToTEM (Terms-of-Trade Economic Model) which is an open-economy, DSGE model developed by \cite{murchison2006rennison}. The Bank of Canada uses this model as a tool for policy analysis and projections for the Canadian economy.
	
	
	\begin{itemize}
		
		\item Aggregate Demand: Households are classified as `lifetime income' consumers and `current income' consumers, reflecting the fact that not all consumers can access credit markets. Lifetime income consumers smooth their consumption across time through borrowing and saving while `current income' consumers consume their current income each period. Lifetime income consumers choose consumption, domestic and foreign bond holdings, labor supply and wages to maximize a utility function non-separable in consumption and leisure subject to a dynamic budget constraint. Both types of households supply differentiated labor services giving them power when negotiating the wages with the domestic producers. However, renegotiation of the wages is allowed only once in six months, on average, and only a constant proportion of wage contracts are renewed every period. The dynamic wage equation is a function of past and expected future wage inflation and an error-correction component.
		
		\item Aggregate Supply: The production sector is comprised of final good producers, an import sector and a commodity sector. Final goods firms produce consumption goods and services, investment goods, and export goods. The production process of these goods is analogous, differing only on the share of imported goods used in production. In this process, first a capital-labor composite is produced using CES technology, which is then combined with a commodity input to produce the domestic good. Final goods then are a combination of the domestic good and the imported good. Through these steps, the firm faces capital adjustment costs, investment adjustment costs and labor adjustment costs. Final goods firms sell their differentiated goods in a monopolistic competitive fashion having power over prices. However, not all firms can re-optimize their prices every period. A share of firms updates prices according to a geometric average of lagged core inflation and expectations of the inflation target. In ToTEM, pricing decisions are considered as strategic complements, where firms have a strong incentive to follow what other firms do. The commodity sector is represented by a domestic firm operating in a competitive market, producing commodities using capital services, labor and land under a CES technology. These raw goods are either sold to a continuum of imperfectly competitive commodity distributors or exported (for the world price of the commodity denominated in Canadian currency). The commodity distributors repackage the commodity goods and sell them to households and to the final goods producers. These distributors face nominal rigidities \`{a} la Calvo in price setting, which limits the degree of exchange rate pass-through to consumer prices in the short-run.
		
		\item The Foreign Sector: The import sector is represented by firms who buy imported goods in the world market for a given world price (law of one price holds). These goods are sold to domestic firms, which use them as inputs in their respective production functions. Imperfect exchange rate pass-through in the short-run is present as the price of imports is temporarily fixed in the currency of the importing country and because import firms face nominal rigidities \`{a} la Calvo when setting prices. As in other sectors, imported goods inflation is a function of past and expected future imported goods inflation and an error-correction component. Export goods firms are part of the final good producers sector as discussed above. They have some degree of market power and therefore face a downward-sloped demand curve (rest of the world demand).
		
		
		\item Shocks: A demand shock, a risk-premium shock, an inflation target shock, a commodity price shock, a technology shock, world demand shock and a price mark-up shock.
		
		\item Calibration/Estimation: Calibrated with parametrization chosen to match univariate autocorrelations, bivariate correlations and variances estimated using Canadian data for the period 1980-2004.
		
	\end{itemize}
	
	
	
	
	\subsection{CL\_MS07: \cite{MedinaSoto2007}}
	\label{CLMS07}
	\cite{MedinaSoto2007} develop a small-open economy DSGE model for the Chilean economy. The CL\_MS07 is structurally similar to models developed by \cite{ChristianoEichenbaumEvans2005}, \cite{AltigChristianoEichenbaumLinde2005}, and \cite{SmetsWouters2007}. Still, a richer specification for the production sector and for fiscal policy is designed to account for special characteristics of the Chilean economy.
	
	\begin{itemize}
		\item Aggregate Demand: There are two types of households, Ricardian and non-Ricardian households. The Ricardian type households maximize a utility function separable in consumption, leisure and real money balances subject to their intertemporal budget constraint. They have access to three types of assets, namely money and one-period non-contingent foreign and domestic bonds. Each of these households is a monopolistic supplier of differentiated labour and only a fraction of them can re-optimize their nominal wage. Rigidity \`{a} la Calvo in wage setting follows \cite{ErcegHendersonLevin2000}. Households that cannot re-optimize their wages follow an updating rule considering a geometric weighted average of past CPI inflation and the inflation target. On the other side, the non-Ricardian households do not have access to any of the assets and own no shares in domestic firms. They simply consume the after-tax disposable income and set their wage equal to the average wage of the Ricardian households. The aggregate consumption for both types of households is a composite of a core consumption bundle (domestic and foreign goods, given by a CES aggregator) and oil consumption.
		
		\item Aggregate Supply: The economy is characterized by three types of firms: intermediate tradable-goods producers, import goods retailers and commodity good producers. Intermediate-goods producers have monopoly power and maximize profits by choosing the prices of their differentiated goods subject to the corresponding demands, and the available technology with labor, capital and oil as inputs. Capital is rented to them from a representative firm which accumulates capital and assembles new capital goods subject to investment adjustment costs. Optimal price setting of intermediate-goods producers is subject to \`{a} Calvo probability. Firms that cannot re-optimize their price follow a rule with partial indexation to past inflation and the inflation target. The pricing structure leads to a hybrid New Keynesian Phillips curve. A commodity good producer is introduced in the model to match a particular relevant sector for the Chilean economy, namely the cooper sector. This firm produces a homogeneous commodity good only for export. The production technology follows an exogenous stochastic process that does not require any input. The price of the homogeneous commodity good is determined in the foreign market.
		
		\item Foreign sector: Local currency pricing is introduced through  Calvo price stickiness faced by import goods retailers, which resale foreign goods in the domestic market. This allows for incomplete exchange rate pass-through in the short-run, important for expenditure-switching effects of the exchange rate. A CES technology is used to combine a continuum of differentiated imported varieties to produce a final foreign good, which is consumed by households and used for assembling new capital goods.
		
		\item Shocks: a transitory productivity shock, a permanent productivity shock, a commodity production shock, a labor supply shock, an investment adjustment cost shock, a preference shock, a government expenditure shock, a monetary policy shock, a foreign commodity price shock, a foreign oil price shock, a foreign output shock, a foreign interest shock, a foreign inflation shock and a price of imports shock.
		
		\item Calibration/Estimation: The model is estimated using Chilean quarterly data for the period 1987:1--2005:4.
		
		%\item Replication: We replicated most of the impulse response functions shown in \cite{MedinaSoto2007}, however only when using the original code from the authors, which presents an extended version from the model given in the paper. Among some of these differences we can mention the presence of capital utilization, not discussed in the paper.
		
	\end{itemize}
	
	
	\subsection{FI\_AINO16: \cite{kilponen2016aino}}
	\label{FIAINO16}
	\cite{kilponen2016aino} present the AINO 2.0 model, which is the DSGE model used at the Bank of Finland for forecasting and policy analysis. It is a small open economy model of the Finnish economy within the Euro Area and the rest of the world. The framework includes standard frictions and rigidities as well as a monopolistically competitive banking sector in the spirit of \cite{Geralietal2010}. 
	
	\begin{itemize}
		\item Aggregate Demand: Households maximize their lifetime utility, where the per-period utility function is separable in consumption and labour. They can invest in the domestic capital stock (via capital goods producers), in euro area bonds, rest of the world bonds and domestic bonds. Households supply labour and act as wage setters in monopolistically competitive labour markets. 
		
		\item Aggregate Supply: Production of domestic intermediate goods is subject to a CES production function with time varying mark-up and Harrod-neutral technological progress under monopolistic competition. Final consumption and investment goods are produced by domestic retailers operating under perfect competition, combining both domestic and imported goods. Export goods are produced by separate exports goods producing firms with a CES production function including domestic intermediate goods and imported goods. Domestic intermediate goods and export goods producers are subject to nominal rigidities in the form of \cite{Calvo1983} pricing.
		
		\item Banking sector: The economy is populated by entrepreneurs who rent capital to the domestic intermediate good firms at the beginning of the period and sell the undepreciated capital to capital producers (owned by households) at the end of the period. Entrepreneurs finance the difference between expenditures and net worth from banks. Banks have market power and set rates on loans, subject to adjustment costs.
		
		\item Shocks: Six types of technology shocks, 3 types of domestic mark-up shocks, 4 types of domestic demand shocks (including a standard government consumption shock), 7 foreign/external shocks and 4 financial shocks (among them the euro area interest rate shock).
		
		\item Estimation: The model is estimated using Bayesian methods on 24 observables of Finnish and foreign data, with the sample period being 1995Q2 to 2014Q4.
		
		\item Replication: We simulated the impulse response functions to a productivity shock and a euro area interest rate shock, Figure 6 and Figure 8 in the paper.
		
		\item Implementation: Monetary policy is exogenous in this framework, as it does not explicitly model the euro area economy and associated monetary policy decisions. Hence, the model is implemented without the option to choose among various monetary policy rules. However, the implementation allows to compare the fiscal policy shock in the model to other models.
		
	\end{itemize}
	
	
	\subsection{HK\_FPP11: \cite{FunkePaetzPytlarczyk2011}}
	\label{HKFPP11}
	\cite{FunkePaetzPytlarczyk2011} develop a small open economy DSGE model and estimate it for Hong Kong with Bayesian techniques. The model adopts the perpetual youth approach and allows for wealth effects from the stock market on consumption
	behavior.
	
	\begin{itemize}
		
		\item Aggregate Demand:  The economy consists of an indefinite number of cohorts facing a constant probability of dying each period, which implies a constant expected effective decision horizon of consumers. Given the lifetime uncertainty, agents' consumption pattern is affected by their expected lifetime wealth (in terms of the wealth in stock market), where the stock price is modeled as the discounted sum of future dividends. In this open economy the consumers are free to allocate their consumption between domestic goods and foreign goods, and the intertemporal allocation is characterized by an otherwise conventional Euler equation that captures the impact of stock-price dynamics.
		
		\item Aggregate Supply: Domestic firms act under monopolistic competition and produce consumption goods. Nominal frictions are introduced in the form of Calvo sticky prices. Non-reoptimizing firms index their prices to previous period's domestic producer price inflation.
		
		\item The Foreign Sector: The rest of the world is modeled exogenously. Foreign output affects domestic output through international risk sharing directly, and also indirectly via the terms of trade channel.
		
		\item Shocks: A productivity shock, a foreign demand shock, a cost push shock and a stock-price gap shock.
		
		\item Calibration/Estimation: The model is estimated using Bayesian methods. \cite{FunkePaetzPytlarczyk2011} employ quarterly data on four observables for the sample 1981:Q1--2007:Q3: the real GDP of Hong Kong, the Hang Seng index, the consumer price index of Hong Kong and US GDP. The last series is used as a proxy for foreign demand.
		
		%\item Replication: We replicated the impulse responses to a positive technology shock, a foreign demand shock and a cost-push shock in Figure 5-7 of \cite{FunkePaetzPytlarczyk2011}.
		
	\end{itemize}
	
	\subsection{HK\_FP13: \cite{FunkePaetz2013}}
	\label{HKFP13}
	\cite{FunkePaetz2013} develop a two-agent, two-sector, open-economy DSGE model and estimate it for Hong Kong with Bayesian methods. The model introduces credit market frictions as a form of a binding collateral constraint on borrowers and adopts a fixed exchange-rate regime as monetary policy.
	
	
	\begin{itemize}
		\item Aggregate Demand: Households consists of borrowers and savers. They both obtain utility from consuming non-housing goods and housing and disutility from providing labor. There is habit formation in consumption, both non-housing goods and housing are CES indices of domestically-produced goods and foreign-produced ones. Borrowers are not able to access to the international financial markets and face the collateral constraint linking to the value of housing and the loan-to-value ratio. Savers can purchase both domestic bonds foreign bonds. A symmetric steady state and perfect international risk/sharing are assumed.
		
		\item Aggregate Supply: Each sector has a two-stage structure of production. Perfectly competitive retailers produce final goods by aggregating intermediate goods according to a CES technology, and monopolistically competitive frims produce intermediate goods subject to nominal rigidity \`{a} la Calvo.
		
		\item The Foreign Sector: The rest of the world is modeled exogenously. Foreign output affects domestic output through international risk sharing directly, and also indirectly via the terms of trade channel.
		
		\item Shocks: Sector-specific productivity shocks, housing preference shocks, a loan-to-value shock, a government expenditure shock, sectoral cost push shocks, a foreign consumption shock, a foreign housing shock and shocks on foreign price distortions.
		
		\item Calibration/Estimation: The model is estimated with Bayesian methods using quarterly data for seven macroeconomic variables ranging from 1981:Q1 to 2007:Q3.
		
	\end{itemize}


	\subsection{UK\_SM11: \cite{millard2011estimated}}
\label{UKSM11}
\cite{millard2011estimated} takes the small-scale open economy DSGE model and estimates it for the U.K. with Bayesian techniques. The main complication is that there are non-energy and energy consumption goods, while energy is split into petrol and utilities. Including energy prices should help to explain UK macroeconomic data.
\begin{itemize}
\item Aggregate Demand: The representative household consumes the three final goods: Petrol, utilities, and ‘non-energy’ subject to external habit formation, investment adjustment costs, and variable capital utilization. Each household supplies differentiated labor to firms as monopoly suppliers. Wage setting exhibits nominal wage rigidity and partial indexation of wages to inflation. In addition, households have the option of holding either foreign or domestic bonds, but trading foreign bonds comes at a quadratic cost. 

\item Aggregate Supply: Each of the consumption goods is produced according to a sector-specific production function. Sticky prices in each sector imply sector-specific New Keynesian Phillips Curves. The production functions themselves involve different combinations of five inputs: Labor, capital, imported (non-energy) intermediates, oil, and natural gas. Value-added is produced by combining domestic capital and labor. Non-energy output is produced by a large number of imperfectly competitive firms by combining value added and imports with energy inputs. The energy input is a Leontief bundle of petrol and utilities. Firms producing petrol and gas are engaged in monopolistic competition. The quantity of oil and natural gas used in U.K.’s utilities and petrol production is the sum of the U.K.’s endowment of oil and net trade in oil with the rest of the world. The fiscal authority levies a duty on petrol and consumes some of the non-energy good.

\item Foreign sector: World energy prices are exogenous. Oil and gas prices adjust immediately to the world prices. U.K. import prices take time to adjust to purchasing power parity.

\item Shocks: The model features 12 shocks including world demand shocks for oil and gas.

\item Calibration/Estimation: The model is estimated for the U.K. with Bayesian techniques for the period 1996 Q2 to 2009 Q3 using ten data series.
\end{itemize}
	
	%\subsection{ESP\_MP17, GRC\_MP17, IRL\_MP17, PRT\_MP17: \texorpdfstring{\cite{martin2017philippon}}{Martin and Philippon (2017)}}
	%\label{MP17}
	%\cite{martin2017philippon} build a New Keynesian small open economy model inside a monetary union. They seek to analyse what the Eurozone crisis was caused by - overly expansive fiscal policy, excessive private debt or a sudden stop. They build these three channels into their model and estimate the structural parameters from Eurozone data. They calibrate their model with the resulting parameters and show that it fits the data well. The authors perform counterfactual experiments for 4 Euro area  countries - Spain, Greece, Ireland and Portugal - to analyse if different policies could have changed the evolution of the crisis. The version of the model that is implemented here is the one provided in the authors' code - the only exception is that 3 parameter values in the country spread equation were adjusted as the code differed from the paper there - and is slightly different from the model in the paper.
	%
	%
	%\begin{itemize}
	%	
	%	\item Aggregate Demand: There are patient and impatient households who maximise utility separable in consumption and labour. Impatient households borrow and are subject to a borrowing limit. Patient households own firms and save. The government taxes households and issues bonds. Households consume a basket of domestic and foreign goods. Also, they monopolistically set wages \`a la Calvo.
	%	
	%	\item Aggregate Supply: Differentiated goods are produced by monopolistic firms using a linear technology with labour being the only production input. Firms set their prices \`a la Calvo.
	%	
	%	\item Policy Channels: First, private debt evolution is modelled via an equation for the evolution of the borrowing limit over time (credit cycle). Second, fiscal policy is modelled with a rule for government expenditure and a rule for government transfers. Third, sudden stops are introduced via an equation for a country-specific spread in funding costs (relative to the eurozone average). This also depends on banks' recapitalisation needs.
	%	
	%	\item Shocks: shock to government expenditure, shock to government transfers, shock to the borrowing limit, shock to the funding cost spread, shock to banks' recapitalisation needs, shock to exports, shock to productivity.
	%	
	%	\item Estimation: The authors estimate several parameters of their specification for the three policy channels (using annual data) and use the resulting parameters for calibration. Other values are taken from the literature by the authors. This includes the two country-specific parameters openness and the share of credit-constrained households for the four countries. The model is calibrated to annual data in the paper and the replication file. In the MMB, the calibration is quarterly via adjusting the necessary intertemporal parameters.
	%	
	%	\item Implementation: Monetary policy is exogenous in this framework as the policy decision is not explicitly modelled and all variables are expressed relative to the euro area average.  Hence, the model is implemented without the option to choose among various monetary policy rules. However, the implementation allows to compare the fiscal policy shock in the model to other models.
	%	
	%\end{itemize}
	
	
	
	
	
	
	\newpage
	
	%\section*{References}
	\setlength{\bibsep}{0.4\baselineskip}
	%\newcommand{\bibpreamble}{\vspace{-0.5\baselineskip}}
	\bibliographystyle{elsarticle-harv}
	\bibliography{../dynaremodelbase}
	
	
	
\end{document} 